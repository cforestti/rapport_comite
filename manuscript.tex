\documentclass[12pt]{report}
\usepackage[utf8]{inputenc}
\usepackage[sectionbib]{natbib}
\usepackage[french]{babel}
\usepackage[titletoc]{appendix}
\usepackage{chapterbib}
\usepackage{url}
\usepackage{amsmath}
\usepackage{graphicx}
\usepackage{subcaption}
\usepackage{fancyhdr}
\usepackage{lmodern}
\usepackage{vmargin}
\usepackage[T1]{fontenc}
\usepackage{float}
\usepackage[table,xcdraw]{xcolor}
\usepackage{caption}
\usepackage{hyperref}
\usepackage{setspace}
%\usepackage{filecontents}
\usepackage{calc}
\usepackage{lipsum}
\usepackage{enumitem}
\graphicspath{ {images/} }
\usepackage{xcolor}
\usepackage{rotating}
\usepackage{gensymb}
\usepackage{siunitx}
\usepackage{textcomp}
\setcounter{secnumdepth}{3}
\usepackage{csquotes}
\MakeOuterQuote{"}
\newcommand{\HRule}{\rule{\linewidth}{0.5mm}}


\pagestyle{fancy}
\fancyhf{}
\lhead{\leftmark}
\rfoot{\thepage}

\begin{document}

\begin{titlepage}
    \begin{center}
%\includegraphics[width=0.35\textwidth]{./logo}~\\[1cm]

\textsc{Institut Pascal}\\

\textsc{Sol Solution}\\[1.cm]

\textsc{CIFRE N° 2018/1627}\\[3.cm]


% Title
\HRule  \\[0.5cm]

{ \bfseries Interprétation de l'essai de chargement dynamique en pointe pénétrométrique \\[0.4cm] }

\HRule \\[0.5cm]

\textsc{Caroline F Oliveira}\\[5.cm]


\end{center}


Encadrement:\\
\textsc{Pierre Breul}\\
\textsc{Bastien Chevalier}\\
\textsc{Miguel Angel Benz Navarrete}\\
\textsc{Quoc Anh Tran}

\vfill

% Bottom of the page
\begin{center}
{\large \today}
\end{center}

\end{titlepage}


 
 
%%________________________________ 
\thispagestyle{empty}
\begin{center}
  \textbf{RESUME}
\end{center}
%\hskip2mm
\begin{spacing}{1.}
La caractérisation de sols est une condition indispensable pour le dimensionnement des fondations en génie civil. A niveau mondial, les travaux de caractérisation, dont l’objectif est de fournir les paramètres du sol permettant d’adapter l’ouvrage au terrain, sont basés sur la réalisation d’essais in situ. Au-delà des limites techniques, la plupart de ces techniques ne permettent pas d’avoir une caractérisation fine des sols présents en surface à faible coût. Un pénétromètre dynamique léger instrumenté (Panda$3\up{\textregistered}$) a été développé afin de répondre à ces besoins. L’instrumentation et l’exploitation des signaux de force et d’accélération intégrés à cet appareil permettent d’obtenir une courbe de chargement en pointe du sol pour chaque impact. Celle-ci traduit la réponse effort-enfoncement sous chargement dynamique et riche de renseignements quant au comportement en déformation et en cisaillement des sols. Le caractère dynamique et transitoire de l’essai couplé à la précision et grand nombre de mesures obtenues facilite la mise en place de différentes méthodes d’analyse intégrant différentes approches (essais de choc, analyse temporelle, analyse spectrale, fonction des transfert …). D’une part, lors des premiers instants suivant le choc, cet essai de chargement dynamique en pointe pénétrométrique (DCLT) est assimilable à un essai de choc. L’exploitation des signaux de force et de vitesse entraînée pour ces instants permet d'étudier le comportement à faibles déformations et de déterminer une mesure de vitesse d’ondes du sol. Par ailleurs, le type de sollicitation, la géométrie et les conditions aux limites de l’essai ressemblent à un essai dynamique de pieu. Comme un essai de chargement dynamique de pieux, la réponse du sol en pointe peut être exprimée à l’aide d’un modèle d’interaction. Actuellement, le modèle appliqué à cet essai est le modèle masse-ressort-amortisseur. Toutefois, sa simplicité ne permet pas traduire correctement l’essai ni d’accéder à des paramètres d’usage courant en géotechnique (\emph{e.g.} paramètres effort-déformation). Il s’avère nécessaire appliquer un modèle d’interaction plus réaliste, basé sur des paramètres permettant de caractériser le comportement du sol et adapté à cet essai. Cela constitue le principal objectif de ce travail de thèse. Pour ce faire on s'appuie des résultats expérimentaux en laboratoire et sur la modélisation numérique de l'essai. Malgré le grand nombre de modèles analytiques existants, l'approche numérique reste la méthode la plus réaliste pour modéliser le battage dans le sol. Celle-ci constitue outil important outil afin mieux comprendre l’essai. D’une part, on souhaite vérifier si les hypothèses considérées dans la méthode d’obtention de la vitesse d’ondes sont valables et d’une autre part, le modèle numérique aidera à mettre au point le modèle d’interaction sol-pointe adapté. Un des atouts du modèle numérique est le fait permettre de confronter la réponse statique et dynamique sous des conditions identiques et ainsi d’aider à dissocier la réponde dynamique de la réponse statique. Une fois mis au point, le modèle d’interaction permettra de mieux saisir la composante dynamique dans la réponse globale de l’essai. On pourra ainsi estimer la réponse statique et l’exploiter à l’aide des modèles d’interprétation existants afin de remonter aux paramètres effort-déformation du sol.
\end{spacing}

\textbf{Mots clés:} reconnaissance de sol – pénétromètre dynamique – courbe charge de chargement dynamique en pointe - paramètres rhéologiques de sols - modèle numérique en différences finies.

\pagebreak
  

% Table des matières
\tableofcontents
\pagebreak
\thispagestyle{empty}
\pagebreak
 
 
\chapter*{Notations}
%\addcontentsline{toc}{chapter}{Notations}
\thispagestyle{empty}
\pagebreak


\chapter{Introduction}

La pénétration est utilisée par différentes techniques afin de caractériser le sol en surface. L'essai pénétrométrique statique consiste à pénétrer à une vitesse constante tandis que dans les essais dynamiques, cela se fait par le battage d'une masse. L'essai de pénétration dynamique est la technique \emph{in situ} la plus utilisée au monde. Toutefois, l'analyse des résultats dynamiques demeurent plus complexe que celle des essais statiques \citep{Byun2015}. 

De manière générale, les pénétromètres, statiques ou dynamiques, fournissent une mesure à la rupture. Dans la pratique du dimensionnement géotechnique, celle-ci est appliquée directement ou interprétée afin d'estimer les paramètres de cisaillement (par exemple, l'angle de frottement effectif $\phi'$, pour les sables, la résistance non-drainé $\tau_u$, des argiles) \citep{Robertson1983a, Schnaid2008, Mayne2009}. Toutefois, les pénétromètres, \emph{a priori}, ne permettent pas d'établir la relation effort-déformation au cours des essais.

Dans l’état actuel des connaissances, l’essai de chargement dynamique en pointe pénétrométrique (DCLT, de l’anglais \emph{Dynamic Cone Load Test}) réalisé à l’aide du pénétromètre $Panda 3\up{\textregistered}$\footnote{Désormais noté Panda 3} \citep{Benz-Navarrete2009, Escobar2015} permet de répondre à un grand nombre des besoins (coût, rapidité, haut rendement de mesures). L’instrumentation et l’exploitation des mesures de force et d’accélération au niveau de tête de l’appareil permettent d’estimer les grandeurs cinématiques à l’interface pointe-sol, à savoir : force, vitesse, déplacement.

La réponse globale du sol à l’aide de l’essai DCLT est exprimée sous la forme d’une courbe chargement en fonction de l’enfoncement pour chaque impact (courbe charge enfoncement en pointe $\sigma_p-s_p$). Dans la méthode proposée par \cite{Benz-Navarrete2009} et par \cite{Escobar2015}, l’exploitation de cette courbe s’appuie sur un modèle d’interaction pointe-sol de type Smith, très utilisé dans les essais de chargement dynamique de pieux. 

Il est représenté par une loi élastoplastique parfaite avec une composante visqueuse de type newtonienne basée sur modèle masse-ressort-amortisseur de \cite{Smith1960}. Bien qu'intéressant, ce modèle ne permet pas caractériser le sol à l'aide de paramètres rhéologiques couramment appliqués en géotechnique. Les nombreux avantages des pénétromètres pour la caractérisation des sols in situ ont poussé les chercheurs à essayer d'interpréter les résultats pénétrométriques afin de remonter aux paramètres caractérisant le comportement du sol \citep{Salgado2013}.

Dans ce contexte, l'objectif de ce travail est de mettre au point une méthode d’interprétation de l’essai de chargement dynamique en pointe pénétrométrique (DCLT). Cela afin de remonter aux propriétés intrinsèques du sol caractérisant son comportement (déformation et rupture) sous ce type de sollicitation. On s'intéresse aux paramètres utilisés pour le dimensionnement de ouvrages courants. Pour ce faire, on propose la démarche générale illustrée par le schéma présenté à la fin de ce chapitre (Figure \ref{fig:demarche}). 


\section{Objectif}

L’objectif de ce travail est de mettre au point une méthode d’interprétation de l’essai de chargement dynamique en pointe pénétrométrique (DCLT). Cela afin de remonter aux propriétés intrinsèques du sol caractérisant son comportement (déformation et rupture) sous ce type de sollicitation. 
    \vspace*{0.5cm}

Les objectifs secondaires sont :
\begin{enumerate}
    \item mettre au point un modèle d’interaction pointe-sol plus réaliste permettant d’une part de bien dissocier la composante dynamique et statique depuis la courbe de chargement ; et d’autre part permettant de traduire la non linéarité de l’essais lors du chargement et du déchargement ;
    \item modéliser l’essai DCLT par le biais du code FLAC$^{3D}$ \citep{Itasca2020} permettant de simuler le battage dynamique dans un milieu homogène, purement frottant ;
    \item mettre au point une méthode d’interprétation permettant de remonter au paramètres mécaniques du sol (effort-déformation).
    \item déterminer la vitesse d’ondes en appliquant les approches développées dans le cadre des essais de choc \citep{Aussedat1970, Omidvar2012}. Ceci afin de remonter au module du sol à faibles déformations ;

\end{enumerate}

\section{Axes de travail}

Afin d’aboutir aux objectifs évoqués, la problématique est étudiée suivant trois approches, complémentaires : analytique, expérimentale et numérique.

\begin{itemize}
      
    \item Axe 1 : Approche analytique :
    
    Cette approche consiste à analyser un certain nombre de modèles théoriques appliqués dans le contexte de la pénétration. A partir de ces modèles phénoménologiques, d'interaction sol-pointe, et explicatifs, comme celui de l'expansion de cavité, on propose la démarche pour l'exploitation de la courbe DCLT afin de déterminer les paramètres rhéologiques du sol.

    \item Axe 2 : Étude expérimentale :
    
    Le développement de tout modèle explicatif ou prédictif nécessite des données issues de l’expérimentation. Cela afin d’évaluer et d’ajuster les prédictions faites par le modèle. Dans ce travail, la réalisation des essais au laboratoire (identification, en chambre d'étalonnage, triaxiaux) ont été réalisé afin de disposer des données expérimentales permettant d’adapter, d’ajuster et d’appliquer les modèles les plus pertinents. 
    
    Toutefois, dans le cadre de ce projet, il ne sera pas possible d’analyser un nombre exhaustif de sols surtout à cause des limitations de temps. Les modèles proposées nécessiteront d’être validés \emph{a posteriori} pour une gamme plus large de sols, soit au laboratoire, soit in situ, dans le cadre de futurs travaux.

    \item Axe 3 : Étude numérique :
    
    Le développement d’un modèle numérique permettant de simuler l’essai DCLT en chambre d'étalonnage dans un massif de sol homogène frottant est envisagé. L’objectif principal est de disposer d’un modèle numérique en différences finies permettant d’évaluer l’influence de certains paramètres du sol difficilement mesurables ou maîtrisables expérimentalement. Ce modèle à l’aide du code FLAC$^{3D}$, sera calibré grâce aux résultats des essais réalisés en chambre d'étalonnage pour des sables modèles.

\end{itemize}

%\section{Structure du rapport}

%Le présent rapport qui reprend les principales activités réalisées tout au long de cette deuxième année de thèse, est divisé en trois partie, suivies d’un bilan général.

%Dans un premier temps, on a choisi d'aborder les modèles d'interaction sol-pointe pour les pieux et l'interprétation appliquée à l'essai CLT. Ceux-ci serviront pour améliorer la méthode d'interprétation de l'essai DCLT.

%Dans la deuxième partie de ce rapport, on présente brièvement les principaux essais réalisés au cours de l'année. 

%La troisième partie présente les travaux réalisés autour du modèle numérique.

%On conclue avec un bilan et avec les principales activités prévues pour la troisième année. 


\begin{sidewaysfigure}[ht]
    \begin{center}
    \includegraphics[scale = 1.]{media/ana_demarche2.PNG}
    \caption{Méthodologie d'interprétation de l'essai DCLT (en bleu les essais réalisés dans le cadre de ce travail}
      \label{fig:demarche}
  \end{center}
\end{sidewaysfigure}  
\newpage

\chapter{Approche analytique} 

\section{Modèles analytiques de la réaction en pointe pour les pieux}

Différents travaux se sont intéressés à interpréter les essais dynamiques de pieux. Ceux-ci sont moins contraignants et plus économiques que les essais de chargement statique. Initialement, l'interprétation de ces essais était basée sur des relations empiriques appliquant la conservation d'énergie : l'énergie appliquée aux pieux correspondrait au travail réalisé par la résistance au pieu. Ces relations ont été d'ailleurs adoptées par un certain nombre de normes de calcul (e.g. Engineering News Record, la formule des Hollandais). Plus tard, certains chercheurs ont constaté les limitations de ces relations et la nécessité d'intégrer l'analyse de la propagation des ondes au sein du pieux afin de mieux comprendre le phénomène \citep{Gonin1979,Loukidis2008}.

De plus en plus, les chercheurs tentent d'interpréter ce problème en prenant en compte les concepts physiques et mécaniques de la pénétration dynamiques et de l'interaction sol-pieux. Ces modèles sont connus comme modèles phénoménologiques. La troisième loi de Newton implique que la force exercée par le pieu dans le sol et égale à la force que le sol exerce dans le pieu. Celle est la somme de réaction latérale $R_L$ et la réaction en pointe $R_b$ (Équation \ref{eq:model}). En basant sur la seconde loi de Newton du mouvement, celles-ci sont exprimées comme des polynôme avec différents ordre de dépendance du déplacement w. La relation générique est donnée par l'équation (\ref{eq:model}).

\begin{equation} \label{eq:model}
    R = R_b + R_L
\end{equation}

\begin{equation} \label{eq:model_lateral}
    R_L = R_L w + R_L \dot w + M \ddot w
\end{equation}

\begin{equation} \label{eq:model_pointe}
    R_b = R_b w + C_b \dot w
\end{equation}

Soit $K_i$, $C_i$ et $M_i$ des coefficients du modèle, i=L pour les coefficient latéraux et i=B pour les coefficients en pointe. Les grandeurs cinématiques sont indiquées: vitesse $\dot{w}$ et accélération $\ddot{w}$. L'approche proposée par \cite{Smith1960} constitue une grande avancée à la compréhension du phénomène mais reste très limité. D'autres modèles ont tenté de modéliser le phénomène en reliant les coefficient du modèles aux paramètres caractérisants le sol \citep{Lysmer1965, Randolph2003, Loukidis2008}. Les travaux dans le domaine évoluent graduellement vers une approche plus réaliste où les modèles sont alimentés par des paramètres du pieu (rayon, profondeur) et du sol ausculté.

Les modèles analytiques décrivant la relation contrainte (ou charge) - enfoncement pour les pieux font souvent la différences entre la réaction latérale et la réaction en pointe, même pour certains le mécanisme de réaction est pareille. Selon \cite{Loukidis2008}, l'approche continue souvent appliquée aux modèles analytiques latéraux comme celle proposée par Randolph (1995) n'est pas adaptée à la réponse en pointe, la réaction développée en pointe ne peut pas être dissocié des mécanismes du milieu adjacent à la pointe. En plus, la réaction en pointe, au contraire de la réaction latérale, est un problème à 2 dimensions.

La déformation plastique autour de la pointe s'étend sur une zone qui va de 1 et à 2 fois le diamètre un pieu. Le mécanisme plastique est très proche de celui observé pour la capacité portante d'une semelle. Vu que le pieu possède une pointe plate, on observe la formation d'une zone conique rigide immédiatement au-dessous de la pointe. Le sol situé à l'intérieur de cette région reste dans le domaine élastique et peut-être considéré comme étant un prolongement de la géométrie du pieu. Les déformations plastiques sont observées au-delà de cette région conique. Le sol placé au-delà de la zone plastifiée \citep{Loukidis2008}.

Dans le cas de l'essai DCLT, on suppose la contribution latéral négligeable devant la réaction en pointe du fait de la géométrie de pointe débordante. La réaction totale obtenue dans cet essai est ainsi assimilable à la réaction en pointe $F_b$ des pieux. Par ailleurs, contrairement aux pieux, les résistances liées d'inertie sont négligeables la masse du pénétromètre étant assez petite.

Différents travaux proposent des modèles pour la réaction en pointe de pieux.  essaient d'exprimer la réponse observée pour les essais de chargement statique de pieux en reliant à contrainte (ou la charge) à l'enfoncement, à savoir : \cite{Smith1960}, \cite{Lysmer1965}, Nguyen et al. (1988), \cite{Holeyman1988}, \cite{Naggar1994} et celui proposé plus récemment par \cite{Loukidis2008}.


\subsection{Modèle de \cite{Smith1960}}

Le modèle proposé par \cite{Smith1960} est largement appliqué pour le battage des pieux. Il a été mis au point dans le but de remplacer les formules de battage (basés sur des concepts d’énergie), utilisant le nombre de coups, par une approche numérique plus précise.

Selon \cite{Smith1960}, la réaction du sol dépend uniquement des déplacements et de la vitesse. La réaction développée est proportionnelle au déplacement jusqu'à une limite $R_u$ à laquelle correspond le \emph{quake} Q (ou enfoncement élastique). Le fonçage rapide du sol autour du pieu donne naissance à un amortissement $J_s$ représenté par un coefficient proportionnel à la vitesse de déplacement $\dot{w}$ (Équation \ref{eq:Rd}). La résistance totale en pointe $R_b$ est la somme des résistances statique $R_s$ et dynamique $R_d$ (Équation \ref{eq:Rt}).

Ce modèle a été appliqué par \cite{Benz-Navarrete2009} et \cite{Escobar2015} à l’essai DCLT. Il a permis de caractériser le sol par le biais d’un coefficient d’amortissement $J_s$ et de dissocier la réponse statique $R_s$ de la réponse dynamique $R_d$ dans la résistance totale mesurée. La partie statique de la résistance $R_s$ est obtenue en admettant que lorsque la vitesse d’enfoncement $\dot{w}$ est nulle, et donc l’enfoncement en pointe est maximal, la résistance totale est due uniquement à la partie statique. A cet instant donné, $q_d = R_s$ \citep{Middendorp1992}.

\begin{equation} \label{eq:Rt}
    R_b = R_s + R_d 
\end{equation}

\begin{equation} \label{eq:Rd}
    R_d = R_s J_s \dot{w} 
\end{equation}

\begin{equation} \label{eq:Rd2}
    R_d = R_s J_n \dot{w}^n 
\end{equation}

Bien que largement appliquée dans le domaine des pieux, la solution de Smith présente des grandes limitations. Certains travaux ont montré que l'amortissement radiatif (en anglais radiation damping) est fonction de la raideur du sol et non de la résistance. En plus, des résultats expérimentaux ont montré que la relation entre l'amortissement visqueux et la résistance statique n'est pas du type linéaire mais exponentielle \citep{Coyle1970}. Globalement, les principales limitations identifiées dans ce modèle sont : 

    • la non prise en compte de l’inertie du sol autour de la pointe ; 
    
    • les pertes d’énergie dues aux différents types d’amortissement (radiatif, l’hystérésis et à viscosité) qui ne sont pas considérées de manière isolée, mais combiné sur un seul paramètre ($J_s$) ;

    • la caractérisation du sol sur la base des paramètres d’usage peu courant en géotechnique (\emph{quake} Q, amortissement $J_s$). 

Certains travaux ont tenté de caractériser les sols à partir J et Q ou de trouver des corrélations entre ces paramètres et les paramètres intrinsèques du sol \citep{Charue2004}. Toutefois la grande variabilité de ces paramètres rend cela très compliqué. En effet, il a été montré que ces paramètres ne dépend pas uniquement des paramètres du sol mais aussi à l'énergie de l'essai \citep{Aoki1992}.

Afin de rendre le modèle initialement proposés par \cite{Smith1960} plus réaliste, d'autres modèle ont été proposé par la suite. Ceux-ci cherchent à mettre au point des modèles alimentés par des paramètres ayant une signification physique et pouvant être déterminés en laboratoire.

\subsection{Modèle de \cite{Lysmer1965}}

\cite{Lysmer1965} ont proposé une solution pour le déplacement d'une semelle circulaire rigide située sur un massif semi-infini élastique parfait, homogène et isotrope soumis à un chargement vertical transitoire (type pulse). Dans cette solution, seuls les efforts normaux sont transmis au niveau de l'interface sol-semelle.

Certains travaux se sont intéressés à la détermination des tassements d'une semelle, mais la solution de Lysmer a la particularité d'être valable pour toutes les fréquences de sollicitation, tandis que les solutions précédentes (Sung, Bycroft, Hsieh) s'appliquent pour de faibles fréquences.

La réaction du sol sur la base de la semelle correspond à la somme de la réaction d'un ressort (fonction du déplacement) et de la réaction d'amortisseur (fonction de la vitesse). La raideur ($K_{b,Lys}$) de ce ressort est celle de la solution statique (Équation \ref{eq:KLys}).

\begin{equation} \label{eq:KLys}
    K_{b,Lys} = \frac{4 G R}{1-\nu} 
\end{equation}

Soit G le module de cisaillement du sol en Pascal, R le rayon de la semelle en mètres et $\nu$ le coefficient de Poisson. Le coefficient d'amortissement est donné par l'équation \ref{eq:CLys}.

\begin{equation}  \label{eq:CLys}
    C_{b,Lys} = \frac{3.4 R^2}{1-\nu} \rho  Vs = \frac{3.4 R^2}{1-\nu} \sqrt{\rho G}  
\end{equation}

Soit $\rho$ la masse volumique en kg/m$^3$ et $V_s$ la vitesse d'ondes de cisaillement du sol en m/s. L'amortissement dans le modèle de Lysmer représente l'amortissement radiatif, c'est-à-dire dissipation d'énergie associée la propagation d'ondes (de volume et Rayleigh) dans le milieu semi-infini. Dans ce modèle, l'amortissement ne dépend pas de la fréquence. Le modèle basé sur l'analogie de Lysmer est adapté pour un sol élastique parfait soumis à un chargement vertical transitoire (pulse). La conformité de la solution générale et les résultats pour un milieu semi-infini reste suffisante pour les applications aux problèmes d'ingénierie. 

La solution de Lysmer est largement employée dans le calcul de fondations de machine. Ce modèle suppose un comportement élastique parfait et le sol présente un comportement élastique restreint à des déformations très faibles. Par conséquent, le module G adapté pour déterminer la raideur K correspond au module à faibles déformations ($G_{max}$ ou $G_0$). Certains chercheurs suggèrent l'utilisation du module associé à la gamme de déformation du problème traité, mais la détermination la gamme de déformation associée au problème n'est pas simple (e.g. fonction de la distance de la source, du temps,...). Quant au module permettant de déterminer $C_{b,Lys}$, ceci est strictement le module à faibles déformations ($G_{max}$).

Le modèle de Lysmer a l'atout d'associer le ressort et l'amortissement de Smith à des paramètres courants du sol (paramètres élastiques et masse volumique). Par conséquent, ce modèle réduit l'empirisme du modèle de Smith, par contre il ne tient pas compte des certains aspects importants de la réaction du sol. On peut citer notamment la non-linéarité de la réponse du sol et les effets visqueux dans la résistance. Par conséquent, le modèle surestime la raideur en pointe.

En effet, la solution présentée par Lysmer est valable pour les sollicitations statiques et dynamiques (toutes les fréquences), tant que l'encastrement est nul (D=0) \cite{Loukidis2008}. Cela requiert attention car le plupart des fondations et notamment 
les fondations profondes D>>B, soit B le diamètre du pieu. Pour les fondations encastrées, le coefficient K observé est supérieur à la valeur théorique. La  présente l'évolution du coefficient K pour le cas d'une fondation rigide \citep{Das2011} (Figure \ref{fig:ana_emb}). Pour les applications en profondeur, \cite{Loukidis2008} propose l'intégration d'un facteur afin de tenir compte de l'encastrement. Ce sujet sera abordé plus tard dans ce rapport.

  \begin{figure}[H]
   \begin{center}
        \includegraphics[scale = 1.]{media/ana_emb.png}
        \caption{Effet de l'augmentation de la raideur avec la profondeur \citep{Whitman1969}}
          \label{fig:ana_emb}
      \end{center}
 \end{figure}

Récemment, \cite{Lee2020} a appliqué l'amortissement matériel proposé par \cite{Lysmer1965} afin de dissocier la partie dynamique de la résistance totale mesurée à l'essai au pénétromètre dynamique. Les auteurs ont montré qu'une fois la partie dynamique soustraite de la résistance totale, la résistance déduite était assez proche de la résistance pseudo-statique $q_c$ (CPT) (avec $q_c = 1,01 q_d$ et $R^2=0,85$). 

Plus tard, d'autres travaux comme celui de \cite{Holeyman1988}, de \cite{Naggar1994} et de \cite{Loukidis2008} ont permis de compléter celui proposé par \cite{Lysmer1965}. Ils ont notamment avancé en prenant en compte les différents types d'amortissement (de radiation et d'hystérésis) qui étaient traduits par Smith par un seul paramètre ($J_s$). En autre limitation de la solution de \cite{Lysmer1965} est le fait de ne pas modéliser des milieux hétérogènes ou non-élastiques. \cite{Loukidis2008} a proposé un modèle afin de corriger certaines de ces limitations.

\subsection{Modèle de \cite{Loukidis2008}}

\cite{Loukidis2008} et \cite{Salgado2015} a proposé un modèle d'interaction qui tient compte de la non-linéarité de la réponse en pointe, de l'effet visqueux dans la résistance et qui distingue les différents types d'amortissement (radiatif et d'hystérésis). La réponse totale en pointe $R_b$ est la somme de la réaction du ressort $R_S$ et de l'amortisseur $R_D$ (Figure \ref{fig:ana_modele}).

  \begin{figure}[H]
   \begin{center}
        \includegraphics[scale = 1.1]{media/ana_modele.JPG}
        \caption{Modèle de la réaction en pointe proposé par \cite{Loukidis2008}}
          \label{fig:ana_modele}
      \end{center}
 \end{figure}

\begin{equation}  \label{eq:Rb}
    R_b = R_S+R_D = R_S+C \dot w
\end{equation}

\begin{equation}  \label{eq:Rbs}
   \dot R_S = \frac{K_{max}}{(1+b_{fb}\frac{\mid R_S+LI R_{S,rev}\mid }{(LI+1) x \mid sgn R_{Sf}-R_S\mid})^2} \dot w
\end{equation}

La raideur non-linéaire du ressort en pointe est donnée par l'équation hyperbolique représentant la réponse charge-déplacement en pointe. Cette raideur est réduite à partir de la valeur maximale $K_{max}$ associée à un comportement purement élastique. La réaction en pointe tend asymptotiquement vers la résistance en pointe $R_{bf}$. Lorsqu'on observe un chargement de signal du déplacement, la raideur est $K_{max}$. En effet pour phase de décharge, l'allure de courbe est gouvernée par le module initial \citep{Holeyman1988}.

La courbure de l'hyperbole dépend du type de pieux et du sol. On assimile le pénétromètre à un pieu battu. \cite{Loukidis2008} observe que $b_f$ varie entre 1,0 et 2,0 pour les pieux battus. L'effet visqueux dans la résistance sont pris en compte d'après la relation proposé par \cite{Coyle1970} (Équation \ref{eq:Coyle}).

\begin{equation}  \label{eq:Coyle}
    R_{Sf} = Q_{b,lim} (1 + m v^n) 
\end{equation}

Avec m et n des coefficients contrôlant l'effet de la vitesse. Le coefficient n présente une variation est assez petite quel que soit le sol, et est souvent pris 0,2. Le coefficient m varie entre 0,34 et 0,56 pour les sables et entre 0,95 et 1,55 pour les argiles.

Lorsque que la vitesse de nulle, $R_{Sf}$ est égale $Q_{b,lim}$, soit $Q_{b,lim}$ la résistance limite en pointe. La résistance $Q_{bL}$ est égale au produit résistance unitaire et de la section du pieu. Selon \cite{Salgado2008}, la résistance unitaire en pointe est approximativement la résistance quasi-statique $q_c$ issue de l'essai CPT.

La raideur non-linéaire intègre les pertes d'énergie lié à la plastification (hystérésis du matériau). Le coefficient d'amortissement (C) intègre l'amortissement radiatif et de l'hystérésis du sol plus loin de la pointe réduit graduellement avec l'augmentation lorsque la réaction du ressort s'approche de la résistance en pointe ultime. 

La raideur maximale du ressort ($K_{max}$) et le coefficient de l'amortissement (C) sont basés sur les coefficient de Lysmer ($K_{b,Lys}$, Équation \ref{eq:KLys} et $C_{b,Lys}$, Équation \ref{eq:CLys}). Comme évoqué, la solution de Lysmer s'est basée sur une fondation chargée en surface. En réalité, dans le cas d'une fondation à une profondeur D, celle-ci va transmettre la chargement à un sol placé en profondeur et généralement plus rigide. Par conséquent, cet effet entraîne l'augmentation du coefficient $K_{max}$ \citep{Gazetas1985}. Il est ainsi important d'intégrer un facteur tenant compte de l'encastrement. \cite{Loukidis2008} a analysé cet effet et a introduit le coefficient d'encastrement $D_f$. L'utilisation de ce coefficient permet d'appliquer la solution de Lysmer aux problèmes en profondeur.

Afin d'estimer ce coefficient, \cite{Loukidis2008} ont fait varié la profondeur d'une semelle au sein d'un matériau parfaitement élastique. Ce coefficient nommé $D_f$ est ainsi estimé à partir du rapport entre la profondeur et le diamètre B (Équation \ref{eq:Df}). Ils ont observé également que sous chargement dynamique ce paramètre était élevé à 1,7 (Équation \ref{eq:Df_dyn}). Pour les rapports D/B>8, $D_f$ devient constant, $D_f$ varie entre 1,36, pour $\nu$=0,5 (argiles saturées), et 1,5, pour $\nu$=0,15 (sables).

\begin{equation} \label{eq:Df}
    D_f = \left( 1,27 - 0,12 \ln \nu \right) - \left( 0,27 - 0,12 \ln \nu \right) \ln \nu \left[ \exp{-0,83 \left( \frac{D}{B} \right)^{0,826}} \right]
\end{equation}

\begin{equation} \label{eq:Df_dyn}
    D_{f,dyn} = \left( D_f  \right) ^{1.7}
\end{equation}

Ce modèle a l'avantage d'être alimenté par des paramètres courants caractérisant le sol. Aucun de ces paramètres n'est particulier au problème du battage des pieux. Cela permet donc de réduire le niveau d'empirisme de l'analyse. Les paramètres du modèle sont déterminés à partir des essais en laboratoire ou estimé à partir des données in situ. Pour estimé la réaction en pointe d'un pieu, les paramètres sont : 

1) densité du sol ;

2) module de cisaillement à faibles déformations ($G_{max}$); 

3) coefficient de Poisson ($\nu$);

4) indice de plasticité (IP);

5) rayon du pieu ;

8) résistance unitaire limite en pointe ($q_c$);

9) paramètres visqueux du sol : m et n 

Certain des paramètres (en l'occurrence la densité et l'IP) sont des paramètres classiques aux rapports géotechniques. La résistance statique est déterminée à partir des essais CPT ou estimé à partir de la densité relative et de l'angle de frottement interne.

Le coefficient de Poisson utilisé est celui à faibles déformations. L'obtention de ce paramètre requiert l'utilisation des capteurs de haute précision. \emph{A priori}, ce paramètre n'est pas déterminé. Toutefois, vu que ceci est peu sensible à la densité et au confinement, on peut prendre de valeurs courantes sans avoir de grands impacts sur les résultats. Les valeurs de 0,1 à 0,25 sont courantes pour de sols ayant un degré de saturation $S_r$<70\%, avec des valeurs plus faibles pour les sables. L'intervalle de 0,15 à 0,22 étant une bonne approximation pour les sols sableux ou argileux. Cette valeur augmente considérablement avec le dégrée de saturation supérieur à 70\%. Pour des sols quasiment saturés ($S_r$>95\%) ou saturés, on admet un $\nu$ de 0,5.

Le module à faibles déformations est obtenu en laboratoire par l'essai de colonne résonnante et à partir de capteur piézoélectriques. \emph{In situ}, ceci est déterminé par des essais de cross-hole et down-hole. Ce paramètre est toutefois peu courant dans la plus part des projets géotechniques. C'est pourquoi, ceci est souvent estimé par le biais de corrélations.


%\section{Modèle appliqué à l'essai CLT}

%L'essai CLT a été introduit par \cite{GourvesR.Mirat1977MethodeSols} dans le but de compléter l'essai pénétrométrique classique. L'essai consiste à arrêter l'essai pénétrométrique à différentes profondeurs et à réaliser des chargements à très faible vitesse (de l'ordre de 0,01 mm/s). En mesurant la force et le déplacement lors de ces chargements, il permet d'établir la relation entre les contraintes et déplacements en pointe.

%Différents travaux ont montré l'intérêt de la technique à afin de compléter l'essai classique \citep{Faugeras1979, Arbaoui2003, Ali2010, Reiffsteck2009,Reiffsteck2018}. \cite{Zhou1997} et \cite{Arbaoui2003} proposent une méthode d'interprétation de l'essai afin d'étudier la déformabilité et le paramètres de cisaillement du sol. Bien que très intéressante, la technique n'est pas économique : celle-ci exige l'installation d'un système de réaction, le temps de chargement est assez important. 

%Dans la méthode d'exploitation proposée par \cite{Zhou1997}, la courbe de chargement CLT est exploitée à partir d'un modèle hyperbolique. Deux paramètres principaux sont déduits : le module pénétrométrique ($E_{pn}$) et la résistance ultime ($q_{ult,CLT}$). Le module est calculé à partir de la pente initiale et la résistance ultime correspond à interception entre les tangentes initiale et finale. Par la suite, on s'intéresse à la démarche permettant d'obtenir le module $E_{pn}$ à partir de l'essai CLT.

%Le module de déformation est globalement défini comme la relation entre la contrainte ($\sigma$) et la déformation ($\varepsilon$). Pour l'essai triaxial, la déformation est obtenue directement à partir de la hauteur initiale de l'échantillon et des déplacements mesurés au cours de l'essai. Dans le cas d'un essai de pénétration, son obtention n'est pas directe étant donné que les déformations sont non-homogènes.

 % \begin{equation} \label{eq:1}
  %  E =\frac{\sigma}{\varepsilon}
  %  \end{equation}

%Soit 

 % \begin{equation} \label{eq:2}
 %   E =\frac{\sigma}{\frac{\Delta h}{h_0}}
 %   \end{equation}

%De manière analogue à l'essai triaxial, \cite{Faugeras1979LessaiAnalogique} propose l'obtention du module pénétrométrique ($E_{pn}$) exprimé par : 

 % \begin{equation} \label{eq:E}
 %   E_{pn} =\frac{\sigma}{\frac{\Delta h}{h_{ee}}}
  %  \end{equation}

%Soit

%$\Delta \sigma$ : variation de contrainte

%$\Delta h$ : enfoncement associé à cette variation de contrainte

%$h_{ee}$ : hauteur élastique équivalente (définie par la suite)

%Pour obtenir le module pénétrométrique, \cite{Faugeras1979LessaiAnalogique} introduit la notion de hauteur élastique équivalente ($h_{ee}$) afin de rendre $E_{pn}$ homogène à une contrainte. Cette hauteur correspond à la hauteur équivalente d'un échantillon cylindrique de 10$cm^2$ de section, soumis à la même pression verticale, à une pression latérale égale à la pression des terres au repos à la profondeur de l'essai et qui aurait une déformation de $\Delta$h. Il doit tenir compte de la forme et des dimensions de la plaque (représenté dans ce cas par le cône), de l'encastrement mais aussi du remaniement du sol car, l'essai CLT est en fait un chargement de deuxième cyclique, le premier cyclique ayant été conduit jusqu'à la « rupture » du matériau.

%Pour estimer cette hauteur équivalente, on considère que le cône est assimilable à une plaque circulaire de rayon R dans un milieu élastique semi-infini de module E et coefficient de Poisson $\upsilon$. Par conséquent, l'application de la formule classique de Boussinesq permet de relier le tassement h et à la contrainte $\sigma$.
    
%  \begin{equation}  \label{eq:h}
%    \Delta h =\frac{\pi R}{2} (1-\upsilon^2) \frac{\sigma}{E}
%    \end{equation}
    
%Soit

%  \begin{equation}  \label{eq:E}
%    E =\frac{\pi R}{2} (1-\upsilon^2) \frac{\sigma}{\Delta h}
%    \end{equation}

%La relation entre le module et la pente initiale appliquée par \cite{Arbaoui2003} est la même que celle proposée pour la raideur linéaire du ressort dans l'analogie de \cite{Lysmer1965}. Cela est normal vu que la solution proposée par \cite{Lysmer1965} est valable pour les sollicitations statiques et dynamiques. En effet lorsqu'on écrit cette relation en fonction de la charge (Q) et du module de cisaillement (G) (Équation \ref{eq:G}), on obtient la raideur linéaire K du ressort du modèle de \cite{Lysmer1965}.

%  \begin{equation}  \label{eq:G}
%    G =\frac{E}{2(1-\nu)} 
%    \end{equation}
    
%      \begin{equation}  \label{eq:K}
%    \sigma =\frac{Q}{A} = \frac{Q}{\pi R^2} 
%    \end{equation}

%Comme \cite{Loukidis2008}, \cite{Arbaoui2003} introduit un facteur d'encastrement (ici nommé $k_M$). Pour des grandes profondeurs (D>>B), la valeur retenue est celle proposée par \cite{Mindlin1936} pour un encastrement infini ($k_M$=2). En égalant les expressions \ref{eq:E} et \ref{eq:h}, on obtient la hauteur $h_{ee}$ (Équation \ref{eq:hee}).

%  \begin{equation}  \label{eq:hee}
%    h_{ee} =\frac{\pi R}{2} (1-\upsilon^2) \frac{1}{k_M}
%    \end{equation}

%Le module pénétrométrique est exprimé par l'équation \ref{eq:Epn}. Ceci est déterminé à partir de pente initiale de la courbe CLT.

%  \begin{equation} \label{eq:Epn}
%    E_{pn} = \frac{\pi R}{2} (1-\upsilon^2) \frac{\Delta \sigma}{\Delta h} \frac{1}{k_M}
%    \end{equation}


%  \begin{equation} \label{eq:Epn}
%    \varepsilon = \frac{h}{h_{ee}} 
%    \end{equation}

%\cite{Faugeras1979} a comparé le module pénétrométrique obtenu dans le sol intact (premier cycle) avec celui issu du deuxième cycle. Dans l'essai CLT, on peut considérer que l'installation de la pointe à la profondeur de l'essai correspond à un premier cycle de chargement sur un terrain vierge. L'essai CLT correspondrait ainsi à un deuxième cycle et aux cycles suivants. Le sol est déjà remanié avant un premier chargement l'essai CLT \citep{Zhou1997}.

%\cite{Arbaoui2003} évoque l’utilité de la réalisation des cycles chargement - déchargement - rechargement au lieu d'un simple chargement monotone. On peut évoquer, à l’appui d’un tel essai, la tentative d’effacer, en partie, les effets du remaniement \citep{Combarieu2001}. 

%D’autres essais géotechniques (essai à la plaque, essai déchargement de fondation profonde, essai œdométrique, etc.) comportent dans leur procédure une phase de « déchargement-rechargement » avec exploitation des caractéristiques de déformation mesurées lors de cette boucle. 

%Si l’on considère par exemple le cas du pressiomètre, le principe de la réalisation d’essais cycliques avec cet appareil n’est pas une nouveauté. En effet, Ménard et Rousseau (1962) ont introduit cette procédure (au cours de la dilatation de la sonde, un déchargement suivi d’un rechargement est effectué) et en baptisant le module de déformation pressiométrique mesuré sur le cycle (diagramme ($\sigma$, $\varepsilon$) de « module alterné Ea ». On constate que ceci s'approche du vrai module élastique (\emph{i.e.} à faibles déformations) et qu'il serait ainsi plus adapté pour étudier le comportement des sols sous machines vibrantes, par exemple \citep{Arbaoui2003}.

%\cite{Zhou1997} et plus tard \cite{Arbaoui2003} ont réalisé une série d'essais CLT en laboratoire. Ils ont constaté que le module tangent de déchargement est sensiblement constant. Le module de chargement est, quant à lui, pratiquement constant à partir du 3$^{ème}$ cycle. Le comportement global enregistré correspond à tout ce que l'on connaît des essais cycliques \cite{Zhou1997}. Sur la Figure \ref{fig:ana_zhou_arbaoui_cycle}, on voit les résultats de \cite{Zhou1997} et de \cite{Arbaoui2003}. 

%  \begin{figure}[H]
 %  \begin{center}
 %       \includegraphics[scale = .7]{media/ana_zhou_arbaoui_cycle.JPG}
%        \caption{Résultats des essais CLT réalisés par Zhou (1997) pour le sable d'Allier (a et b) et par Arbaoui (2003) %pour le (c) sable de Fontainebleau et pour (d) le sable d'Allier}
%          \label{fig:ana_zhou_arbaoui_cycle}
%      \end{center}
% \end{figure}

\section{Application des modèles d'interaction à l'essai DCLT} 

On a pu voir l'effort des différents chercheurs afin d'améliorer le modèle de réaction initialement proposé par \cite{Smith1960}. Différents aspects du phénomène de la pénétration sont intégrés afin de mieux modéliser la résistance en pointe.

La courbe DCLT permet d'obtenir la réaction en pointe suite à l'impact. Celle-ci traduit la réponse globale en pointe à la sollicitation dynamique. Par conséquent, elle est qualitativement comparable à une courbe de chargement dynamique en pointe jusqu'à la rupture.

Or, le pénétromètre est assimilable à un pieu isolé dont la réaction se doit à la réaction en pointe. Comme évoqué, les méthodes développées dans le domaine de pieux permettent de relier la réaction en pointe aux paramètres rhéologiques (E, $\phi$, c, ...) du sol ausculté. Il s'avère vérifier dans quelle mesure les relations établies pour les pieux sont valable pour l'essai DCLT et comment les adapter à ce contexte.

Comme on a évoqué, certains paramètres d'entrées sont essentielles pour l'application ces modèles : la résistance ultime et la raideur. Pour l'application à l'essai DCLT certains questions pratiques se posent. Une question importante est celle de la définition de la valeur de résistance à utiliser afin de remonter aux paramètres des rupture. Une autre question incontournable est celle de la définition de la raideur permettant de remonter aux paramètres de rupture.

On va s'intéresser pas la suite à vérifier si les relations permettant de remonter aux paramètres de déformabilité sont valable pour l'essai DCLT. Ceci sous des conditions bien maitrisés et pour un sol homogène élasto-plastique parfait. Pour appliquer la démarche, on a choisi d'exploiter la pente en décharge de la courbe DCLT. La raideur permettant de remonter aux paramètres de déformabilité est définie comme celle entre les points de vitesse nulle et celui où la contrainte est réduite à la moitié.

Dans un deuxième temps, on s'intéresse aux différences définitions de résistance ultime afin de déterminer la résistance ultime de l'essai DCLT.

\subsection{Vérification numérique : application de la raideur de Lysmer et d'encastrement à l'essai DCLT}
 
A l'exception de modèle de Smith, tous les modèles présentés se sont basés sur le coefficient de Lysmer ($K_{b,Lys}$) exprimant le déplacement en pointe en fonction des paramètres de déformabilité du sol ($\nu$, $G$). Pour les utilisation en profondeur, \cite{Loukidis2008} ont proposé de corriger l'effet d'augmentation de $K_{b,Lys}$ à l'aide des coefficients d'encastrement ($D_f$ ou $D_{f,dyn}$).

Même si les différents points communs entre les deux techniques : le type de sollicitation, la géométrie et les conditions aux limites de l’essai ressemblent à un essai dynamique de pieu, l'application de ces relations à un pénétromètre dynamique léger reste limité. Les difficultés liés d'obtention d'une courbe de chargement dynamique in situ rend l'application de ces relations impossible pour la plus part des techniques courants.

Il s'avère important vérifier si les relations mises ont point pour le cas d'une plaque circulaire en surface, et plus tard appliquées aux pieux, sont valables pour l'essai DCLT. Cela pour un milieu comme celui supposé par \cite{Lysmer1965} et par \cite{Loukidis2019} : homogène, isotrope et parfaitement élastique. On souhaite vérifier si pour un tel milieu, il est possible de remonter au module d'élastique en appliquant le coefficient de Lysmer en simulant l'essai DCLT. A la différence de la solution originale, ici on respectera la géométrie du pénétromètre. Cela permettra de vérifier de la solution, développée pour une plaque circulaire rigide en surface, est valable pour la géométrie de l'essai DCLT. Par ailleurs, on vérifie si le coefficient introduit par \cite{Loukidis2008} pour les applications aux pieux est valable pour l'essai DCLT. La Figure \ref{fig:ana_DCLT} compare le trois cas. Le cas supposé originalement par \cite{Lysmer1965}, le cas des pieux utilisant les paramètres de Lysmer avec un coefficient d'encastrement et la cas de l'essai DCLT.

  \begin{figure}[H]
   \begin{center}
        \includegraphics[width = \linewidth ]{media/ana_DCLT.PNG}
        \caption{Schéma comparant entre les différents cas : (a) conditions de la solution initiale proposé par \cite{Lysmer1965}, (b) cas de pieux appliqué à l'aide du coefficient d'encastrement et (c) cas de l'essai DCLT}
          \label{fig:ana_DCLT}
      \end{center}
 \end{figure}

Afin de vérifier cela, on a simulé l'essai en appliquant d'une vitesse d'impact de 3 m/s dans trois milieux homogènes parfaitement élastiques. Les étapes de la mise au point du modèle numérique sont détallé dans le prochain chapitre, consacré à l'étude numérique. Les paramètres physiques et rhéologiques du milieu sont les mêmes dans toutes les simulations, à l'exception du module d'élasticité (E) qu'on a fait varié. On a testé sept valeurs de module allant de 10 MPa à 400 MPa. Les autres paramètres utilisés sont indiqués dans le titre de la Figure \ref{fig:ana_DCLT_Eun}. 

Pour chacun de ces valeurs, la simulation est réalisé de la même façon. On crée le pénétromètre l'intérieur de la cavité d'un massif cylindrique, à une profondeur de 32 cm. Les nœuds de faces latérales des parois de la cavité où le pénétromètre est installé sont fixes. On applique une vitesse de 3 m/s au cylindre représentant le marteau. On enregistre les forces et les vitesses dans la tige, à la proximité de la tête et en pointe. Ceci afin d'avoir les courbes DCLT mesurées et estimées par découplage des ondes. Par la suite, on exploite les courbes numériques obtenues pour les différentes valeurs de E.

Afin obtenir le module d'élasticité à partir de la courbe DCLT, on calcule la pente en phase de décharge. La pente est calculée entre le point où la vitesse en pointe dévient nulle et le point où la contrainte est la moitié de la valeur pour le premier point. A partir de cette pente, on détermine G par la biais de la raideur de ($K_{b,Lys}$) et du coefficient de Poisson $\nu$ du milieu. Le milieu simulé étant parfaitement élastique, le module d'élasticité E peut être obtenu directement à partir de G et de $\nu$ (Équation \ref{eq:E}). Le coefficient de Poisson est constant égal à 0,33 pour toutes les simulations. Comme l'essai DCLT est simulé en profondeur (D>>B). Comme dans le cas de fondations profondes, on doit corriger l'augmentation de ($K_{b,Lys}$) et donc de G en profondeur. Pour ce faire, on applique le coefficient $D_{f,dyn}$ proposé par \cite{Loukidis2008} (Équation \ref{eq:Df_dyn}).
 
\begin{equation} \label{eq:E}
    E = 2 G (1+\nu)
\end{equation}

La Figure \ref{fig:ana_DCLT_Eun} présente les courbes DCLT numériques pour les différentes valeurs de module E testés ainsi que les modules $E_{un,50}$ obtenus à partir des courbes DCLT suivant la démarche décrite.


  \begin{figure}[H]
   \begin{center}
        \includegraphics[width = \linewidth ]{media/ana_DCLT_Eun.PNG}
        \caption{Courbes DCLT numériques pour les valeurs de E testés et les valeurs $E_{un,50}$ obtenus}
          \label{fig:ana_DCLT_Eun}
      \end{center}
 \end{figure}

La Figure \ref{fig:ana_Eun_E} présente la synthèse des résultats obtenus. Elle compare les valeurs de $E_{un,50}$ obtenues en appliquant la raideur de Lysmer et le coefficient d'encastrement aux modules d'élasticité des milieux testés. On constate une bonne correspondance entre les modules d'élastiticté des massifs  On constate que la solution proposé par \cite{Lysmer1965} et le coefficient d'encastrement sont valable pour l'essai DCLT, malgré les différences géométriques entre ces trois cas. Cela permet de montrer pour le cas d'un milieu homogène isotrope parfaitement élastique et en connaissant $\nu$, il est possible de déterminer le module d'élasticité à partir de la courbe DCLT.
 
  \begin{figure}[H]
   \begin{center}
        \includegraphics[width = \linewidth ]{media/ana_DCLT_Eun_E.PNG}
        \caption{Synthèse des résultats numériques : comparaison entre E du modèle rhéologique (élasto-plastique parfait) et $E_{un,50}^{DCLT}$}
          \label{fig:ana_Eun_E}
      \end{center}
 \end{figure}
 
 

\section{Proposition pour l'essai DCLT : intégration de la non-linéarité}


%Les modèles développés plus tard montrent que la raideur du ressort (KLys) et le coefficient d'amortissement (CLys) ne sont pas proportionnel à la résistance limite, mais fonction du module de cisaillement du sol, de la densité du sol et rayon du pieu \citep{Lysmer1965VerticalFooting}, Nguyen et al. (1988), de \cite{Holeyman1988ModellingBase}, Deeks et Randolph (1995)). Quasiment tout le modèle mis au point emploie l'analogie de Lysmer. La solution de \cite{Lysmer1965VerticalFooting} considère une semelle en surface d'un espace semi-infini. Comme le rapport entre la profondeur et le diamètre est assez élevé pour les fondations profondes, la raideur doit être corrigée à l'aide d'un coefficient tenant compte de la profondeur.

%L'effet visqueux dans la résistance est fonction de la vitesse de chargement. Cela est traduit soit par une relation linéaire soit par une relation exponentielle. Par conséquent, pour ces différents modèles, au moment où la vitesse en pointe est nulle, les effets visqueux devrait être négligeable.

%Or, \cite{Loukidis2008AssessmentProcedure} intègre le comportement non-linéaire du sol à partir de la dégradation de la raideur $K_{max}$ à partir d'un modèle hyperbolique.


Dans les différents modèles évoqués, on a vu que la réponse globale correspond à la somme de la réponse statique et dynamique. Par ailleurs, au contraire de ce que \cite{Smith1960} a proposé, les différents amortissements (amortissement radiatif, hystérésis et effet visqueux dans la résistance) sont expliqués par des mécanismes bien différents. L'amortissement radiatif est modélisé comme un amortisseur avec le coefficient qui est fonction des paramètres élastiques du sol ($G_{max}$, $v_p$) \citep{Lysmer1965, Loukidis2008}. L'effet visqueux dans la résistance est fonction de la vitesse de chargement. Cela est traduit soit par une relation linéaire \citep{Smith1960} soit par une relation exponentielle \citep{Coyle1970}.

Dans le cas de l'essai DCLT, les vitesses en pointe étant assez importantes, l'augmentation de la résistance attribuée aux effets visqueux sont non-négligeable. Toutefois, dans cet essai, on permet de connaître les grandeurs (force, vitesse, déplacement) en pointe tout au long du chargement, notamment lorsque la vitesse devient nulle. En ce moment précis, \emph{a priori}, la résistance mesurée se doit uniquement à la réaction statique en pointe. Pour les instants qui suivent, les valeurs de vitesse en pointe étant assez faibles, on suppose que la composante dynamique dans la raideur en phase de décharge est négligeable.

%Les différents travaux relient la raideur de courbe charge-déplacement aux paramètres de déformabilité du sol \citep{Lysmer1965VerticalFooting, Randolph2003ScienceDesign, Loukidis2008AssessmentProcedure, Arbaoui2003MesurePenetrometre}.

Selon les modèles présentés, cette raideur est fonction de paramètres élastiques du sol. Pour \cite{Loukidis2008}, cette raideur varie à partir de la valeur maximale (domaine supposé élastique) avec l'enfoncement et doit être corriger pour les applications en profondeur. Cette raideur maximale initiale correspondrait à la pente en décharge \citep{Holeyman1988, Loukidis2008}.

Néanmoins, on sait que le sol présente un comportement élastique pour des déformations très petites (en général, $\varepsilon<10^{-4}$, \cite{Burland1989}). C'est pourquoi la plupart des techniques classiques ne permettent pas de remonter aux vrais paramètres élastiques ($G_{max}, E_{max}, v_s, v_p$). \emph{A priori}, l'obtention de ces paramètres requiert des méthodes bien spécifiques permettant d'entraîner des déformations suffisamment petites \citep{Burland1989, Atkinson2000}. 

\emph{In situ}, il est spécialement compliqué d'obtenir les paramètres à faibles déformations. Souvent même l'installation de l'équipement ou le procédure avant le chargement entraîne un remaniement non-négligeable empêchant d'accéder à des paramètres associés à de déformation inférieure à 0,01. Étant donné cela, il serait compliqué de remonter aux paramètres élastiques (\emph{i.e.} à petites déformations) à partir d'une courbe de chargement.

En effet, la pente en décharge pourrait permettre d'accéder à un module plus important que le module initial en chargement (notamment pour le premier chargement, \cite{Arbaoui2003}), par conséquent, associé à une gamme de déformation moins importante (effet décrit comme d'effacement du remaniement \citep{Combarieu2001}. Toutefois, difficilement, aux modules à faibles déformations ($G_{max}$ ou $E_{max}$).

Comme évoque par \cite{Loukidis2008}, même si à la rigueur, la solution proposée \cite{Lysmer1965} suppose un comportement parfaitement élastique, dans la pratique la relation qui a proposé aussi appliquée pour de gamme de déformations plus importantes, notamment pour des problèmes dynamiques. Cette approche est celle appliquée dans le cadre d'autres travaux \citep{Zhou1997, Arbaoui2003, Reiffsteck2009} afin de remonter à un module pénétromètre.

En supposant que pour cette phase, la raideur correspond à la valeur maximale ${K_{max}}$ et que la contribution des composantes visqueuses sont négligeables (la vitesse étant faible pour cette phase de l'essai), on propose d'exploiter la phase en décharge de la courbe DCLT afin de remonter à deux modules. 

Pour ce faire, on applique l'analogie de Lysmer afin de définir deux modules en phase de décharge. Les déformations associées à chacune de ces valeurs de module sont définies à partir la définition du module et la déduction de la hauteur élastique équivalente (Équation \ref{eq:E} et Équation \ref{eq:hee}). Suivant cette démarche, on définie deux modules à partir de deux raideurs déduites de la courbe DCLT. La première raideur correspond à la pente entre le point A (moment où la vitesse est nulle) et le point C (moment où la vitesse est minimale), comme présenté sur la Figure \ref{fig:ana_Eun3}. La deuxième raideur correspond à la pente entre le point A et le point B (moment où la résistance vaut la moitié de la valeur du point A).

  \begin{figure}[H]
   \begin{center}
        \includegraphics[width = \linewidth]{media/ana_methodeDCLT.PNG}
        \caption{Interprétation de la courbe DCLT : détermination des modules $E_{un}^{DCLT}$, à partir de la pente entre le point A et le point B, et du module $E_{un,50}^{DCLT}$ à partir de la pente entre le point A et le point C}
          \label{fig:ana_Eun}
      \end{center}
 \end{figure}

Par la suite, on propose d'évaluer ses modules $E_{un}^{DCLT}$ et $E_{un,50}^{DCLT}$ à partir de résultats triaxiaux. Ces deux valeurs de module permettent d'avoir des mesures à deux niveaux de déformation et ainsi de mieux situer les mesures issues DCLT. Ces valeurs sont par la suite comparés aux modules sécants issus des essais triaxiaux pour des densités et des niveaux de contrainte du même ordre. Cela permettra de valider la pertinence de cette méthode pour les matériaux testés sur la base d'une méthode classique d'usage courant en géotechnique. 


%On admet que le module à un niveau de déformation de 0,1\% correspond à un module initial triaxial car ceci est le niveau de déformation plus petit associé à l'essai triaxial classique (statique) \citep{Obrzud2018, Benz2007}. Le module sécant à 50\% du pic de résistance au cisaillement ($E_{50}$) permet d'avoir une deuxième mesure de référence associé à niveau de déformation plus important. Par ailleurs, ce module connaît de nos jours une utilisation de plus en plus répandue dans la modélisation du comportement des sols par éléments finis \citep{Schanz1999, Cami2017}.

Par la suite, on propose d'estimer l'intégralité de la courbe de dégradation du module à partir de l'essai. Pour ce faire, on applique un modèle hyperbolique simple \citep{Fahey1993} (Équation \ref{eq:Esec}).

\begin{equation} \label{eq:Esec}
    E = \frac{E_{max}}{1+f (\varepsilon) ^ g} 
\end{equation}

Le module $E_{max}$ est obtenu, de manière similaire aux autres techniques in situ (\emph{e.g.} sDMT\footnote{Dilatomètre sismique}, sCPT\footnote{Sismo-cône}, \emph{cross-hole}) : à partir de la vitesse d'ondes $v_{p}$ et de la masse volumique du milieu $\rho$ (Équation \ref{eq:Emax_}). 

\begin{equation} \label{eq:Emax_}
    E_{max} = v_p^2 \rho 
\end{equation}

Souvent, on n'a pas d'information sur la masse volumique \emph{in situ}. Dans la pratique, celle-ci est souvent supposé à 1800 $kg/m^3$ \citep{Burns2007} ou encore estimée à partir les valeurs indicatives selon le type de matériau (sable, limon ou argile) \citep{Plumelle2013}. Pour appliquer cette relation au sCPT, par exemple, ce paramètre est estimé à partir de la résistance de pointe ($q_c$). Ceci car la résistance de pointe est fortement corrélée à la masse volumique \citep{Burns2007, Robertson2010}. 

Dans le cas de l'essai DCLT, différents travaux ont étudié la relation entre les paramètre d'état (de densité et hydrique) et résistance dynamique $q_d$ \citep{Zhou1997, Chaigneau2001, Morvan2016}. \cite{Chaigneau2001} a démontré que pour un sol donné et pour un l’état hydrique connu, il existe une relation biunivoque entre $q_d$ et la densité sèche $\gamma_d$. Cette relation du type logarithmique a été établie en laboratoire différents type de sol.

Le module $E_{max}^{DCLT}$ estimé, ceci sert à alimenter le modèle (Équation \ref{eq:Esec}) comme la valeur limite du module. Les modules déduits de la phase de décharge à partir de la courbe DCLT ($E_{un}^{DCLT}$ et  $E_{un,50}^{DCLT}$) fournissent des mesures de déformabilité dans la gamme de grandes ou moyennes déformations. La Figure \ref{fig:ana_degradation_pressio}b illustre la démarche proposée.

Canépa et al. (2002) a appliqué une approche similaire au pressio-pénétromètre. Les auteurs ont estimé la dégradation du module de cisaillement en ajustant le modèle hyperbolique (Équation \ref{eq:Esec}) à l'aide des modules différents. Un module est obtenu à partir d'un cycle de faible amplitude et l'autre, nettement plus faible, est issu du pressiomètre. Sauf, que pour appliquer cette démarche, les auteurs devaient connaître $G_{max}$. Ceci est donc obtenue par essais de propagation d’ondes au sCPT \citep{Cami2017}. La Figure \ref{fig:ana_degradation_pressio}b illustre la démarche proposée par l'auteur.

  \begin{figure}[H]
   \begin{center}
        \includegraphics[scale = .8]{media/ana_degradation_pressio.PNG}
        \caption{(a) Courbe de dégradation illustrant la méthode proposée et (b) Courbe normalisée obtenue par \cite{Canepa2002} avec le pressio-pénétromètre (coefficient d'ajustement : f=82.5, g=0.69)}
          \label{fig:ana_degradation_pressio}
      \end{center}
 \end{figure}

Il s'avère nécessaire d'évaluer la pertinence de la courbe estimée. Pour ce faire, on propose des confronter les courbes obtenues à partir de cette démarche aux résultats rapportés dans la littérature pour les sables d'Hostun et de Fontainebleau \citep{Duttine2005}.

%\cite{Reiffsteck2009MeasurementsPenetrometer} a comparé les modules sécants CLT issus de cette démarche à de différents modules obtenus à partir d'autres méthodes. Pour des essais in situ réalisés, \cite{Reiffsteck2009MeasurementsPenetrometer} n'a pas pu accéder à de gammes de déformation inférieure à 1\%.


\subsection{Résistance ultime}

Dans tous les modèles d'interaction présentés, la résistance ultime est un paramètre essentielle afin d'estimer la réaction en pointe. L'essai DCLT comme d'autres essais pénétrométriques est capable d'évaluer la résistance du sol ausculté. Toutefois, contrairement à d'autres pénétromètres (i.e. SPT ou CPT) fournissant uniquement une valeur de résistance, ceci fournit une courbe de chargement en pointe. Bien que plus complet, ce résultat soulève la question de la définition de la résistance ultime associée à l'essai.

Or, le caractère dynamique transitoire du chargement rend plus complexe la définition de cette valeur de résistance. Cela car la composante dynamique présente dans le résultat global obtenue n'est pas négligeable et est fonction, entres autres, de la vitesse de pénétration. Dans le cas d'un pieu isolé, la résistance ultime total $Q_{ult}$ est la somme des résistances latérales et en pointe.

\begin{equation} \label{eq:Qult}
   Q_{ult} = Q_b + Q_L
\end{equation}


\begin{equation} \label{eq:Qbult}
   Q_{b,ult} = q_{b,ult} A_b
\end{equation}

Avec $q_{b,ult}$ la résistance de pointe unitaire et $A_b$ la section transversale du pieu. Pour le calcul de $Q_L$, la partie du pieu pénétré dans le sol est divisé en couches.

\begin{equation} \label{eq:Qbult}
   Q_{L} = \sum_{i=1}^{n} q_{L,i} A_{L,i}
\end{equation}

Soit $q_{L,i}$ la résistance latéral unitaire pour la $n_{ème}$ couche et $A_{L,i}$ l'aire latérale correspondante. Les couches sont établie généralement à partir de résultats pour le terrain. 

La résistance limite est différente de résistance ultime. Les résistances, de pointe et latérale, augmentent en profondeur mais le taux d'augmentation diminue avec la profondeur. A partir d'un certain déplacement, la résistance atteint la valeur maximale. Cette valeur maximale est la résistance limite. Le déplacement nécessaire afin d'atteindre cette résistance limite pour la résistance latérale est relativement petit (de l'ordre de 0,25-1,0\% du diamètre). Par conséquent, il est courant d'admettre que $Q_L$ soit complètement mobilisé à l'état ultime de dimensionnement, à l'exception de pieu dont la contribution latérale est très importante \citep{Franke1993}.

Contrairement à la résistance latérale, la résistance de pointe limite est $Q_{b,lim}$ est mobilisé uniquement pour des déplacements importants, parfois supérieurs à 20\% du diamètre pour des sables. Dans le cas de chargement des pieux, on peut atteindre la résistance ultime (bearing capacity failure) dans les argiles compressibles et sables lâches, pour les autres matériaux les équipements courants ne permettent pas d'atteindre cette valeur. Pour la plus part, il est impossible d'atteindre $Q_{b,lim}$ car cela demande des déplacements importants exigeant l'utilisation d'un système de réaction non-viable économiquement. La Figure \ref{fig:ana_qult} illustre l'évolution de la réaction latérale et en pointe.


\begin{figure}[H]
  \begin{center}
       \includegraphics[width = \linewidth]{media/ana_qult.png}
        \caption{Évolution des résistances de pointe (base resistance) et latérale (shaft resistance) pour un pieu isolé sous chargement statique}
          \label{fig:ana_qult}
      \end{center}
\end{figure}

Les structures courantes aurait dépassait leur l'état ultime avant que le déplacement nécessaire pour atteindre $Q_{b,lim}$. Dû aux difficultés techniques de déterminer $Q_{b,lim}$ et aux déplacements excessives qui finirait aussi pour entrainer l'état ultime, $Q_{b,ult}$ est plus utilisé que $Q_{b,lim}$. Les critères pour déterminer $Q_{b,ult}$ sont souvent fixé afin d'éviter l'endommagement des structures plutôt que sur la valeur $Q_{b,lim}$. Plusieurs de critère existent. Ceux-ci sont souvent établi à la partir de la courbe charge-enfoncement. Une partie se sont basés sur l'extrapolation de la courbe afin d'estimer la valeur de $Q_{b,lim}$ (\cite{Chin1970}, \cite{VanderVeen1953}) et une autre partie est associé à un enfoncement donnée (\cite{Davisson1972, Terzaghi1943}). Un autre critère est celui introduit par \cite{DeBeer1968} qui définit la valeur $Q_{b,ult}$ comme la charge au moment où la courbure de courbe assume la valeur maximale. Ce point est généralement identifié en traçant la courbe charge-enfoncement en graphique double-logarithmique afin de mettre en évidence de moment où la courbure est maximale.


Pour les applications courantes, il est raisonnable d'appliquer des critères basés sur un enfoncement car pour la plupart de structure, l'état limite ultime est atteinte bien avant la $Q_{b,lim}$. Le critère basé sur l'enfoncement le plus répandu internationalement est celui définissant $Q_{b,ult}$ comme la charge à 10\% du diamètre proposé par \cite{Terzaghi1943}. Un autre critère très répandu, notamment aux États-Unis est celui de proposé par \cite{Davisson1972}. Pour les applications courantes les critères basées associés à un enfoncement données sont plus applicables car les ceux utilisant $Q_{b,lim}$ fournissent de valeurs importantes pour lesquels la plus part des structures aurait déjà dépassait l'état ultime.


%Des résultats expérimentaux et analytiques ont montré que la résistance de pointe limite $Q_{b,lim}$ est approximativement la résistance pseudo-statique issue du CPT \citep{Lee1999, Salgado2008}.

Ce travail porte sur l'obtention des paramètres de rupture et de déformation à partir de courbe DCLT. Il n'est pas question ici de mettre au point une méthode dimensionnement des pieux à partir la courbe. Le but n'étant pas celui de définir ou appliquer une critère afin d'assurer l'utilisation d'une résistance ultime directement dans le dimensionnement des pieux. L'objectif est de remonter à un paramètre de rupture. Comme on a vu, la rupture est associé à la valeur $Q_{b,lim}$. 

On fait appelle la valeur $Q_{b,ult}$, définie avant $Q_{b,lim}$, dans domaines des pieux car il est parfois difficile d'atteindre $Q_{b,lim}$ e/ou afin limiter d'enfoncement à un niveau raisonnable pour les structures. 

Contrairement aux essais des pieux, l'essai DCLT permet obtenir la courbe jusqu'à la rupture en adaptant plus facilement l'énergie de battage aux sols auscultés. Par conséquent, l'essai permet d'atteindre la complète mobilisation de la résistance de pointe pour la plus part de sols courants (enfoncements supérieures à 5 mm).

Les critères pertinents dans le cas de ce travail sont ceux basés sur la valeur $Q_{b,lim}$. Pour l'essai CLT, $Q_{b,lim}$ peut être déduit à partir de l'exploitation de la courbe monotone à l'aide un modèle hyperbolique (critère de \cite{Chin1970}). En revanche, dans le cas le d'essai DCLT cela n'est pas praticable, la composante dynamique en phase de chargement étant non-négligeable. 


\chapter{Étude expérimentale}

Dans ce chapitre, on s'intéresse à une partie expérimentale réalisé en deuxième année. Les essais réalisés ont différents objectifs. Les essais de laboratoire classiques (physiques et triaxiaux) servent d'une part à alimenter les lois de comportement adoptées dans le modèle numérique et d'une autre part à valider les paramètres rhéologiques déterminés par les essais DCLT.

Dans un deuxième temps, on s'intéresse à la série d'essais dynamiques (DCLT) et statiques (CLT) réalisé en chambre d'étalonnage. Ces essais ont pour objectif d'ajuster le modèle de comportement du modèle numérique. Additionnellement, ils permettent comparer quantitativement les courbes de chargement issues deux techniques.

En laboratoire, les matériaux utilisés sont deux sables de référence : le sable d'Hostun et le sable de Fontainebleau. Ces sables propres siliceux ont déjà fait l’objet de travaux importants tant en France qu’à l’étranger.

\section{Matériaux utilisés}

Dans ce travail, on utilise le sable d'Hostun et de Fontainebleau. Ceux-ci sont des deux sables propres de granulométries proches, la principale différence entre ces matériaux étant l'angularité des grains. On a soumis ces différents sols granulaires à des essais de caractérisation comprenant l'analyse granulométrie, la détermination des indices de vides maximale $e_{max}$ et minimale $e_{min}$ et la détermination de la densité relative des grains ($G_s$).

Le sable d'Hostun (SH ou, en anglais, HS) provient de carrières situées à Hostun (département de la Drôme). Composé environ à 98,8 \% de silice, ce sable fin est de couleur gris-blanc à beige-rosé avec des grains sub-anguleux à anguleux. Le sable d’Hostun est issu de la désagrégation de la roche, il existe sous différentes granulométries. Ce sable est disponible en plusieurs fractions granulométrique (HN38, HN34, HN31, HN0,4/0,8, ...). On utilise la fraction HN31 (autrefois nommée Hostun RF). Celle-ci étant référence en France pour les essais de laboratoire \citep{Lancelot1996}.

Le sable de Fontainebleau (SF ou, en anglais, FS) est un dépôt naturel issue de la région parisienne. Ce sable à granulométrie fine (<400$\mu$m) est essentiellement composé de silice (97 à 99 \%) ce qui lui confère la couleur blanche caractéristique. On utilise le matériau normalisé NE34. Les grains ont une morphologie arrondie \citep{Aris2012}.

La Figure \ref{fig:exp_caracterisques_physiques} présente les propriétés de deux sables. On présente les résultats obtenus dans ce travail et, à titre de comparaison, les valeurs rapportées dans la littérature.

  \begin{figure}[H]
   \begin{center}
        \includegraphics[width=\linewidth]{media/exp_caracterisques_physiques.PNG}
        \caption{Caractéristiques des sables : valeurs rapportées dans la littérature et valeurs obtenues dans ce travail}
          \label{fig:exp_caracterisques_physiques}
      \end{center}
 \end{figure}

Dans la Figure \ref{fig:exp_caracterisques_physiques}, on constate que les valeurs de $G_s$ varient très peut pour chaque sable. C'est pourquoi, on retient la valeur de 2,65 pour les deux matériaux. On observe des variations entre les paramètres granulométriques rapportées et les résultats obtenus. Globalement, les résultats obtenus suggèrent des sables plus grossiers légèrement que ceux rapportés. La Figure \ref{fig:granulometrie} présente les courbes granulométriques des deux sables.

Quant aux résultats d'indices de vides ($e_{min}$ et $e_{max}$), on constate des variations. Concernant le protocole utilisé pour la détermination des indices des vides, on a suivi la norme française en vigueur (NF P 94-059). La valeur de $e_{min}$ pour le sable de Fontainebleau est légèrement supérieure à la moyenne des valeurs rapportées tandis que $e_{max}$ reste au-dessous. Pour ce sable, la variation entre les résultats et valeurs rapportées reste inférieure à 15\%. Pour le sable d'Hostun, $e_{min}$ est 22\% supérieure à la moyenne des valeurs rapportées tandis que $e_{max}$ varie très peu (-2\%) de la moyenne. 

  \begin{figure}[H]
   \begin{center}
        \includegraphics[width=\linewidth]{media/exp_granulometrie.PNG}
        \caption{Courbes granulométriques pour le sable d'Hostun et le sable de Fontainebleau}
          \label{fig:granulometrie}
      \end{center}
 \end{figure}

\section{Essais triaxiaux}

Les caractéristiques de rupture d’un sol ainsi que son comportement à grandes déformations peuvent être déduites à partir des essais triaxiaux classiques. On souhaite déterminer des paramètres rhéologiques drainés pour les sables étudiés. Ceux-ci vont permettre (a) d'alimenter les lois de comportement adopté dans les modèles numériques et (b) de valider les paramètres déterminés expérimentalement à partir de l'essai DCLT.

Les facteurs clés influençant le comportement des sables propres sont l'état de contrainte et l'état de densité. Ainsi, les sables étudiés sont testés à différents états de densité et pour différentes pressions de confinement. La Figure \ref{fig:exp_tx_eprouvette} présente l'état de densité initiaux et les confinements appliqués à l'appareil triaxial.

  \begin{figure}[H]
   \begin{center}
        \includegraphics[width=\linewidth]{media/exp_tx_eprouvette.PNG}
        \caption{Conditions des essais triaxiaux réalisés}
          \label{fig:exp_tx_eprouvette}
      \end{center}
 \end{figure}

\subsection{Conception de l'éprouvette et protocole d'essai}

Les éprouvettes reconstituée ont été préparées par compactage dynamique. Les sables ont été compactés par couche (de 3 à 6 couches) à l'intérieur du moule cylindrique (diamètre de 70 mm et hauteur de 140 mm) de façon à atteindre la densité souhaitée. Différentes densités ont été réalisées allant de état lâche (DR$\approx$20\%) jusqu'à l'état très dense (DR$\approx$90\%). Pour chaque densité, on réalise trois éprouvettes à différentes pressions de confinement. L'essai est réalisé en conformité avec les normes françaises \citep{AFNOR1994a,AFNOR1994b}. Le protocole est celui d'un essai consolidé drainé (CD).

La vitesse de chargement est définie à partir de résultats de la phase de consolidation
afin de garantir la condition drainée pendant toute la phase de cisaillement, à savoir de 0,10 mm/min (ou 0,07 \%/min) pour tous les essais.

\subsection{Résultats}

On présente les courbes issues de la phase de cisaillement : (a) déviateur q en fonction de la déformation axiale $\varepsilon$ et (b) la déformation axiale $\varepsilon$ en fonction de la déformation volumique $\varepsilon_v$.  

   \begin{figure}[H]
   \begin{center}
        \includegraphics[scale = 1.]{media/exp_tx_SH.PNG}
        \caption{Courbes triaxiales pour le sable d'Hostun : déviateur q en fonction de déformation axiale $\varepsilon$ et variation volumique $\varepsilon_v$ en fonction de déformation axiale $\varepsilon$}
          \label{fig:exp_tx_SH}
      \end{center}
 \end{figure}

  \begin{figure}[H]
   \begin{center}
        \includegraphics[scale = 1.]{media/exp_tx_SF.PNG}
        \caption{Courbes triaxiales pour le sable de Fontainebleau : déviateur q en fonction de déformation axiale $\varepsilon$ et variation volumique $\varepsilon_v$ en fonction de déformation axiale $\varepsilon$}
          \label{fig:exp_tx_SF}
      \end{center}
 \end{figure}
 
Les Figures \ref{fig:exp_tx_SF} et \ref{fig:exp_tx_SH} mettent en évidence le changement de comportement des sables avec l'augmentation de la densité des échantillons. Pour les états les plus denses des deux sables, on observe la formation du pic. Les courbes de la déformation volumique en fonction de la déformation axiale renfoncent cette constatation : on observe un comportement contractant pour les éprouvettes lâches et une faible phase de contractance initiale puis une phase de dilatance marquée pour les éprouvettes denses et très denses. Comme attendu, la dilatance est d’autant plus importante que les grains sont initialement serrés. La contractance des éprouvettes lâches augmente avec le confinement.

La Figure \ref{fig:exp_tx_rupture} présente deux échantillons cisaillés de sable de Fontainebleau. On peut constater différents modes de rupture selon l'état de densité. Pour l'éprouvette plus dense (Figure \ref{fig:exp_tx_rupture}b), on observe la formation des bandes de cisaillement tandis que pour l'éprouvette lâche (Figure \ref{fig:exp_tx_rupture}a), on constate une rupture plus plastique avec une déformation en \og tonneau\fg{}.

  \begin{figure}[H]
   \begin{center}
        \includegraphics[scale = .7]{media/exp_tx_rupture.PNG}
        \caption{Échantillons du sable de Fontainebleau en fin d’essai : (a) échantillon lâche et (b) échantillon dense}
          \label{fig:exp_tx_rupture}
      \end{center}
 \end{figure}


\subsection{Paramètres de déformabilité}

Du fait du comportement non-linéaire des sols, à partir de la courbe contrainte-déformation issue de l'essai triaxial, on peut définir différents modules. On s'intéresse principalement aux modules sécants, c'est-à-dire, les modules calculés sur le rapport contrainte-déformation pour une déformation donnée de la courbe relié à l’origine, 0. Ces modules sont adaptés principalement pour prédire le tassement d’un sol de fondation dû à l’application d’une charge (apportée par une semelle, par exemple) \citep{Briaud2001}. 

Parmi eux, les modules les plus courants dans les dimensionnements sont le module tangent initial ($E_{ini}$) (théoriquement équivalent à un module sécant pour la déformation nulle) et le module sécant à 50\% de la contrainte de rupture ($E_{50}$).

La détermination du module initial à partir de résultats triaxiaux s'avère parfois compliqué car pour des faibles déformations, on constate des fluctuations importantes pour ce paramètre. Ceci est aussi rapporté par \cite{Lancelot1996} sur ses résultats triaxiaux pour de sable d'Hostun. Par conséquent, on choisit de retenir le module sécant à déformation axiale 0,1\% comme le module initial \citep{Lancelot1996,Dano2001}. En effet, la résolution de l'essai triaxial classique, en général, ne permet pas de déterminer des modules à de déformations inférieures à $10^{-3}$ \citep{Obrzud2018, Benz2007}. Dans la pratique, on admet que les résultats essais triaxiaux permettent de décrire l’évolution du module pour la plage des grandes déformations $10^{-2}$ \citep{Pham2008}.

%_________table synthèse + courbes dégradations triaxial

\subsection{Paramètres de rupture}

Lorsqu'on représente les résultats dans le plan Mohr-Coulomb (la contrainte de cisaillement $\tau$ en fonction de la contrainte normale $\sigma'$). Il est possible de définir, pour chaque état et dans un certain domaine de contraintes, des enveloppes linéaires tangentes aux cercles de Mohr. Ces enveloppes obéissent au modèle de Mohr-Coulomb (Équation \ref{eq:tau}).

\begin{equation} \label{eq:tau}
    \tau = \sigma' \tan \phi' + c
\end{equation}

Avec c’ la cohésion, $\phi'$ l’angle de frottement interne. On peut ainsi déterminer les paramètres de rupture. On sait que la cohésion est quasiment nulle dans le cas d’un sable propre. Ces deux matériaux présentent une cohésion apparente respective variant de 0 kPa, pour les confinements les plus faibles, à 11 kPa, pour les confinements plus importants.

Il est courant d'admettre que l'angle de frottement de pic des sols granulaires dépend de la densité et l'état de contrainte \citep{Rouse2018,Lancelot1996}. Toutefois, cette variation étant plus prononcée plus de contraintes faibles (< 50 kPa). Comme les essais triaxiaux présentés ont été réalisés à des pressions de confinement égales ou supérieures à 50 kPa, on suppose que variation de $\phi'$, pour la gamme de confinement analyser, est négligeable. Ainsi, on présente les résultats en fonction de l'indice de vides (e). 

A titre de comparaison, on présente les résultats aux valeurs obtenues à l'appareil travail (CD) dans autres travaux pour ces sables. Pour sable de Fontainebleau \citep{Combarieu1997, Dano2001, Dupla2007, Arbaoui2003, Latini2016} et pour le sable d'Hostun \citep{Lancelot1996, Benahmed2004}, Mokrani, 1991).

Or, on constate que nos résultats, comme ceux rapportés, sont plus élevés pour le sable d'Hostun que pour le sable de Fontainebleau. Cela peut être expliqué en partie par l'angularité des grains, le sable d'Hostun présentant des grains anguleux à sub-anguleux, tandis ceux de sable de Fontainebleau sont plutôt arrondis \citep{Aris2012, Benahmed2004}.

  \begin{figure}[H]
   \begin{center}
        \includegraphics[width=\linewidth]{media/exp_frictionangle3.PNG}
        \caption{Angle de frottement ($\phi'$) en fonction de l'indice de vides (e) : résultats de la littérature et résultats issus de ce travail} 
          \label{fig:exp_phi_SH_SF_tx}
      \end{center}
 \end{figure}
 

\section{Essais DCLT-CLT en chambre d'étalonnage}

Une série d'essais a été réalisé en laboratoire en utilisant les deux sables de référence (SF et SH) à différents états de densité. L'objectif de ces essais était, d'une part d'ajuster le modèle de comportement appliqué dans le modèle numérique et de comparer les courbes de chargement en pointe issues de deux types de sollicitation : statique (CLT) et dynamique (DCLT).

Les matériaux sont préparés dans une chambre d'étalonnage cylindrique de 372 mm de diamètre et 805 mm de hauteur. Avant de déposer les sables à l'intérieur de la chambre, une nappe plastique est collée à ces parois latérales afin de minimiser le frottement latéral. Un système de guidage est fixé à la chambre d'étalonnage afin de garantir la verticalité du pénétromètre et de minimiser le possible flambement des barres au cours du chargement.

Pour chacun de ces sables, deux échantillons ont été préparés par damage humide à différentes densités. Ces échantillons ont été préparés en huit couches de sol d'environ 10 cm de façon à garantir l'homogénéité de l'éprouvette. Le compactage dynamique a été réalisé manuellement à l'aide une dame. Des échantillons sont prélevés afin de déterminer la teneur en eau et d'obtenir la densité sèche en place. La Figure \ref{fig:exp_DCLT_CLT_eprouvettes} présente les densités initiales des éprouvettes testées.

\begin{figure}[H]
\begin{center}
    \includegraphics[width=\linewidth]{media/exp_DCLT_CLT_eprouvettes.png}
    \caption{Caractéristiques des éprouvettes (Paramètres adoptés : pour SF : $e_{min}$=0.54 et $e_{max}$=0.96 ; pour SH : $e_{min}$=0.62 et $e_{max}$=0.96 ; $G_s$ = 2.65 pour les deux sables)}
      \label{fig:exp_DCLT_CLT_eprouvettes}
  \end{center}
\end{figure}

Dans l'axe de l'éprouvette, on a effectué des chargements dynamiques et statiques. Initialement, on a réalisé l'essai dynamique jusqu'à une profondeur d'environ 30 cm. On installe par la suite le système de chargement statique (capteurs LDVT, barre filetée et pièce adaptative associant le pénétromètre au portique de réaction) (voir Figure \ref{exp_DCLT_CLT}). 

On poursuit avec l'essai CLT. La force et le déplacement sont enregistrés directement pendant que la vitesse de chargement est constante à 0,01 mm/s. Le pilotage s’effectue en déplacement. L’effort est mesuré au moyen du capteur de force de 10 kN à 0,01 kN près, tandis que le déplacement de la pointe est mesuré au moyen de deux capteurs de déplacement LVDT. Les données sont enregistrées, à raison d’une mesure par seconde, par la centrale de mesure à laquelle sont reliés les différents capteurs. 

Le capteur de force est fixé à un plateau sur lequel vient s’appuyer le vérin ; une pièce métallique assure la liaison du capteur avec le système de guidage qui est relié à l'enclume du pénétromètre. Une fois le vérin mis en contact avec le plateau, les capteurs sont mis à zéro (correspondant à l’origine des mesures). Chaque essai de chargement statique est mené jusqu’à un enfoncement de l'ordre de 5 mm. On réalise de cinq à six cycles chargement-déchargement-rechargement. Le déchargement est rapide jusqu’à une pression nulle et le rechargement est mené suivant le même chemin emprunté lors d’un chargement monotone.

Ensuite, on enlève le dispositif d’enfoncement statique et on poursuit l'essai DCLT sur environ 10 cm. Cette procédure est répétée de façon à avoir réalisé des essais CLT à trois profondeurs différentes. Une fois le chargement CLT réalisé à la troisième profondeur, on poursuit avec l'essai DCLT jusqu'au fond du moule.


\begin{figure}[ht] 
\begin{center}
    \includegraphics[width=\linewidth]{media/exp_DCLT_CLT.PNG}
    \caption{Comparaison entre les techniques DCLT et CLT. (a) Schéma de la chambre d'étalonnage présentant les profondeurs de réalisation de chaque essai. Photo présentant les différents éléments de chaque technique (b) DCLT and (c) CLT}
    \label{exp_DCLT_CLT}
\end{center}
\end{figure}

La Figure \ref{fig:exp_penetro} présente les pénétrogrammes des échantillons et les trois profondeurs de chargement CLT.

\begin{figure}[H]
\begin{center}
    \includegraphics[scale = .5]{media/exp_penetro.png}
    \caption{Pénétrogrammes des échantillons montrant les trois niveaux de réalisation de chargement CLT (niveaux 1, 2 e 3)}
      \label{fig:exp_penetro}
  \end{center}
\end{figure}


On présente par la suite l'ensemble des courbes CLT obtenues. Les courbes DCLT présentés sont celles de 5 impacts réalisés immédiatement avant et après les chargements CLT. Les différentes couleurs indiquent les profondeurs (marqué \emph{level} sur la Figure) : bleus, rouges et vertes correspondent respectivement aux profondeurs 1 a($\approx$ 30 cm), 2 ($\approx$ 43 cm) et 3 ($\approx$ 56 cm) de chaque éprouvette. 

\newpage
\begin{figure}[H]
\begin{center}
    %\includegraphics[scale = .7]{media/exp_DCLT-CLT_Hostun.png}
    \includegraphics[width=\linewidth]{media/a.png}
    \caption{Courbes CLT et DCLT pour le sable de Fontainebleau à DR=46\%}
      \label{fig:a}
  \end{center}
\end{figure}

\newpage
\begin{figure}[H]
\begin{center}
    %\includegraphics[scale = .7]{media/exp_DCLT-CLT_Hostun.png}
    \includegraphics[width=\linewidth]{media/b.png}
    \caption{Courbes CLT et DCLT pour le sable de Fontainebleau à DR=71\%}
      \label{fig:b}
  \end{center}
\end{figure}

\newpage
\begin{figure}[H]
\begin{center}
    %\includegraphics[scale = .7]{media/exp_DCLT-CLT_Fontainebleau.png}
    \includegraphics[width=\linewidth]{media/c.png}
    \caption{Courbes CLT et DCLT pour le sable d'Hostun à DR=51\%}
      \label{fig:c}
  \end{center}
\end{figure}

\newpage
\begin{figure}[H]
\begin{center}
    %\includegraphics[scale = .7]{media/exp_DCLT-CLT_Fontainebleau.png}
    \includegraphics[width=\linewidth]{media/d.png}
    \caption{Courbes CLT et DCLT pour le sable d'Hostun à DR=67\%}
      \label{fig:d}
  \end{center}
\end{figure}


Dans les Figures \ref{fig:a},\ref{fig:b},\ref{fig:c} et \ref{fig:d}, on constate que les deux techniques sont bien sensibles à la variation de la densité des éprouvettes. On observe que les séries de courbes DCLT sont assez répétables. Pour les courbes CLT, on observe des différences entre le premier chargement et ceux qui suivent. Le premier chargement CLT présentant des contraintes légèrement plus faibles que les autres d'une même série. Cet effet est conforme aux constatation des \cite{Arbaoui2003} et \cite{Zhou1997}. Il peut être attribué en partie à la densification du sol autour de la pointe. Globalement, les courbes DCLT sont comparables au premier chargement CLT. 

En plus, on constate que les courbes CLT vertes pour les éprouvettes plus denses (HS-M2-CLT3 et FS-M2-CLT3) présentent une augmentation des contraintes. On attribue cela à la proximité du fond de l'éprouvette. Cette constatation est en accord avec autres résultats rapportées pour des sables en chambre d'étalonnage \citep{Salgado1998, Balachowski2014}. Ceux-ci ont montré que l'augmentation de la résistance statique attribuée aux effets de bords sont plus expressives pour de sables plus dense (ayant un comportement dilatant).

%Par ailleurs, cet effet est encore plus expressif pour le sable de Fontainebleau (Figure \ref{fig:d}). Il est important que rappeler la principal différence entre ces deux matériaux : la forme de grains. Comme déjà évoqué les grains du sable de Fontainebleau sont plus arrondie tandis que ceux d'Hostun sont anguleux. Certains travaux ont montré que les grains ayant une forme arrondie présentent un comportement plus dilatant que ceux de forme angulaire \citep{}.

Pour ce cas de figure analysé, les courbes de chargement issues des deux techniques sont qualitativement comparables. Ces résultats permettront de plus tard ajuster la loi de comportement appliquée au modèle numérique.

Comme évoque dans le premier chapitre de document, la résistance limite en pointe peut être déterminé à partir des courbes de chargement statiques ($Q_b$). Selon, la méthode \cite{Chin1970}, cette valeur correspond à l'inverse du coefficient d'un calage hyperbolique simple. Souvent pour les courbes de chargement des pieux, il est difficile d'appliquer cette méthode car les équipements courants ne permettent pas d'atteindre de déplacements conséquents. Toutefois, pour les essais de chargements en pointe réalisés, on a pu atteindre de déplacement de l'ordre de 5 mm.

Pour les cas de l'essai DCLT, l'application de la méthode de \cite{Chin1970} n'est pas direct vu qu'un calage hyperbolique de la phase de chargement de courbe intégrerait la composante dynamique. Ce calage serait ainsi fonction de la vitesse de chargement.

La Figure \ref{fig:exp_DCLT_CLT_qlim} présente la comparaison entre les résistances limites statique et dynamiques pour la chargement en profondeur intermédiaire des échantillons.


\begin{figure}[H]
\begin{center}
    \includegraphics[width=\linewidth]{media/exp_DCLT_CLT_qlim.PNG}
    \caption{Comparaison entre la résistance limites d'essai CLT et des courbes DCLT pour le niveau 3 des échantillons}
      \label{fig:exp_DCLT_CLT_qlim}
  \end{center}
\end{figure}

Comme évoqué par d'autres travaux, l'obtention des résistance de pointe en chambre de calibration n'est pas une tâche facile. Les effets liés à la taille de la chambre et les conditions aux limites ont été sujets de nombreux travaux ((Bellotti et al. 1982; Sisson 1990; Mayne and Kulhawy 1991). Les incertitudes liés à la densité relative sont de l'ordre de $\pm$5\% \citep{Salgado1998}. Les échantillons ne sont pas parfaitement uniforme en profondeur. En effet, la non-homogeneité des éprouvettes toute seule entraine de variations de l'ordre de $\pm$5\% dans les résistances statiques obtenues \citep{Salgado1998}.


\section{Bilan}

Dans ce chapitre on a présenté les essais réalisés en laboratoire pour les sables de références d'Hostun et de Fontainebleau. Les essais en laboratoire classique ont pour but de caractériser les matériaux et de fournir les paramètres rhéologiques afin de valider la méthode proposée et alimenter le modèle numérique.

En ce concerne les paramètres physiques, on a présenté une comparaison avec les résultats rapportées dans la littérature. Cela a permis d'évaluer les résultats obtenus qui ont été dans l'ordre de grandeur attendu. 

Grâce aux résultats triaxiaux réalisé pour différentes états de densité et différentes confinements, on a pu déterminer les paramètres rhéologiques qu'on souhaite déterminer : modules de déformations et l'angle de frottement.

En ce qui concerne l'angle de frottement, on a présente l'évolution des valeurs obtenus en fonction l'indice de vides pour ces sables. Cela pour des échantillons testés à des confinements égale et supérieurs à 50 kPa.

Par ailleurs, on a réalisé une série d'essai de chargement CLT et DCLT en chambre d'étalonnage. Au-dèla de comparer les deux techniques, ces tests seront utile afin de proposer et valider le modèle d'interaction pour l'essai DCLT. Les courbes CLT et DCLT obtenus seront également importantes afin d'ajuster la loi de comportement du modèle numérique.




%____________________________________________________________________________________
%\newpage    
 %   \vspace*{5cm}
 %       \begin{center}
  %          \section*{\textbf{NUMÉRIQUE}}
  %      \end{center}
  %  \vspace*{\fill}
%\newpage


%----------------------------------------------------
  \chapter{Étude numérique}
  \label{sec:num}
  
A l’heure actuelle, les méthodes numériques dites continues (\emph{e.g.} FEM, FDM) constituent l’approche la plus réaliste afin de simuler l’interaction sol-pointe \citep{Loukidis2008, Le2013}. Celles-ci permettent de reproduire toute la complexité de la pénétration (comportement non-linéaire des sols, prise en compte des différents types d'amortissement, ...). C’est pourquoi, on propose 
d'appliquer une méthode continue, et plus précisément en différences finies, aux battage dynamique pénétrométrique. On souhaite développer un modèle numérique qualitativement et quantitativement représentatif de l'essai DCLT.

L'objectif de ce modèle est d’étudier l’influence des paramètres rhéologiques du sol ausculté sur le principal résultat de l'essai : la courbe charge-enfoncement dynamique (ou courbe DCLT). Les paramètres rhéologiques étudiés sont les plus courants pour le dimensionnement en géotechnique : en déformation, le module d'élasticité (E) et, en cisaillement, l'angle de frottement interne ($\phi$) et la cohésion (c). En plus, ce modèle numérique permettra d’étudier: 

-	l’influence du type de sollicitation : statique et dynamique ;

-	les déformations entraînées par l’essai dans le sol.

Dans ce chapitre, on rappelle d'abord certains travaux réalisés sur le même sujet et certains aspects numériques importants pour comprendre le logiciel. Par la suite, on présente une partie du travail numérique réalisé au cours de la deuxième année.
 
  \section{État de l'art}
  \label{sec:etat de l art}
  
On présente par la suite l'état de l'art sur la modélisation numérique. Dans notre étude, initialement, on va se focaliser sur les modèles numériques appliquant des méthodes continus à la pénétration dynamique. Additionnellement, on abordera les modèles numériques développés autour des essais DCLT et CLT.

\subsection{Modélisation numérique de la pénétration dynamique}
\label{subsec: modelisation numerique des essais DCLT et CLT}

La pénétration dynamique reste un des problèmes les plus sophistiqué des applications numériques en géotechnique. Il s’agit d’un problème fortement non-linéaire et qui entraîne des déformations très importantes \citep{Nazem2012}.

De nombreux travaux numériques s’intéressent à simuler la pénétration dans les milieux granulaires. Parmi eux, la méthode discrète reste est la plus courante \citep{Zhang2019, Quezada2014a}. Les modèles rapportés simulant la pénétration dynamique à l’aide des méthodes continues (FEM ou FDM) sont moins courants \citep{Murthy2020, Chatterjee2015a, Fakharian2014,  Aguiar2009, Pinto2008, Nath1990}. Parmi eux, les travaux simulant le battage sont encore plus rares. Généralement, la sollicitation dynamique est simulée en appliquant une force sinusoïdale à la tête du pieux.

\cite{Masouleh2008} ont simulé le chargement dynamique d’un pieu à l’aide de FLAC$^{2D}$. Ils ont présenté les résultats numériques des analyses de propagations d’ondes et la réponse dynamique sol-pieux au cours du battage. Les signaux obtenus numériquement ont été assez conformes aux résultats des chargements dynamiques des pieux. La bonne correspondance entre les résultats numériques et expérimentaux montre la faisabilité d’application de l’approche continue pour modéliser des problèmes dynamiques de ce type. Dans leur modèle, \cite{Masouleh2008} font varier les caractéristiques du sol (paramètres de résistance et de déformation) et de l’interface sol-pieu afin de faire du signal matching (similaire au CAPWAP) d’après les résultats expérimentaux.

\cite{Fakharian2014} ont simulé le battage de pieux à partir des deux approches continues (FEM et FDM). Les résultats de force, vitesse et déplacement observés dans le modèle sont conformes aux essais de chargement dynamique réalisés. Le but étant d’introduire l’utilisation des méthodes numériques continues afin de soulever certaines limitations du système masse-ressort-amortisseur (comme celle de \cite{Smith1960}. A la différence de la méthode masse-ressort-amortisseur, l’approche continue est alimentée par des paramètres mécaniques classiques (\emph{i.e.} E, c, $\phi$, $\nu$). Les paramètres du modèle unidimensionnelle de Smith, comme l’amortissement de Smith (J$_s$) et l’enfoncement élastique (quake) sont d’usage moins courant en géotechnique.

\cite{Pinto2008} ont également simulé le battage des pieux par le biais de la méthode FEM. Leur but était précisément de comparer celle-ci à la méthode Smith. Ses résultats ont montré la bonne conformité entre ces deux méthodes ainsi qu’aux résultats expérimentaux. Cette comparaison est pertinente car même si le modèle unidimensionnel de Smith est simple, il est largement accepté comme une bonne représentation du problème du battage.

En ce qui concerne l'application de la modélisation numérique aux essais DCLT et CLT, un nombre de travaux ont été réalisé. On peut citer des travaux appliquant l’approche discrète \citep{Tran2019, Escobar2015, Tran2015, Benz-Navarrete2009,Zhou1997} ou l’approche continue \citep{Ali2010, Arbaoui2003, Zhou1997}. Les objectifs de ces travaux étaient divers. \cite{Zhou1997} et \cite{Arbaoui2003} ont simulé l’essai CLT réalisé avec le pénétromètre Panda. Ils ont utilisé différents codes appliquant des méthodes continues (FLAC$^{2D}$, CESAR). Plus tard, \cite{Ali2010} simula l'essai CLT, cette fois-ci sur la base du pénétromètre statique CPT. Leur but était, entre autres, de réaliser l’analyse inverse de la courbe de déformabilité de cet essai et ainsi d’estimer des paramètres rhéologiques (E, c et $\phi$) du sol ausculté.

\cite{Zhou1997} a aussi employé la méthode numérique discrète (PFC$^{2D}$) afin de valider le calcul de résistance de pointe Panda à partir de la formule des Hollandais ainsi que les hypothèses (choc élastique entre marteau et piston). L’approche discrète a été employée plus tard par \cite{Benz-Navarrete2009} et par \cite{Quezada2012} afin d’étudier le mécanisme d’enfoncement. Les modèles discrets proposé par \cite{Tran2019} et par \cite{Tran2015} ont permis, entre autres, de valider la méthode d'obtention de la courbe charge-enfoncement en pointe à partir de l'exploitation des signaux mesurés au niveau de la tête.

Ces travaux ont permis de mettre en évidence la pertinence du couplage numérique- expérimental. Ils ont été important afin de validation de la technique et mieux comprendre l'essai, toutefois aucun parmi eux ne se sont intéressés à la modélisation de l’essai DCLT dans le but d’interpréter la courbe charge-enfoncement.

\section{Mise au point du modèle numérique}
\label{sec:Mise au point du modèle numérique}

On emploie le logiciel $FLAC^{3D}$\footnote{\samepage Désormais nommé FLAC} (\emph{Fast Lagrangian Analysis of Continua in 3 Dimensions}). Il s’agit d’un logiciel aux différences finies explicite qui permet d’étudier le comportement mécanique d’un milieu continu en trois dimensions. Il applique des principes simples (définition des déformations et loi du mouvement) et des lois constitutives qui décrivent un matériau pour résoudre des équations différentielles à partir d’un schéma de discrétisation temporelle explicite. Les calculs sont réalisés en boucle pour chaque élément dans chaque pas de temps de façon indépendante (Figure \ref{fig:num_flac}).

  \begin{figure}[H]
   \begin{center}
        \includegraphics[scale = .7]{media/num_flac.PNG}
        \caption{Boucle de calcul explicite dans FLAC \citep{Itasca2020}}
          \label{fig:num_flac}
      \end{center}
 \end{figure}

A la différence de la méthode aux éléments finis, la méthode explicite intégrée dans FLAC ne requiert pas la création d’une matrice de rigidité. Par conséquent, ce logiciel peut suivre un chemin de contrainte-déformation non-linéaire sans une augmentation significative du temps de calcul.

La méthode explicite utilisée dans FLAC est associée à une analyse lagrangienne. Les coordonnées des  du maillage dans l’espace correspondent à des particules du milieu dont les positions sont référencées par rapport à la configuration initiale. Ainsi, la fonctionnalité dite « en grandes déformations » permet de modéliser des problèmes à grandes déformations car les coordonnées des nœuds sont mises à jour à chaque pas de temps. Cette méthode s’oppose à la méthode eulérienne où le matériau de déplace et se déforme tandis que le maillage est fixe. Selon \cite{Ghee2011}, un des avantages de la méthode intégrée dans FLAC constitue le qu’il fonctionne en petites ainsi qu’en grandes déformations. Dans la fonctionnalité dite en « grandes déformations », le maillage est mis à jour lors de déformations ou de mouvements. Ce mode est particulièrement intéressant pour traiter les problèmes à grandes déformations comme c’est le cas d’un essai de pénétration.

On décrit par la suite certains aspects de la mise en point du modèle numérique, à savoir : géométrie, conditions aux limites, calibration du battage, modèle de comportement du sol et le choix des paramètres rhéologiques du modèle.

\subsection{Géométrie et conditions aux limites}
\label{subsec:Géométrie et conditions aux limites}

On souhaite simuler les essais DCLT et CLT en chambre d'étalonnage. Les essais sont simulés au sein d'un massif homogène représentant un sol frottant sec. Ce massif possède la forme et les dimensions de la chambre d'étalonnage. Afin de simuler les conditions aux limites de la chambre d'étalonnage utilisée, les déplacements sont empêchés latéralement.

Les différents éléments composant le pénétromètre sont modélisés comme des cylindres. Les dimensions sont celles de la géométrie réelle : tige de 14 mm de diamètre et un mètre de longueur ; une pointe cylindro-conique débordante (sommet de 90\degree) de section transversale de 4 $cm^2$ (22,5 mm de diamètre). Les éléments composants le pénétromètre sont modélisées des corps élastiques dont les paramètres sont ceux de l'acier ($\nu$ = 0,3, E=206 GPa). La géométrie de la chambre d'étalonnage ainsi que celle du pénétromètre sont présentées dans la Figure \ref{fig:exp_penetro}. 

Un état initial de contraintes d’origine géostatique est pris en considération. Il correspond à l’effet du poids de couche de sol, avec le poids volumique sec $\gamma_d$. Dans le modèle, la contrainte latérale est engendrée par le coefficient des terres au repos $K_0$, $K_0$ étant le rapport entre les contraintes radiales et verticales ($K_0 =\sigma_{rr}/\sigma_{zz}$). Pour les sables, ceci est estimée à l’aide de la relation proposée par \cite{Jaky1944} : $K_0=1-\sin \phi'$ , avec $\phi'$ l’angle de frottement drainé du massif. Le choix de l'angle de frottement ainsi que d'autres paramètres physiques et rhéologiques sera aborder plus tard. Une cavité est crée au sein du massif afin de donner lieu au pénétromètre. Les parois de cette cavité sont stabilisés, c'est-à-dire les déplacements sont empêchés latéralement

Étant donné la caractère axisymétrique du problème et dans le but de réduire le temps de calcul, on a choisi de modéliser uniquement un quart de la géométrie. Ainsi, les déplacements sont empêchés dans les plans normaux latéralement (x ou y, selon le plan concerné). Pour ce faire, on a vérifié, dans un premier temps, si cela entraînait de différence dans le résultat. On a simulé l'essai DCLT pour les deux cas : (a) géométrie intégrale et (b) 1/4 de la géométrie. Comme on constate dans la Figure \ref{fig:num_axy.PNG}, le résultat de l'essai, synthétisé par la courbe DCLT, est essentiellement le même. C'est pourquoi on a choisi d'utiliser la géométrie (b) pour les prochaines simulations. 

   \begin{figure}[H]
        \begin{center}
            \includegraphics[scale = .7]{media/num_axy.PNG}
            \caption{Comparaison entre les courbes DCLT pour deux cas différents : (a) géométrie intégrale et (b) 1/4 de la géométrie}
            \label{fig:num_axy.PNG}
      \end{center}
    \end{figure}


Dans le contexte du battage des pieux et des essais au pénétromètre dynamique, l’onde de contrainte engendrées par le battage se propage autour de la tige/pieu et se dissipe graduellement (dissipation d’énergie causée par la radiation des ondes au sein du matériau). Dans le but de reproduire cela et d’empêcher la possible réflexion des ondes sur les parois, on introduit des parois dites « absorbantes ». 

Celles-ci sont constituées d’amortisseurs dans les deux directions perpendiculaires aux noeuds de éléments de sol placés aux limites (bords latérales) \citep{Lys1969}. Ces amortisseurs apportent des forces normales et transversales ($t_n$ et $t_s$). Ces forces sont fonction des vitesses normales et transversales ($v_n$ et $v_s$), de la masse volumique $\rho$ et des célérités d’ondes ($c_p$ et $c_s$) des éléments placés aux limites. L'application de ces forces aux nœuds placés sur les faces limitant le modèle permet d'empêcher la réflexion des ondes aux bords du modèle. 

\begin{equation} \label{bords_amortis}
    \begin{aligned}
        t_s= - \rho  v_s c_s \\
        t_n= - \rho  v_n c_n \\
    \end{aligned}
\end{equation}

La qualité du maillage est essentielle pour la précision et fiabilité des résultats. Un grand défi pour la modélisation continue de la pénétration est le fait d'entraîner de grandes déformations dans le sol. Les limitations quant aux déformations du maillage varient selon la méthode intégrée dans le code. En général, le code associant la méthode Lagrangienne et la méthode Eulérienne permettent de mieux gérer les déformations (\emph{e.g.} code Abaqus). La méthode Lagrangienne intégrée par le code FLAC permet de mettre à jour les coordonnées des nœuds au cours du calcul. Cette méthode n’est pourtant pas adaptée lorsque les déformations deviennent trop importantes \citep{Moug2019, Nagula2018}.

C’est pourquoi la pénétration entraînée et le maillage adopté doivent être correctement choisis afin de limiter les distorsions et ainsi assurer la fiabilité des résultats. Lorsqu’on analyse les modèles utilisant le code FLAC pour la pénétration, on peut avoir une idée des ordres de grandeurs des pénétrations et des éléments du maillage.

La plupart des modèles emploient un maillage composé par des éléments homogènes. Les enfoncements correspondent de 10 à 55\% de la taille des éléments du maillage. Quant au rapport entre le diamètre du pieu et la taille d’éléments du maillage, ceci varie de 5\% à 80\%. Différentes tailles d’élément de maillage ont été testées afin d’analyser la sensibilité des résultats. Le but étant de trouver un maillage suffisamment fin afin de converger vers une solution unique sans toutefois trop augmenter le temps de calcul. Le maillage retenu est composé de 20160 nœuds et 16448 éléments avec des éléments quadrilatères à huit nœuds.



\subsection{Interface}
\label{subsec:interface}

Afin de simuler l’interaction entre le pénétromètre et le sol, il s’avère nécessaire introduire une interface entre les deux matériaux. Celle-ci permettra de reproduire correctement le choc au niveau de la pointe et le glissement entre les parois de la pointe et le sol. Dans FLAC, l’interface est modélisée par un maillage de triangles définis par trois nœuds placés sur la surface de l’objet. Celle-ci est caractérisée par des paramètres de raideur et de rupture (critère de rupture Mohr-Coulomb). Ces paramètres doivent être choisis afin de minimiser son effet dans le modèle.

On a créé deux interfaces au niveau de la pointe : une latérale et une au sommet (Figure \ref{fig:num_interface2}). Le comportement de l’interface est caractérisée par (a) les paramètres de raideur : normale k\textsubscript{n} et tangentielle k\textsubscript{s} données en contrainte par mètre (Pa/m), et (b) de rupture suivant le critère Mohr-Coulomb, avec la cohésion (en Pa) et les angles de frottement et de dilatance (en dégrée). 

Les raideurs apparentes de l’interface d’une zone perpendiculaire sont souvent estimées à partir de la taille de l’élément le plus petit dans la direction normale $\Delta Z_{min}$ et le module (E) du milieu adjacent \citep{Itasca2020}. \emph{A priori}, les raideurs de l’interface sont estimées à partir du module d'élasticité du matériau le plus raide, à l’exception des problèmes où les modules des deux matériaux sont trop différents. Dans ce cas spécifique, on estime les raideurs de l’interface à partir du module du matériau le moins raide (ici, le sol) (Équation \ref{eq:ks}).

\begin{minipage}{0.45\textwidth}
   \begin{figure}[H]
    \includegraphics[scale =.4]{media/num_interface_kn.PNG}
    \label{fig:num_interface_kn}
    \caption{Définition de $\Delta Z_{min}$}
    \end{figure}
\end{minipage}%
%\hfill
\begin{minipage}{0.45\textwidth}
    \begin{equation} \label{eq:ks}
    k_{n} = k_{s} \geq 10 \frac{K+\frac{4}{3} G}{\Delta Z_{min}}
    \end{equation}
\end{minipage}

%\vspace{1.5}
%\newline
%\\[2 cm]
La définition des paramètres de l’interface constitue une tâche importante car ceux-ci influencent la réponse globale et interviennent comme « nouveau matériau ». Afin de vérifier si la relation proposée par le logiciel (Équation \ref{eq:ks}) permet d'empêcher cela, on a réalisé une étude paramétrique.

Cinq valeurs de raideurs ont été testées en fixant toutes les autres variables. La raideur minimale été calculée à partir de l'équation \ref{eq:ks} pour un massif de caractéristiques connues (à savoir : E = 80 MPa, $\rho_d$ = 1700 kg/m$^3$, c = 0 kPa, $\phi$= 38\textdegree{} et $\psi$= 8\textdegree{}). Pour ce massif, la raideur minimale ($k_{min}$) devrait être de 5,3x$10^{9}$ Pa/m (\citep{Itasca2020}, Équation \ref{eq:ks}). Afin de vérifier si cette valeur permet de s'affranchir de l'effet de l'interface, on a testé autres cinq valeurs de raideur.  

   \begin{figure}[H]
        \begin{center}
            \includegraphics [width=\linewidth]{media/num_interface3.PNG}
            \caption{(a) Détail au niveau de la pointe montrant la localisation de l'interface (en jaune), (b) Courbes DCLT pour les différentes valeurs de raideurs testées}
            \label{fig:num_interface2}
      \end{center}
    \end{figure}
 
De la Figure \ref{fig:num_interface2}b, on constate que pour les raideurs inférieures à $k_{min}$ (ici 5,3 x 10$^9$ Pa/m), l'allure de la courbe présente un changement non-négligeable, présentant des contraintes plus petites et des enfoncements plus importants. Pour les valeurs supérieures à $k_{min}$, les courbes sont quantitativement similaires, on observe uniquement l'augmentation de bruits (courbes rose, jaune et rouge). On conclut que l'équation \ref{eq:ks} permet d'estimer à partir des paramètres de déformabilité du sol, les raideurs d'interface minimales de l'interface sol-pointe afin de garantir que celles-ci n'interviennent pas significativement dans les résultats.

\subsection{Sollicitation}
\label{subsec:sollicitation}

On a simulé le chargement pointe pénétrométrique statique et dynamique. Le chargement statique a été simulé en appliquant une vitesse constante (0,01 mm/s) à la face supérieure de l'extrémité de la tige. 

En ce qui concerne l'essai dynamique, comme déjà évoqué, la plupart des travaux numériques rapportés simulent le chargement dynamique en appliquant une contrainte sinusoïdale à l’extrémité supérieure des pieux \citep{Fakharian2014, Masouleh2008,Pinto2008}. Dans le cas de ce travail, comme dans les travaux précédents \citep{Benz-Navarrete2009, Escobar2015, Tran2019} simulant l'essai DCLT, on a choisi de simuler le battage.

Pour ce faire, il s'avère nécessaire d'ajuster la sollicitation pour que celle-ci soit représentative de l'onde de choc observée expérimentalement. L'onde de compression générée suite à l'impact est triangulaire avec une légère asymétrie, celle-ci peut être caractérisée par la durée du choc ($\Delta t$) et une contrainte maximale au pic ($\sigma_{max}$). On décrit par la suite la calibration du battage réalisé afin d'obtenir une onde représentative de celle observé expérimentalement. Dans un premier, on vérifie la capacité d'un modèle simple en différences finies de simuler la propagation d'une onde de compression. Pour ce faire, on simule le cas de l'essai SPT. Dans un deuxième temps, on se focalise sur l'essai DCLT. Le battage numérique est calibré à l'aide des résultats expérimentaux en laboratoire.


\subsubsection {Cas du SPT : simulation du battage et propagation de l'onde}
 
On souhaite simuler l'essai DCLT. L'essai DCLT consiste à enfoncer le pénétromètre par battage. Afin d'être réaliste vis-à-vis de la technique et pouvoir mieux comparer les résultats en terme d'énergie, on a choisi de simuler le battage.

Afin de vérifier dans premier temps la faisabilité, dans un premier temps, on a simulé un essai similaire, mais plus simple : le SPT. En effet, le battage SPT étant plus simple (en terme de géométrie et de matériaux utilisés), le problème a été déjà modélisé analytiquement par d'autres travaux \citep{Schmertmann1979, Gonin1979, Fairhurst1959}. Ces travaux ont obtenus des bonnes correspondances entre les prédictions analytiques à des résultats expérimentaux.

En suivant la même démarche, on souhaite vérifier si on est capable de simuler le battage SPT en différences finies. Pour ce faire, il s'avère nécessaire comprendre des phénomènes dynamiques ayant lieu après l'impact. Ce sujet sera brièvement abordé par la suite.

\subsubsection{Théorie}

L'onde de compression u(x,t) crée par l'impact du marteau se déplace le long des tiges. La déformation entraînée par le passage de cette onde $\varepsilon$ peut être exprimée par l'équation \ref{eq:wave_def}.

\begin{equation} \label{eq:wave_def}
    \varepsilon = \frac{\partial u}{\partial x}
\end{equation}

%En appliquant la relation deuxième loi de Newton (F=ma), avec la masse m comme m=Atrhoρt dx et l'accélération $a=\partial^2/\partial t^2$ pour une portion %de la tige, on obtient à l'équation \ref{eq:stress}.

\begin{equation} \label{eq:wave_stress}
    \frac{ \partial \sigma}{\partial^2}= \rho_t^2 \frac{\partial u}{\partial x}
\end{equation}

Par ailleurs, en supposant la tige élastique,  $\sigma=E_t \varepsilon$, avec $\varepsilon$ donné par l'équation \ref{eq:wave_def}, on obtient l'équation différentielle qui gouverne la propagation de l'onde u(x,t), connue simplement équation de l'onde. 

\begin{equation} \label{eq:wave}
    \frac{ \partial \sigma}{\partial t^2}= c_t^2 \frac{\partial u}{\partial x}
\end{equation}

Il est possible l'équation de l'onde analytiquement par différentes méthodes (transformée de Laplace, séparation des variables, méthode des caractéristiques) ou numériquement. Chacun de ces méthodes est discuté par \cite{Verruijt2009}. Dans le cadre de ce travail, on retient la solution obtenue par la méthode des caractéristiques, proposé initialement par \cite{Saint-Venant1898}. Selon celle-ci, l'équation \ref{eq:wave} correspond à la superposition deux ondes $u_{d}(t-x/c_t)$ (descendante) et $u_{u}(t-x/c_t)$ (remontante) se propageant le long des tiges de manière continue à une vitesse d'ondes ($c_t$) (Équation \ref{eq:wave_deplacement}).

\begin{equation} \label{eq:wave_deplacement}
u(x,t) = u_d(t-x/c_t)-u_u(t+x/c_t)
\end{equation}

L’impact engendre une onde de compression u(x,t) dans le pénétromètre qui se propage à vitesse constante dans le sens longitudinal. La vitesse de propagation d’une onde de compression $c$ dans un milieu solide infini est déterminée par l'équation (\ref{eq:wave_c}). La vitesse de propagation $c$ étant une fonction des propriétés du matériau du milieu traversé (module d'Young E, coefficient de Poisson $\nu$ et densité $\rho$).

\begin{equation} \label{eq:wave_c}
    c = \sqrt{\frac{1}{\rho}\frac{E(1-\nu)}{(1+\nu)(1-2\nu)}}
\end{equation}

Pour le cas des tiges élancées, la vitesse de propagation de l’onde de compression u(x,t) se déplaçant longitudinalement est indépendante du coefficient de Poisson $\nu$ \citep{Timoshenko1951} et est donc donnée par l’équation \ref{eq:wave_ct}.

\begin{equation} \label{eq:wave_ct}
    c_t = \sqrt{\frac{E_t}{\rho_t}}
\end{equation}

Soit $c_t$, $E_t$, et $\rho_t$ la vitesse de propagation de l'onde, le module d'Young et la densité des tiges respectivement. 

Le passage de l'onde u(x,t) à un instant t engendrera des variations de déformation et de vitesse particulaire v(x,t) à un point donnée de la tige x.  La vitesse particulaire v(x,t) est obtenue en dérivation de l'équation \ref{eq:wave_deplacement} en fonction du temps.

\begin{equation} \label{eq:wave_vitesse}
v(x,t) = v_d(t-x/c_t)-v_u(t+x/c_t)
\end{equation}

La déformation entraînée par le passage de l'onde u(x,t), celle-ci obtenue à partir de la dérivation de l'équation \ref{eq:wave_def} en fonction de x. Cette déformation $\varepsilon$ peut être exprimée en terme de la vitesse particulaire des ondes, incidente et remontante, afin d'obtenir l'équation \ref{eq:wave_def2}.

\begin{equation} \label{eq:wave_def2}
    \varepsilon(x,t)=-\frac{1}{c_t}[v_d(t-x/c_t]-v_u(t+x/c_t)]
\end{equation}

En tenant compte de l'élasticité des tiges, c'est-à-dire de $F=\varepsilon A_t E_t$, on obtient l'équation \ref{eq:wave_FZv}

\begin{equation} \label{eq:wave_FZv}
    F(x,t)=-\frac{A_t E_t}{c_t}[v_d(t-x/c)-v_u(t-x/c)]
\end{equation}

Comme constaté sur l'équation \ref{eq:wave_FZv}, dans le problème de la propagation longitudinale d'une onde mécanique dans un milieu élastique, la force F(x,t) et la vitesse particulaire v(t,x) sont proportionnels. Le rapport de proportionnalité entre ceux deux grandeurs étant ${A_t E_t}/{c_t}$, c'est-à-dire, l'impédance mécanique de la tige $Z_t$.

\subsubsection{Énergie transportée par l'onde}

La propagation d’une onde mécanique dans un milieu élastique, tel qu’une tige, est caractérisée par le transport d’énergie. Selon \cite{Fairhurst1959}, pour un pulse dont toutes les composantes se propagent dans une même direction, l'énergie totale transmise ($E_t$) est divisé également en deux parties : l'énergie potentielle (ou énergie de déformation) $(E_U)$ et l'énergie cinétique $(E_K)$.

\begin{equation} \label{eq:wave_EK}
    E_K=\frac{1}{2} \frac{A_t E_t}{c_t} \int v(x,t)^2 dt
\end{equation}


\begin{equation} \label{eq:wave_EU}
    E_U=\frac{1}{2} \frac{c_t}{A_t E_t} \int F(x,t)^2 dt
\end{equation}

L'énergie totale transmise étant le double de $E_U$, celle-ci peut être obtenue en intégrant la force mesurée expérimentalement à l'aide des jauges de déformation (équation \ref{eq:wave_EF2}). Cette méthode a été appliqué à l'essai SPT afin de connaître la vrai énergie transmise par l'impact du marteau. Plus tard, en appliquant la proportionnalité entre les signaux de force et de vitesse (Équation \ref{eq:wave_FZv}), \citep{SyA.;Campanella1991} proposent le calcul de l'énergie transmise à partir de la force (à l'aide des jauges de déformation) et la vitesse particulaire (à l'aide d'accéléromètre)(Équation \ref{eq:wave_EFV}). Cette méthode est connue comme EFV, tandis que l'obtention de l'énergie uniquement à partir de la force est connue comme méthode EF2.

\begin{equation} \label{eq:wave_EF2}
    E_t^{EF2}= \frac{c_t}{A_t E_t} \int F(x,t)^2 dt      
\end{equation}

\begin{equation} \label{eq:wave_EFV}
    E_t^{EFV}= \int F(x,t) v(t,x) dt
\end{equation}

La Figure \ref{fig:num_spt} présente les signaux calculés et ceux obtenus à partir de simulations numériques avec et sans enclume. Pour le cas avec enclume, on observe que la durée du choc, c'est-à-dire le temps de contact entre le marteau et l'enclume, est très court. Le marteau reste muni d'une vitesse rémanente et un deuxième impact aura lieu environ 1,1 ms après le premier impact.

\begin{figure}[H]
	\centering
    \includegraphics[width=\linewidth]{media/num_spt.png}
	\caption{Comparatifs analytique-numériques pour deux cas : sans enclume (à gauche) et avec enclume (à droite)}
          \label{fig:num_spt}
\end{figure}

On a constaté un bonne correspondance entre les prévisions analytiques et les résultats numériques. Cela permet de conclure qu'un modèle numérique en différences finies comme qu'on propose est capable de simuler correctement le battage et la propagation des ondes au sein des tiges.


\subsubsection{Calibration du battage}
\label{subsubsec:Calibration du battage}

Lorsque le marteau heurte la tête de battage, il se crée une onde de compression dont les caractéristiques - durée, amplitude et forme - dépendent essentiellement des caractéristiques géométriques et mécaniques des différents éléments : marteau, tête de battage, train de tige.

Pour que la sollicitation dynamique simulée dans le modèle numérique soit véritablement représentative de celle observée expérimentalement, il s'avère nécessaire de d'assurer que l'onde d'impact soit comparable à l'onde expérimentale. On s'intéresse par la suite à la caractérisation de l'onde mesurée expérimentalement et à l'ajustement de l'onde de compression numérique. Pour ce faire, une série d’essais ont été réalisée à l’aide d’un pendule en laboratoire. Le but est de caractériser l’onde engendrée par l'impact du marteau sous des conditions bien définies.

L'essai consiste à disposer le pénétromètre Panda 3 horizontalement reposé sur des roulements à billes et à effectuer le battage à l'aide d'un pendule de choc. Cela afin de maintenir son horizontalité au cours du battage et de s’affranchir des efforts de frottement le long des tiges. La disposition horizontale du système n’a aucune influence sur la propagation de l’onde car les forces gravitationnelles ne modifient pas la propagation de l’onde \citep{Fairhurst1959}.

L'utilisation du pendule à choc permet de maîtriser l’énergie de battage et, par conséquent, la vitesse de l'impact. Cet appareillage permet facilement de faire varier l’énergie de battage ainsi que la longueur du train de tiges. On emploie un train de tige assez long (à savoir de 7 m) afin d'empêcher la superposition des ondes incidentes et réfléchies en sein du train de tiges. Comme conditions aux limites, la pointe à l'extrémité du train de tige est fixée horizontalement. Le marteau est lâché en condition de pendule et heurte la tête du pénétromètre (voir Figure \ref{fig:num_exp}).

  \begin{figure}[H]
   \begin{center}
        \includegraphics[scale = .9]{media/num_exp.png}
        \caption{Influence de la raideur de l'interface tête-marteau}
          \label{fig:num_exp}
      \end{center}
 \end{figure}


L'énergie de battage ($E_b$) et la vitesse d'impact ($v_m$) du marteau sont exprimées respectivement par les équations \ref{eq:Eb} et \ref{eq:vm}.

\begin{equation} \label{eq:Eb}
    E_b = m g L (1-\cos{\theta})
\end{equation}

Où m est la masse du marteau, g est l’accélération de la pesanteur, L est la longueur du pendule et $\theta$ est l’angle de chute.

\begin{equation} \label{eq:vm}
    v_m = \sqrt{2 g L(1-\cos{\theta})}
\end{equation}

Une série d'impacts a été réalisé en faisant varier l'angle de chute $\theta$ (20$\degree$, 35$\degree$, 70$\degree$, ...). Pour chaque valeur de $\theta$, on a réalisé une série d'impact afin de vérifier la répétabilité du signal et de définir la durée de l'onde et la force maximale moyenne. Pour chaque valeur de $\theta$, on connaît l'énergie de battage (Équation \ref{eq:Eb}) et la vitesse de l'impact (Équation \ref{eq:vm}). L'énergie transmise est calculée directement à partir du signal de force et des caractéristiques de la tige (section $A_t$, module $E_t$ et célérité d'ondes $c_t$) (Équation \ref{eq:wave_EF2}).

On a simulé cet essai à l'aide d'un modèle numérique afin d'ajuster l'impact aux résultats expérimentaux. Dans le modèle, les éléments composant le pénétromètre sont modélisés par des formes cylindriques, à l'exception de la pointe conique. Étant donné la géométrie doublement axisymétrique du problème, on a fait le choix de modéliser unique un quart de celui-ci. Les dimensions adoptées sont celles présentées dans la Figure \ref{fig:num_imp}. Les différents éléments sont modélisés comme des matériaux élasto-plastiques parfaits. Les caractéristiques des éléments composant le pénétromètre (pointe, train de tige et tête) sont celles de l'acier (E = 206 GPa et $\rho$ = 7850 $kg/m^3$). 

  \begin{figure}[H]
   \begin{center}
        \includegraphics[scale = 1.]{media/num_imp2.PNG}
        \caption{(a) Schéma présentant des différents éléments composant le pénétromètre : (1) marteau, (2) tête de battage, (3) train de tige et (4) pointe conique débordante}
          \label{fig:num_imp}
      \end{center}
 \end{figure}

Le marteau est composé de deux embouts en polystyrène à haut impact (HIPS) et d'une cavité cylindrique de parois de 2 mm en acier. Celle-ci est remplie de limaille de fer. Afin de simplifier l'analyse, dans le modèle, le marteau est modélisé comme un cylindre homogène de 6 cm de diamètre et 13,6 cm de hauteur. Celui-ci est considéré comme étant composé par un seul matériau, parfaitement élastique. Le module d'Young équivalent attribué au marteau $(E_m)$ est déduit à partir de sa géométrie et des modules de l'acier et du HIPS. On a adopté des valeurs de module pour la HIPS de 0,5 GPa, selon les informations du fabricant (Tableau \ref{tab:marteau}).

\begin{table} [H]
\caption{Définition du module d'Young équivalent du marteau ($E_m$)}
\label{tab:marteau}
\centering
\begin{tabular}{cccc}
\hline
{Élément} & {Épaisseur} & {Diamètre} & {Module d'Young}\\
            -     & (mm)                & (mm)               & (Pa) \\
\hline
Embout (x2)     & 25    & 60   & 5,0 $10^{8}$ \\
Cavité centrale & 86    &D$_{ext}$: 60 / D$_{int}$: 56,4 & 2,06 $10^{11}$ \\
\hline
Marteau         & 136   & 0.75  & 2,5 $10^{10}$ \\
\hline\\
\end{tabular}
\end{table}

Pour simuler l'impact entre deux objets dans FLAC, on doit créer une interface. L'interface est accolée à une (ou les deux) faces qui interagissent. Son comportement est caractérisé par des paramètres de rupture ($\phi$ et c) et de déformation (raideur normale $k_n$ et tangentielle $k_s$). 

Une fois le module du marteau défini, on s'intéresse à ajuster la raideur normale de l'interface afin d'avoir la durée de choc observée expérimentalement ($\approx$ 1,5 ms). La Figure \ref{fig:num_raideur} présente les signaux de force mesurée dans tige pour les différentes valeurs de raideur $k_n$ testées. On présente ainsi le signal expérimental moyen pour la même vitesse d'impact (6,25 m/s). On constate une raideur de $2,0 . 10^{10}$ Pa/N permet d'engendrer une onde représentative de celle obtenue expérimentale en terme de durée de choc ($\approx$ 1,25 ms) et contrainte maximale $\sigma_{max}$ (Figure \ref{fig:num_raideur}). 

  \begin{figure}[H]
   \begin{center}
        \includegraphics[scale = .5]{media/num_radieur.png}
        \caption{Influence de la raideur de l'interface tête-marteau}
          \label{fig:num_raideur}
      \end{center}
 \end{figure}

De manière similaire à la démarche expérimentale, on a simulé une série d'impact en faisant varier la vitesse d'impact : 1,0 m/s, 3,0 m/s et 5,0 m/s. Pour chacun de ces impacts, on enregistre le signal de force, conformément à l'essai Panda 3, sur la tige à 10 cm à partir de la face supérieure de la tête (voir Figure \ref{fig:num_vitesse}).

Les résultats expérimentaux ainsi que numériques montrent que l’accroissement de la vitesse d’impact $v_m$ entraîne une augmentation de la valeur $\sigma_{max}$. Une relation linéaire est observée entre la vitesse d’impact du marteau $v_m$ et la contrainte au pic $\sigma_{max}$ de l’onde de compression mesurée sur la tige. Toutefois, la pente observée expérimentalement est nettement plus faible que celle issue des simulations numériques.

   \begin{figure}[H]
   \begin{center}
        \includegraphics[scale = .5]{media/num_vitesse.png}
        \caption{Relation entre la contrainte maximale ($\sigma_{max}$) et la vitesse d'impact ($v_m$) : comparaison entre les résultats numériques et expérimentaux}
          \label{fig:num_vitesse}
      \end{center}
 \end{figure}
 
En effet, les différents sources de pertes énergétiques présentes dans l'essai ne sont par intégrées dans le modèle numérique celui-ci étant non-amorti. Afin de corriger cela, on introduit un coefficient correcteur  «\emph{a}» qui sera appliqué à la vitesse d'impact expérimentale afin de déduire la vitesse devant être utilisée dans le modèle numérique.


%On sait que l'onde de compression dépend de la géométrie et des caractéristiques mécaniques du marteau et de la tête. On fixe la géométrie et les caractéristiques mécaniques de la tête et on fait varier la géométrie et les caractéristiques mécaniques du marteau afin d'obtenir un signal de force représentative.

%Pour calibrer l'impact, on suit une démarche similaire à celle utilisé les travaux numérique précédents \cite{Benz-Navarrete2009MESURES2,Escobar2015Mise3,Tran2019NumericalTest}. On teste différentes valeurs de raideur de contact tête-marteau capable d'assurer un signal de force dont la durée est celle du signal expérimentale et la relation entre l'amplitude et la vitesse de frappe sont respectées. Les résultats expérimentaux sur lesquels on va se baser sont ceux obtenus par \cite{Benz-Navarrete2009MESURES2}.

%De manière similaire à la démarche expérimentale suivie par \cite{Benz-Navarrete2009MESURES2}. La géométrie du pénétromètre est celle du Panda 3. On utilise une tige assez longe (de 8 mètres) afin d'empêcher la superposition des ondes au sein de la tige. On enregistre les forces à la proximité de la tête suite à une série d'impact à différentes vitesses de frappe.

%\include{rapports/calibration_battage}

%Pour ce faire, on crée un impactant cylindrique de section identique à la tige (diamètre de 14 mm) et de hauteur 14 mm. La masse de l'impactant est de 422.5 g (soit un quart de celle du marteau réel, étant donné la géométrie axisymétrique adoptée).

%L'interaction entre l'extrémité de la tige et l'impactant est traduite par l'interface collée à la face inférieure est ceci. La raideur linéaire adoptée pour cette interface a été calibré à partir des données expérimentales. Pour cela, on procède de manière similaire à l'approche suivie par \citep{Tran2019NumericalTest}. On réalise une série d'impacts à différentes vitesses de frappe (1 kN, 2 kN, 5 kN) en testant différentes raideurs jusqu'à que la relation entre la force maximale et la vitesse de frappe soit cela observée expérimentalement.

\subsubsection{Simulation de l'essai DCLT}

Grâce à l'instrumentation et l'analyse des ondes appliqués au pénétromètre Panda 3, il est possible de suivre des les forces, vitesses et déplacements en pointe suite à chaque l'impact. Cela constitue un des grands atouts et une des principales particularités de cette technique. Le résultat de l'essai étant souvent synthétisé sous la forme d'une courbe dynamique de chargement en pointe.

Pour que le modèle numérique soit représentatif de cet essai, il doit permettre de reconstruire les signaux de force, vitesse et déplacement en pointe à partir de mesures de signaux de force et de vitesse obtenus à proximité de l'enclume.

A chaque impact, l’onde de choc engendrée parcours le train de tige jusqu'à la pointe. Au niveau de la pointe, une partie est transmise au sol et une partie est réfléchie vers le point de mesure à proximité de l’enclume. En supposant que les efforts extérieurs (frottement latéral) appliqués soient négligeables et que les tiges soient des corps élastiques, homogènes et de section uniforme, la propagation de l’onde de compression u(x,t) est décrite par l'équation de l'onde (équation \ref{eq:wave}). La solution correspond à la superposition de deux ondes (descendante $u_d$ et remontante $u_u$). La connaissance des ondes $\varepsilon_d$ et $\varepsilon_u$ permet de décrire entièrement le phénomène. Toutefois, dans la pratique la durée du chargement dépasse le temps de aller-retour de l’onde dans la tige donc celles-ci se superposent. Il s’avère nécessaire d’appliquer des méthodes de découplages afin de les séparer. 

Différentes méthodes peuvent être employées pour réaliser le découplage des ondes $\varepsilon_d$ et $\varepsilon_u$ \citep{Lundberg1990, Casem2003, Jung2005, Othman2001}. Celles-ci diffèrent quant au type de mesure utilisée (déformation et/ou accélération), quant à la quantité de capteurs à utiliser ainsi que par l’imposition de certaines conditions aux limites. \cite{Benz-Navarrete2009} a montré que la méthode proposée par \cite{Casem2003} et par \cite{Jung2005} est celle qui s’adapte le mieux à cet essai. Celle-ci utilise une mesure de déformation et une mesure de vitesse. Cette solution est assez précise et aucune condition aux limites ne doit être vérifiée. Il s'agit de la technique de découplage des ondes souvent appliquée lors des essais de chargement dynamique de pieux. 

Ainsi, à partir des enregistrements de déformation $\varepsilon_A(t)$ et de vitesse $v_A(t)$ réalisés sur les tiges dans une section A au voisinage de l’enclume, les ondes $\varepsilon_d$ et $\varepsilon_u$ sont découplées selon les Équation \ref{eq:decouplage_down} et \ref{eq:decouplage_up} respectivement. 

\begin{equation} \label{eq:decouplage_down}
    \varepsilon_d (t) = \frac{1}{2} \left[ \varepsilon_A (t) - \frac{v_A(t)}{c_t} \right]
\end{equation}

\begin{equation} \label{eq:decouplage_up}
    \varepsilon_u (t) = \frac{1}{2} \left[ \varepsilon_A (t) + \frac{v_A(t)}{c_t} \right]
\end{equation}

En connaissant les signaux $\varepsilon_A(t)$ et $v_A(t)$, il est possible de reconstruire les signaux de déformation $\varepsilon(t)$, de force $F(t)$ et de vitesse particulaire $v(t)$ pour une section donnée C situé en dessous du point de mesure. Cela en supposant que les efforts extérieurs (frottement latéral) nuls entre le point de mesure (A) et la section analysée (C). Avec une tige homogène et de section constante $A_t$ (impédance uniforme $Z_t$), la reconstruction des signaux de force $F(t)$ et de vitesse $v(t)$ est donné par les Équations \ref{eq:reconstruction_F} et \ref{eq:reconstruction_v} respectivement. Soit $\Delta x = x_C - x_A$.


\begin{equation} \label{eq:reconstruction_F}
    F (t) = \frac{E_t A_t}{2} \left[ \varepsilon_A (t + 2 \Delta x / c_t) + \varepsilon_A (t) \right] + \left[ \frac{Z_t}{2}  v_A (t) + v_A (t + 2 \Delta x/c_t)  \right]
\end{equation}


\begin{equation} \label{eq:reconstruction_v}
    v (t) = \frac{1}{2} \left[ v_A \left( t + \frac{2 \Delta x}{c_t} \right) + v_A (t)  \right] + \left( \frac{E_t A_t}{2 Z_t} \right) \left[ \varepsilon_A \left( t + \frac{2 \Delta x}{c_t} \right) + \varepsilon_A (t) \right]
\end{equation}

Le déplacement s(t) d'un point de la section C est déduit par la suite par intégration du signal de vitesse v(t) (Équation\ref{eq:reconstruction_s}).

\begin{equation} \label{eq:reconstruction_s}
    s (t) = \int{v(t)} \,dt      
\end{equation}

On a appliqué ce principe au modèle numérique afin de vérifier s'il est capable de reproduire la technique intégrée dans l'essai DCLT. Pour ce faire, on a simulé l'essai DCLT au sein d'un massif élasto-plastique parfait représentant un sol sec. Pour cette analyse, les paramètres du massif sont ceux du sable de Fontainebleau à l'état moyennement dense : E de 100 MPa, $\nu$ de 0,33, $\phi$' de 35° et $\psi$ de 5°.

La géométrie et les conditions aux limites sont celles décrites dans la section \ref{subsec:Géométrie et conditions aux limites} de ce document. Le battage est simulé comme décrit dans la section précédente. On a enregistré les historiques de force, de vitesse et de déplacement à la proximité de l'enclume (section A) comme montré sur la Figure \ref{fig:num_imp}, et en pointe (section C). Cela pour les 40 milliseconds suivants l'impact.

La Figure \ref{fig:num_technique} (a) présente les signaux de force et de vitesse mesurés à la proximité de l'enclume (point A). Ceux-ci permettent d'estimer les ondes descendante et remontante au sein de la tige (Figure \ref{fig:num_technique} (b)). Une fois les ondes découplées, on présente le signal de force estimées (Équation \ref{eq:reconstruction_F}) et celui mesuré au niveau de la pointe. Le signal de vitesse et de déplacement en pointe estimées sont également comparés aux signaux mesurés en pointe (Figure \ref{fig:num_technique} (c)). On a constaté une bonne correspondante entre les signaux estimés et mesurés.
\newpage

\begin{figure}[H]
\centering
\begin{subfigure}[H]{.85\linewidth}
\includegraphics[width=\linewidth]{media/num_F_Zv.PNG}
\caption{Mesures dans la tige (point A) : force ($F_A (t)$) et produit de la vitesse ($v_A (t)$) et de l'impédance en fonction du temps}\label{fig:num_F_Zv}
\end{subfigure}
\begin{subfigure}[H]{.85\linewidth}
\includegraphics[width=\linewidth]{media/num_tech_Ft.PNG}
\caption{Résultats des calculs pour signaux de force des ondes : onde descendante $F_d (t)$, onde remontante $F_u (t)$ et onde transmise estimée en pointe $F(t)$ (Equation \ref{eq:reconstruction_F}) ; signal de force mesuré en pointe}\label{fig:num_tech_Ft}
\end{subfigure}
\begin{subfigure}[H]{.85\linewidth}
\includegraphics[width=\linewidth]{media/num_tech_vt_st.png}
\caption{Signaux de vitesse et de déplacement en pointe : signaux estimés et signaux mesurés en fonction du temps}\label{media/num_tech_vt_st.png}
\end{subfigure}
\caption{Application de la technique de découplage et reconstruction des ondes pour un impact avec $v_{m}$ = 1,75 m/s}
\label{fig:num_technique}
\end{figure}

Le résultat de l'essai DCLT est souvent synthétisé par la courbe de chargement en pointe présentant la contrainte en pointe en fonction du déplacement en pointe. Qualitativement, la forme de la réponse obtenue est conforme à ce qui peut être classiquement observé expérimentalement (voir Figures \ref{fig:a} à \ref{fig:d} du chapitre précédent).

La réponse mécanique se compose d’une première phase de montée en charge rapide correspondant à l’augmentation initiale de la vitesse de la tige. Dans un second temps, l’effort sur la tige reste positif pendant l’enfoncement de la tige jusqu’à sa pénétration maximale. Enfin, on observe une phase de déchargement-rechargement correspondant à la stabilisation de la tige autour d’une position d’équilibre finale.

 \begin{figure}[H]
   \begin{center}
        \includegraphics[scale = .5]{media/num_DCLT.png}
        \caption{Courbe DCLT estimées et courbes DCLT mesurées pour trois vitesses d'impact (1,75 m/s, 3,5 m/s et 7 m/s)}
          \label{fig:num_DCLT}
      \end{center}
 \end{figure}

Enfin, on présente la courbe DCLT pour l'impact en question (signaux de la Figure \ref{fig:num_technique}) et pour d'autres deux impacts avec des vitesses de frappe différentes, à savoir de 3,5 m/s et de 7 m/s. On observe une bonne correspondance pour les trois impacts.  Par ailleurs, les courbes obtenues sont aussi qualitativement et quantitativement comparables aux résultats numériques présentés par \cite{Tran2019} obtenus pour les mêmes énergies de frappe.


\subsection{Modèle de comportement du sol}
\label{subsec:comportement}
  
Divers travaux numériques s’intéressent à étudier et à modéliser le comportement des sols lors de la pénétration. A l'heure actuelle, la plupart de modèles numériques appliquent des critères de rupture simple : Mohr-Coulomb (MC), Tresca, von Mises et Cam-clay pour les argiles non-drainées et MC et Drucker-Prager pour les sables drainés.

Le critère de Mohr-Coulomb est utilisé afin de modéliser le comportement des sols pulvérulents (sables), des sols cohérents à long terme (argiles et limons) et de certaines roches. Les paramètres de rupture alimentant ce critère sont : la cohésion c, l’angle de frottement $\phi$ et la dilatance $\psi$. 

En mécanique des sols, la cohésion et l’angle de frottement sont traditionnellement calculés dans le plan de Mohr ($\sigma$’, t) à partir des états de contraintes à la rupture, estimés pour chaque essai triaxial. La loi de Tresca, qui est un cas particulier de la loi de Mohr-Coulomb, est utilisée pour l’étude des argiles et des limons à court terme ($\phi$=0\degree et $\psi$=0\degree) \citep{MestatJean-Pierrre;Magnan1991}.

Selon \cite{Ahmadi2005b}, ce critère de rupture simple est capable de traduire correctement le comportement des sables. Les auteurs l’appliquent afin de modéliser le comportement d’un sable purement frottant (comme le sable de Ticino). L’angle de frottement est obtenu à partir d’une série d’essais triaxiaux tandis que l’angle de dilatance est estimé à partir de l’angle de frottement effectif $\phi$’ et l’angle de frottement à l’état critique $\phi_{cv}$.

\begin{equation} \label{eq:dilatance}
    sin \psi' = sin \phi'- sin \phi_{cv}
\end{equation}

On souhaite modéliser le comportement d’un sable propre : le sable de Fontainebleau. On a vu que le critère de rupture Mohr-Coulomb est adapté à ce type de matériau et pour les modèles numériques rapportés. C’est pourquoi on choisit de l’appliquer dans le cadre de travail. On s'appuie sur les résultats d'une série d'essais triaxiaux.

On souhaite ajuster le modèle rhéologique adopté dans le modèle numérique aux résultats expérimentaux. Pour ce faire, on va se baser plus précisément sur les courbes CLT obtenues à 30 cm de profondeur pour le sable de Fontainebleau moyennement dense ($\rho_d=1575 kg/m^3$, soit DR = 58\%) et lâche ($\rho_d=1490 kg/m^3$, soit DR = 43\%) en chambre d'étalonnage (Figure \ref{fig:exp_eprouvettes}).

 % \begin{figure}[H]
 %  \begin{center}
 %       \includegraphics[scale = .45]{media/exp_ajustement_modele.png}
  %      \caption{Courbes CLT expérimentales : sable de Fontainebleau moyennement dense (profondeur = 30 cm)}
  %        \label{fig:exp_ajustement_modele}
  %    \end{center}
 %\end{figure}

On adopte initialement un modèle rhéologique simple : élastoplastique parfait avec critère Mohr-Coulomb. On aborde par la suite le choix des paramètres nécessaires pour alimenter ce modèle rhéologique, à savoir : module d'Young ($E_{Young}$), coefficient de Poisson $\nu$, angle de frottement $\phi$, angle de dilatance $\psi$ et la cohésion c.

\subsubsection{Paramètres rhéologiques alimentant le modèle}
\label{subsubsec:fontainebleau}

L’angle de frottement $\phi$ est couramment compris entre 15\textdegree{} et 45\textdegree{}. Les valeurs inférieures ou autour de 30\textdegree{} sont typiques des argiles, tandis que des valeurs supérieures, entre 25\textdegree{} et 45\textdegree{}, caractérisent les sables. A compacité donnée, l’angle de frottement est pratiquement indépendant de la teneur en eau du sol.

En supposant la variation de $\phi'$ avec le confinement est négligeable, on a tracé l'évolution de $\phi$ en fonction de l'état de densité de trois éprouvettes testées à l'appareil triaxial. Le but étant d'estimer l'angle de frottement correspondant à l'état de densité des éprouvettes réalisées (e = 0,68 et e = 0,78, voir Figure \ref{fig:exp_eprouvettes}). La relation entre l'indices de vides et l'angle de frottement est présentée dans la Figure \ref{fig:exp_phi_SH_SF_tx}. Les valeurs de $\phi'$ obtenues sont : 30,8\textdegree{} et 35,9\textdegree{} pour les éprouvettes lâches et moyennement dense, respectivement.

L’angle de dilatance $\psi$ est quant à lui généralement compris entre 0\degree et 15\degree. Les sables lâches et les argiles ont des valeurs de dilatance très faibles, quelques degrés à peine voire zéro. Il est souvent déterminé en laboratoire à partir de résultats de l'essai triaxial. Il existe également des relations empiriques simples reliant l’angle de dilatance et l’angle de frottement interne \citep{Levasseur2007}. De manière similaire à \cite{Ahmadi2005a}, on estime l'angle de dilatance à partir de l'angle de frottement en rupture et l'angle de frottement critique (Équation \ref{eq:dilatance}). Comme évoqué, les valeurs rapportées dans la littérature pour le $\phi_{cr}$ varient peu. Celles-ci sont comprises entre 29,3\textdegree{} et 31,0\textdegree{} \citep{Dano2001, Dupla2007, Vernay2017, Andria-Ntoanina2011}.

%Les sols pulvérulents n’ont pratiquement pas de cohésion, 0 < c < quelques kiloPascals. Pour ce type de matériau, on parle de cohésion capillaire ou de cimentation en place.

%Les valeurs de cohésion obtenues à l'appareil triaxial restent inférieures à 15 kPa. 
%Ainsi, dans le modèle, on considère la cohésion nulle.  

%Les valeurs habituellement retenues pour le coefficient de Poisson sont  $\nu$=$\nu_u$ = 0,5 si les sols sont saturés (déformation à volume constant). Dans la pratique la valeur de 0,5 n’étant pas acceptable en élasticité, on utilise l’expédient consistant à considérer $\nu_u$ = 0,49, $\nu_u$ = $\nu_u$ = 0,33 dans les autres cas (sols secs ou sols drainés).

Le module d'élasticité alimentant la loi de comportement est déduit à partir des résultats triaxiaux. Dans un premier temps, la valeur du module d'Young de chaque éprouvette est déterminée à partir d'un calage élasto-plastique parfait des courbes contrainte-déformation issues des essais triaxiaux (Figure \ref{fig:exp_tx_Eyoung2}a). Par la suite, afin d'intégrer l'effet de l'évolution du module avec le confinement, on applique la relation proposée par Janbu (1963) (Équation \ref{eq:EJanbu}) pour estimer les modules d'Young des éprouvettes. A partir des modules obtenus par le calage, on a tracé l'évolution du module en fonction des confinements pour les différentes densités testées (Figure \ref{fig:exp_tx_Eyoung2}b).

\begin{equation} \label{eq:EJanbu}
    E = E_{ref} \frac{\sigma_3}{P_{ref}}^n
\end{equation}
    
Soit

$P_{ref}$ : pression atmosphérique (souvent prise égal à 100 kPa)

n : coefficient compris entre 0,33 et 1 (souvent pris égal à 0,5)

\begin{figure}[H]
    \begin{center}
        \includegraphics[scale = .6]{media/num_tx_SF.PNG}
        \caption{(a) Exemple du calage élasto-plastique parfaite appliqué courbes contrainte-déformation : éprouvettes de SF à 1496 kg/$m^3$, (b) Évolution du modules d'Young du modèle en fonction du confinement}
        \label{fig:exp_tx_Eyoung2}
  \end{center}
\end{figure}



%$E_{ref}$ = module de référence 

%$\sigma_3$ = la pression de confinement 

%$P_{ref}$ = pression de référence (atmosphérique, 100 kPa)

%n = coefficient compris entre 0,33 et 1,0

%La pression de référence $P_{ref}$ est souvent prise égale à la pression atmosphérique ($P_{ref}$=100 kPa). L'exposant n est un coefficient compris entre 0,33 et 1. 

%Le modèle élasto-plastique parfait avec critère de Mohr-Coulomb utilisé est un modèle de comportement simple et robuste. Il peut être considéré comme une approximation au premier ordre du comportement réel d’un sol. Ce modèle constitue une approche acceptable pour analyser la rupture des sols dans les problèmes de fondation, stabilité de pentes, de tunnel et de soutènement. Cependant, il suppose une rigidité constante pendant la phase de chargement alors qu’en réalité cette rigidité n'est pas constante. Elle dépend à la fois du niveau de contrainte et du niveau de déformation du sol. De plus, expérimentalement il est possible d’observer des déformations irréversibles même pour des chemins de contraintes qui n’atteignent jamais la rupture \citep{Levasseur2007AnalyseGenetiques}.

%Ce travail a pour but d'étudier la déformabilité des sols de manière plus réaliste, en tenant compte de la niveau de déformation associé aux valeurs de modules. L'application d'un modèle de comportement élasto-plastique parfait est donc insuffisant. C'est pourquoi, il choisit d’adopter un modèle élastoplastique plus complexe qui tienne compte de l’irréversibilité du comportement des sols. Un modèle assez répandu permettant d'intégrer cela est le HSM (\emph{Hardening Soil Model} ou, en français, modèle élastoplastique avec écrouissage).

%La loi de comportement HSM implémentée dans FLAC est celle présenté par \cite{Schanz1999TheVerification}. Celle s'est basée sur le comportement hyperbolique initialement proposé par Duncan et Chang (1970). La surface de rupture déviatoire est donnée par le modèle de Mohr-Coulomb.

\section{Bilan}

On a présenté une partie des travaux réalisées autour du modèle numérique. Un certain nombre d'améliorations ont été mis en place au cours de cette année afin d'avoir un modèle plus représentatif de l'essai DCLT :

- Géométrie: celle-ci respecte maintenant la forme et les dimensions du pénétromètre.

- Sollicitation: a la place d'un signal de force en tête, le battage a été simulé après calibration pour obtenir la bonne vitesse d'impact.

On s'est intéressé par la suite à la loi de comportement et au choix des paramètres rhéologiques. On a proposé d'alimenter le modèle de comportement à l'aide des résultats triaxiaux tout en tenant compte des principaux facteurs influençant ces paramètres : l'état de densité et la pression de confinement. 

Les améliorations réalisées visent à atteindre un des objectifs de ce travail de thèse : mettre au point un modèle numérique véritablement représentatif de l'essai DCLT.


%%%%%%%__________________________________________________
\subsection{Vérification du modèle de comportement du modèle numérique}
\label{subsec:vérification}

On simule les essais de chargement en pointe dynamique et statique au sein des massifs élasto-plastique parfait et HSM. Les paramètres des modèles de comportement sont ceux obtenus à l'appareil triaxial pour des échantillons des sables étudies (SF et SH).

En effet, on observe que le module d'élasticité nécessaire afin d'avoir une correspondance avec les résultats expérimentaux doit être considérablement plus important que ceux obtenu à l'état de densité et une contrainte moyenne à l'essai triaxial. En effet, le rapport entre le module nécessaire est celui qu'on attribue au matériau est de 6. 

Cette différence a été observé par d'autres travaux comparant des courbes numériques et expérimentales de chargement de pieu. Ces travaux ont observé des rapports du même ordre \citep{Pinto2008, Masouleh2008, Fakharian2014}.

Plus de cas d'étude avec des chargements en pointe sont essentielles afin d'expliquer le sujet des valeurs de module plus important nécessaire. La création de cavité eu sein du sol lors de la pénétration entraine des changements à l'état de contrainte initial, notamment une augmentation de la contrainte radiale (horizontale) vis-à-vis de la contrainte initiale correspondante à l'état $K_0$.


%%%%%%%__________________________________________________

\chapter{Conclusion et perspectives}

Dans ce rapport, après un rappel du contexte et des objectifs, on a présenté une partie du travail réalisé au cours de la deuxième année de la thèse.

En laboratoire, deux sables de référence ont été testés. Une série d'essais triaxiaux a été réalisé à différentes densités et à différentes pressions de confinement. Ces essais vont permettre d'une part d'alimenter la loi de comportement du modèle numérique et aussi de valider les paramètres obtenus à partir de l'exploitation de l'essai DCLT.

Par ailleurs, une série d'essais en chambre d'étalonnage ont été effectué avec ces sables. Ces essais de chargement en pointe permettront d'ajuster la loi de comportement appliquée dans le modèle numérique. Additionnellement, ces essais de chargement en pointe permettent de comparer les deux types de sollicitation : dynamique et statique sous conditions semblables. On a pu constater que, pour le cas de figure analysé, les courbes DCLT sont qualitativement comparables aux courbes CLT. Une analyse qualitative des résultats permet également de confirmer la sensibilité et la répétabilité de la cette technique.

Du côte numérique, on a apporté un certain nombre d'améliorations au modèle. La géométrie actuelle est plus réaliste vis-à-vis de la forme et dimensions du pénétromètre (sections circulaires, dimensions conformes à celles du pénétromètre e et de la chambre d'étalonnage). Par rapport à sollicitation dynamique, au lieu d'appliquer un signal de force sinusoïdale, comme réalisé dans la plus part de travaux rapportés, on a choisi de simuler l'impact. Cela permet d'être plus réaliste par rapport à la technique et de mieux se caler par rapport aux résultats expérimentaux.

L'étude bibliographique réalisée à propos des modèles d'interaction a permis de souligner leur points convergeant et leurs limitations. A l'issue de cette étude, on envisage d'appliquer un modèle d'interaction sol-pointe plus réaliste (comme à celui de \cite{Loukidis2008}) afin de dissocier la composante dynamique et de pouvoir appliquer des modèles d'interprétation comme l'expansion de cavité à cette technique. 

En tenant compte des aspects abordés dans ce rapport, les principales activités prévues pour la troisième année sont : 
\begin{itemize}
    \item valider le modèle numérique sur la base des résultats expérimentaux obtenus en chambre d'étalonnage pour des sables propres de référence ;
     
    \item définir la méthode de détermination des paramètres alimentant le modèle d'interaction sol-pointe (résistance ultime $Q_{bL}$, raideur maxille, $K_{max}$, coefficient visqueux m et n, ...) ;
    
    \item à partir des courbes statiques estimées, appliquer la méthode d'expansion de cavité et évaluer les valeurs de $\phi$' obtenues sur la base de résultats triaxiaux ;
    
    \item évaluer les modules issues de l'essai DCLT sur la base des modules issus des résultats triaxiaux.
\end{itemize}


\newpage
%\thispagestyle{empty}

\begin{center}
\textbf{VALORISATION DES TRAVAUX}
\end{center}

\begin{itemize}
    \item \underline{Conférences nationales} :\\[0.005cm]
    
\emph{Oliveira, C. F. ; Breul, P. ; Benz-Navarrete, M. A. ; Tran, Q. A. Essais de chargement dynamique et statique en pointe pénétrométrique : comparatifs au laboratoire. 37èmes Rencontres Universitaires de Génie Civil de l’AUGC, Sophia Antipolis, juin 2019.}\\

\emph{Oliveira, C. F. ; Breul, P. ; Chevalier, B. ; Benz-Navarrete, M. A. ; Tran, Q. A. ; Bacconnet, C. Modélisation du battage dynamique d'un pénétromètre en différences finies. Journées Nationales de Géotechnique et de Géologie de l’Ingénieur (JNGG), Lyon, novembre 2020.}\\[0.025cm] 
    
\item \underline{Conférences internationales} :\\[0.005cm]

\emph{Oliveira, C. F. ; Benz-Navarrete, M. A. ; Breul, P. ; Tran, Q. A. ; Chevalier, B. ; Bacconnet, C. Assessment of small-strain modulus through wave velocity measurement with dynamic penetrometer Panda 3®. 6th International Conference on Geotechnical and Geophysical Site Characterization, Budapest, Septembre 2021.}

\end{itemize}

%%%________BIBLIO__________
\newpage
\lhead{Bibliographie}
\nocite{}
\bibliographystyle{apalike}
\bibliography{library}

  
 \end{document}


