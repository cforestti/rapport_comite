\documentclass{article}
\usepackage[utf8]{inputenc}
\usepackage[sectionbib]{natbib}
\usepackage[french]{babel}
\usepackage[titletoc]{appendix}
\usepackage{chapterbib}
\usepackage{url}
\usepackage{amsmath}
\usepackage{graphicx}
\usepackage{fancyhdr}
\usepackage{lmodern}
\usepackage{vmargin}
\usepackage[T1]{fontenc}
\usepackage{float}
\usepackage[table,xcdraw]{xcolor}
\usepackage{caption}
\usepackage{hyperref}
\usepackage{setspace}
%\usepackage{filecontents}
\usepackage{calc}
\usepackage{lipsum}
\usepackage{enumitem}
\graphicspath{ {images/} }
\usepackage{xcolor}
\usepackage{rotating}
\usepackage{gensymb}
\usepackage{siunitx}
\usepackage{textcomp}
\setcounter{secnumdepth}{3}

\pagestyle{fancy}
\fancyhf{}
%\lhead{\leftmark}
\rfoot{\thepage}

\begin{document}

\begin{titlepage}
    \begin{center}
    Thèse CIFRE N° 2018/1627
    
    Sol Solution
    
    Institut Pascal – Axe M3G
            \vspace*{5cm}
              %  \vspace*{5cm}
    
    %{Note technique}\\[0.3cm]
    
     {\textbf{Note technique : modèle phénomélogiques d'intéraction sol-pointe $c_p$ - $c_p^{DCLT}$}}\\[1.cm]
     {Caroline F. Oliveira}\\[0.3cm]

     
    \end{center}
        \vspace*{8cm}
    
    
\textbf{Encadrement}
\begin{itemize}
	\item[] \textsc{Pierre BREUL}
	\item[] \textsc{Bastien CHEVALIER}
	\item[] \textsc{Miguel Angel BENZ NAVARRETE}
	\item[] \textsc{Quoc Anh TRAN}
\end{itemize}

        \vspace*{1cm}

\centering
\today

\end{titlepage}


\section{Introduction}

Un nombre de modèles existant et appliqués dans le domaine des essais de chargement des pieux permettent d'interpréter les mesures expérimentale (signaux de force et de vitesse) afin de remonter aux paramètres des sols.

On cherche à remonter aux paramètres caractérisant le comportement des sols en déformabilité et en rupture. Cela afin de les appliquer aux dimensionnement courants en géotechnique (fondation superficielle et profondes).

Dans ce document, on s'intéresse par les principaux modèles afin de choisir et de l'adapter, si nécessaire, celui qui s'adapte le mieux à l'essai DCLT.

Dans ce document, on présente les différents modèles, ces points forts et ces limitations en appliquant à des signaux expérimentaux DCLT obtenus en laboratoire.

\section{Modèles analytiques de la réaction en pointe pour les pieux}

Différents travaux se sont intéressés à interpréter les essais dynamiques de pieux. Ceux-ci sont moins contraignants et plus économiques que les essais de chargement statique. Initialement, l'interprétation de ces essais était basée sur des relations empiriques appliquant la conservation d'énergie : l'énergie appliquée aux pieux correspondrait au travail réalisé par la résistance au pieu. Ces relations ont été d'ailleurs adoptées par un certain nombre de normes de calcul (e.g. Engineering News Record, la formule des Hollandais). Plus tard, certains chercheurs ont constaté les limitations de ces relations et la nécessité d'intégrer l'analyse de la propagation des ondes au sein du pieux afin de mieux comprendre le phénomène \citep{Gonin1979ReflexionPieux,Loukidis2008AssessmentProcedure}.

De plus en plus, les chercheurs tentent d'interpréter ce problème en prenant en compte les concepts physiques et mécaniques de la pénétration dynamiques et de l'interaction sol-pieux. Ces modèles sont connus comme modèles phénoménologiques. Ceux-ci se basent sur la seconde loi de Newton du mouvement. 

Le modèle proposé par \cite{Smith1960Pile-drivingEquation} constitue une grande avancée à la compréhension du phénomène mais reste très limité. D'autres modèles ont tenté de modéliser le phénomène à l'aide de paramètres caractérisants le sol pénétré \cite{Randolph2003ScienceDesign, Loukidis2008AssessmentProcedure}.

Les travaux dans le domaine évoluent graduellement vers une approche plus réaliste où les modèles sont alimentés par des paramètres du pieu (rayon, profondeur) et du sol ausculté.

Ils existent différents modèles analytiques décrivant la relation contrainte (ou charge) - enfoncement pour les pieux. Ceux-ci sont regroupés en modèles de la réaction latérale et modèles de réaction en pointe. Selon \cite{Loukidis2008AssessmentProcedure}, l'approche continue souvent appliquée aux modèles analytiques latéraux comme celle proposée par Randolph (1995) n'est pas adaptée à la réponse en pointe, la réaction développée en pointe ne peut pas être dissocié des mécanismes du milieu adjacent à la pointe. En plus, la réaction en pointe, au contraire de la réaction latérale, est un problème à 2 dimensions.

La déformation plastique autour de la pointe s'étend sur une zone qui va de 1 et à 2 fois le diamètre un pieu. Le mécanisme plastique est très proche de celui observé pour la capacité portante d'une semelle. Vu que le pieu possède une pointe plate, on observe la formation d'une zone conique rigide immédiatement au-dessous de la pointe. Le sol situé à l'intérieur de cette région reste dans le domaine élastique et peut-être considéré comme étant un prolongement de la géométrie du pieu. Les déformations plastiques sont observées au-delà de cette région conique. Le sol placé au-delà de la zone plastifiée \citep{Loukidis2008AssessmentProcedure}.

Différents modèles analytiques de la réaction en pointe essaient d'exprimer la réponse observée pour les essais de chargement statique de pieux en reliant à contrainte (ou la charge) à l'enfoncement, à savoir : Smith (1960), \cite{Lysmer1965VerticalFooting}, Nguyen et al. (1988), \cite{Holeyman1988ModellingBase}, Deeks and Randolph (1995) et celui proposé plus récemment par \cite{Loukidis2008AssessmentProcedure}.

Un paramètre important pour l'application des différents modèles . La résistance ultime issue de l'essai DCLT est définie comme la résistance en pointe à l'instante de vitesse nulle. Cette méthode de détermination de la résistance ultime pour des essais de chargement en pointe est connue comme Méthode du Point de Déchargement (de l'anglais Unloading Point Method (UPM) a été introduit par \citep{Middendorp2006,Holscher2012}.


\subsection{Modèle de Smith}

Le modèle proposé par \cite{Smith1960} est largement appliqué pour le battage des pieux. Il a été mis au point dans le but de remplacer les formules de battage (basés sur des concepts d’énergie), utilisant le nombre de coups, par une approche numérique plus précise.

Selon \cite{Smith1960}, la réaction du sol dépend uniquement des déplacements et de la vitesse. La réaction développée est proportionelle au déplacement jusqu'à une limite $R_u$ à laquelle correspond le \emph{quake} Q (ou enfoncement élastique). Le fonçage rapide du sol autour du pieu donne naissance à un amortissement $J_s$ représenté par un coefficient proportionnel à la vitesse de déplacement (v) (Équation \ref{eq:Rd}). La résistance totale $R_t$ est la somme des résistances statique $R_s$ et dynamique $R_d$ (Équation \ref{eq:Rt}).

Ce modèle a été appliqué par \cite{Benz-Navarrete2009} et \cite{Escobar2015} à l’essai DCLT. Il a permis de caractériser le sol par le biais d’un coefficient d’amortissement $J_s$ et de dissocier la réponse statique $R_s$ de la réponse dynamique $R_d$ dans la résistance totale mesurée. La partie statique de la résistance $R_s$ est obtenue en admettant que lorsque la vitesse d’enfoncement $v$(t) est nulle, et donc l’enfoncement en pointe est maximal, la résistance totale est due uniquement à la partie statique. A cet instant donné, $q_d = R_s$.

\begin{equation} \label{eq:Rt}
    R_t = R_s + R_d 
\end{equation}

\begin{equation} \label{eq:Rd}
    R_d = R_s J_s v 
\end{equation}

\begin{equation} \label{eq:Rd2}
    R_d = R_s J_n v^n 
\end{equation}

Bien que largement appliquée dans le domaine des pieux, la solution de Smith présente des grandes limitations. Certains travaux ont montré que l'amortissement radiatif (en anglais radiation damping) est fonction de la raideur du sol et non de la résistance. En plus, des résultats expérimentaux ont montré que la relation entre l'amortissement visqueux et la résistance statique n'est pas du type linéaire mais exponentielle \citep{Coyle1970EmpiricalClays}. Globalement, les principales limitations identifiées dans ce modèle sont : 

    • la non prise en compte de l’inertie du sol autour de la pointe ; 
    
    • les pertes d’énergie dues aux différents types d’amortissement (radiatif, l’hystérésis et à viscosité) qui ne sont pas considérées de manière isolée, mais combiné sur un seul paramètre ($J_s$) ;

    • la caractérisation du sol sur la base des paramètres d’usage peu courant en géotechnique (\emph{quake}, amortissement $J_s$).












Afin de rendre le modèle initialement proposés par \cite{Smith1960} plus réaliste, d'autres modèle ont été proposé par la suite. Ceux-ci cherchent à mettre au point des modèles alimentés par des paramètres ayant une signification physique et pouvant être déterminés en laboratoire.











\subsection{Modèle basé sur l'analogie de Lysmer}

\cite{Lysmer1965} ont proposé une solution pour le déplacement d'une semelle circulaire rigide située sur un massif semi-infini élastique parfait, homogène et isotrope soumis à un chargement vertical transitoire (type pulse). Dans cette solution, seuls les efforts normaux sont transmis au niveau de l'interface sol-semelle.

Certains travaux se sont intéressés à la détermination des tassements d'une semelle, mais la solution de Lysmer a la particularité d'être valable pour toutes les fréquences de sollicitation, tandis que les solutions précédentes (Sung, Bycroft, Hsieh) s'appliquent pour de faibles fréquences.

La réaction du sol sur la base de la semelle correspond à la somme de la réaction d'un ressort (fonction du déplacement) et de la réaction d'amortisseur (fonction de la vitesse). La raideur ($K_{Lys}$) de ce ressort est celle de la solution statique (Équation \ref{eq:KLys}).

\begin{equation} \label{eq:KLys}
    K_{Lys} = \frac{4 G R}{1-\nu} 
\end{equation}

Soit G le module de cisaillement du sol en Pascal, R le rayon de la semelle en mètres et $\nu$ le coefficient de Poisson. Le coefficient d'amortissement est donné par l'équation \ref{eq:CLys}.

\begin{equation}  \label{eq:CLys}
    C_{Lys} = \frac{3.4 R^2}{1-\nu} \rho  Vs = \frac{3.4 R^2}{1-\nu} \sqrt{\rho G}  
\end{equation}

Soit $\rho$ la masse volumique en kg/m$^3$ et $V_s$ la vitesse d'ondes de cisaillement du sol en m/s. L'amortissement dans le modèle de Lysmer représente l'amortissement radiatif, c'est-à-dire dissipation d'énergie associée la propagation d'ondes (de volume et Rayleigh) dans le milieu semi-infini. Dans ce modèle, l'amortissement ne dépend pas de la fréquence. Le modèle basé sur l'analogie de Lysmer est adapté pour un sol élastique parfait soumis à un chargement vertical transitoire (pulse). La conformité de la solution générale et les résultats pour un milieu semi-infini reste suffisante pour les applications aux problèmes d'ingénierie. 

La solution de Lysmer est largement employée dans le calcul de fondations de machine. Ce modèle suppose un comportement élastique parfait et le sol présente un comportement élastique restreint à des déformations très faibles. Par conséquent, le module G adapté pour déterminer la raideur K correspond au module à faibles déformations ($G_{max}$ ou $G_0$). Certains chercheurs suggèrent l'utilisation du module associé à la gamme de déformation du problème traité, mais la détermination la gamme de déformation associée au problème n'est pas simple (e.g. fonction de la distance de la source, du temps,...). Quant au module permettant de déterminer $C_{Lys}$, ceci est strictement le module à faibles déformations ($G_{max}$).

Le modèle de Lysmer a l'atout d'associer le ressort et l'amortissement de Smith à des paramètres courants du sol (paramètres élastiques et masse volumique). Par conséquent, ce modèle réduit l'empirisme du modèle de Smith, par contre il ne tient pas compte des certains aspects importants de la réaction du sol. On peut citer notamment la non-linéarité de la réponse du sol et les effets visqueux dans la résistance. Par conséquent, le modèle surestime la raideur en pointe.

En effet, la solution présentée par Lysmer est valable pour les sollicitations statiques et dynamiques (toutes les fréquences), tant que l'encastrement est nul (D=0) \cite{Loukidis2008AssessmentProcedure}. Pour les applications en profondeur, \cite{Loukidis2008AssessmentProcedure} propose l'intégration d'un facteur afin de tenir compte de l'encastrement. Ce sujet sera abordé plus tard dans ce rapport.

Récemment, \cite{Lee2020InstrumentedCharacterization} a appliqué l'amortissement matériel proposé par \cite{Lysmer1965VerticalFooting} afin de dissocier la partie dynamique de la résistance totale mesurée à l'essai au pénétromètre dynamique. Les auteurs ont montré qu'une fois la partie dynamique soustraite de la résistance totale, la résistance déduite était assez proche de la résistance pseudo-statique $q_c$ (CPT) (avec $q_c = 1,.01 q_d$ et $R^2=0.85$). 

Plus tard, les travaux de Nguyen et al. (1988), de \cite{Holeyman1988ModellingBase} et de \cite{Deeks1995ALoading} ont permis d'améliorer la solution de \cite{Lysmer1965VerticalFooting}. Ils ont notamment avancé en prenant en compte les différents types d'amortissement (de radiation et d'hystérésis) qui étaient traduits par Smith par un seul paramètre ($J_s$). En autre limitation de la solution de \cite{Lysmer1965VerticalFooting} est le fait de ne pas modéliser des milieux hétérogènes ou non-élastiques. \cite{Loukidis2008AssessmentProcedure} a proposé un modèle afin de corriger certaines de ces limitations.

\subsection{Modèle de \cite{Loukidis2008AssessmentProcedure}}

\cite{Loukidis2008AssessmentProcedure} a proposé un modèle d'interaction qui tient compte de la non-linéarité de la réponse en pointe, de l'effet visqueux dans la résistance et qui distingue les différents types d'amortissement (radiatif et d'hystérésis). La réponse totale en pointe R est la somme de la réaction du ressort $R_S$ et de l'amortisseur $R_D$ (Figure \ref{fig:ana_modele}).

  \begin{figure}[H]
   \begin{center}
        \includegraphics[scale = 1.1]{media/ana_modele.JPG}
        \caption{Modèle de la réaction en pointe proposé par \cite{Loukidis2008AssessmentProcedure}}
          \label{fig:ana_modele}
      \end{center}
 \end{figure}

\begin{equation}  \label{eq:Rt}
    R = R_S+R_D = R_S+C \dot w
\end{equation}

\begin{equation}  \label{eq:Rbs}
   \dot R_S = \frac{K_{max}}{(1+b_{fb}\frac{\mid R_S+LI R_{S,rev}\mid }{(LI+1) x \mid sgn R_{Sf}-R_S\mid})^2} \dot w
\end{equation}

La raideur non-linéaire du ressort en pointe est donnée par l'équation hyperbolique représentant la réponse charge-déplacement en pointe. Cette raideur est réduite à partir de la valeur maximale $K_{max}$ associée à un comportement purement élastique. La réaction en pointe tend asymptotiquement vers la résistance en pointe $R_{bf}$. Lorsqu'on observe un chargement de signal du déplacement, la raideur est $K_{max}$. En effet pour phase de décharge, l'allure de courbe est gouvernée par le module initial \citep{Holeyman1988ModellingBase}.

La courbure de l'hyperbole dépend du type de pieux et du sol. On assimile le pénétromètre à un pieu battu. \cite{Loukidis2008AssessmentProcedure} observe que $b_f$ varie entre 1,0 et 2,0 pour les pieux battus. L'effet visqueux dans la résistance sont pris en compte d'après la relation proposé par \cite{Coyle1970EmpiricalClays} (Équation \ref{eq:Coyle}).

\begin{equation}  \label{eq:Coyle}
    R_{Sf} = Q_{bL} (1 + m v^n) 
\end{equation}

Avec m et n des coefficients contrôlant l'effet de la vitesse. Le coefficient n présente une variation est assez petite quel que soit le sol, et est souvent pris 0,2. Le coefficient m varie entre 0,34 et 0,56 pour les sables et entre 0,95 et 1,55 pour les argiles.

Lorsque que la vitesse de nulle, $R_{Sf}$ est égale $Q_{bL}$, soit $Q_{bL}$ la résistance limite en pointe. La résistance $Q_{bL}$ est égale au produit résistance unitaire et de la section du pieu. Selon \cite{Salgado2008TheFoundations}, la résistance unitaire en pointe est approximativement la résistance quasi-statique $q_c$ issue de l'essai CPT.

La raideur non-linéaire intègre les pertes d'énergie lié à la plastification (hystérésis du matériau). Le coefficient d'amortissement (C) intègre l'amortissement radiatif et de l'hystérésis du sol plus loin de la pointe réduit graduellement avec l'augmentation lorsque la réaction du ressort s'approche de la résistance en pointe ultime. 

La raideur maximale du ressort ($K_{max}$) et le coefficient de l'amortissement (C) sont basés sur les coefficient de Lysmer ($K_{Lys}$, Équation \ref{eq:KLys} et $C_{Lys}$, Équation \ref{eq:CLys}). Comme évoqué, la solution de Lysmer s'est basée sur une fondation chargée en surface. En réalité, dans le cas d'une fondation à une profondeur D, celle-ci va transmettre la chargement à un sol placé en profondeur et généralement plus rigide. Par conséquent, cet effet entraîne l'augmentation du coefficient $K_{max}$ \citep{Gazetas1985VerticalFoundations}. Il est ainsi important d'intégrer un facteur tenant compte de l'encastrement. \cite{Loukidis2008AssessmentProcedure} a analysé cet effet et a introduit le coefficient d'encastrement $D_f$. L'utilisation de ce coefficient permet d'appliquer la solution de Lysmer aux problèmes en profondeur.

\cite{Loukidis2008AssessmentProcedure} ont fait varié la profondeur d'une semelle au sein d'un matériau parfaitement élastique. Le coefficient $D_f$ est ainsi estimé à partir du rapport entre la profondeur et le diamètre B (Équation \ref{eq:Df}). Ils ont observé également que sous chargement dynamique ce paramètre était élevé à 1,7.


Pour les rapports D/B> 8 où $D_f$ devient constant,  $D_f$ varie entre 1,36, pour $\nu$=0,5 (argiles saturées), et 1,5, pour $\nu$=0,15 (sables).

Ce modèle a l'avantage d'être alimenté par des paramètres courants caractérisant le sol. Aucun de ces paramètres n'est particulier au problème du battage des pieux. Cela permet donc de réduire le niveau d'empirisme de l'analyse. Les paramètres du modèle sont déterminés à partir des essais en laboratoire ou estimé à partir des données in situ. Pour estimé la réaction en pointe d'un pieu, les paramètres sont : 

1) densité du sol ;

2) module de cisaillement à faibles déformations ($G_{max}$); 

3) coefficient de Poisson ($\nu$);

4) indice de plasticité (IP);

5) rayon du pieu ;

8) résistance unitaire limite en pointe ($q_c$);

9) paramètres visqueux du sol : m et n 

Certain des paramètres (en l'occurrence la densité et l'IP) sont des paramètres classiques aux rapports géotechniques. La résistance statique est déterminée à partir des essais CPT ou estimé à partir de la densité relative et de l'angle de frottement interne.

Le coefficient de Poisson utilisé est celui à faibles déformations. L'obtention de ce paramètre requiert l'utilisation des capteurs de haute précision. \emph{A priori}, ce paramètre n'est pas déterminé. Toutefois, vu que ceci est peu sensible à la densité et au confinement, on peut prendre de valeurs courantes sans avoir de grands impacts sur les résultats. Les valeurs de 0,1 à 0,25 sont courantes pour de sols ayant un degré de saturation $S_r$<70\%, avec des valeurs plus faibles pour les sables. L'intervalle de 0,15 à 0,22 étant une bonne approximation pour les sols sableux ou argileux. Cette valeur augmente considérablement avec le dégrée de saturation supérieur à 70\%. Pour des sols quasiment saturés ($S_r$>95\%) ou saturés, on admet un $\nu$ de 0,5.

Le module à faibles déformations est obtenu en laboratoire par l'essai de colonne résonnante et à partir de capteur piézoélectriques. \emph{In situ}, ceci est déterminé par des essais de cross-hole et down-hole. Ce paramètre est toutefois peu courant dans la plus part des projets géotechniques. C'est pourquoi, ceci est souvent estimé par le biais de corrélations.


\section{Modèle appliqué à l'essai CLT}

L'essai CLT a été introduit par \cite{GourvesR.Mirat1977MethodeSols} dans le but de compléter l'essai pénétrométrique classique. L'essai consiste à arrêter l'essai pénétrométrique à différentes profondeurs et à réaliser des chargements à très faible vitesse (de l'ordre de 0,01 mm/s). En mesurant la force et le déplacement lors de ces chargements, il permet d'établir la relation entre les contraintes et déplacements en pointe.

Différents travaux ont montré l'intérêt de la technique à afin de compléter l'essai classique \citep{Faugeras1979LessaiAnalogique, Arbaoui2003MesurePenetrometre, Ali2010CaracterisationProfondes, Reiffsteck2003CaracterisationSitu}. \cite{Zhou1997CaracterisationPanda} et \cite{Arbaoui2003MesurePenetrometre} proposent une méthode d'interprétation de l'essai afin d'étudier la déformabilité et le paramètres de cisaillement du sol. Bien que très intéressante, la technique n'est pas économique : celle-ci exige l'installation d'un système de réaction, le temps de chargement est assez important. 

Dans la méthode d'exploitation proposée par \cite{Zhou1997CaracterisationPanda}, la courbe de chargement CLT est exploitée à partir d'un modèle hyperbolique. Deux paramètres principaux sont déduits : le module pénétrométrique ($E_pn$) et la résistance ultime ($q_{ult,CLT}$). Le module est calculé à partir de la pente initiale et la résistance ultime correspond à interception entre les tangentes initiale et finale. Par la suite, on s'intéresse à la démarche permettant d'obtenir le module $E_{pn}$ à partir de l'essai CLT.

Le module de déformation est globalement défini comme la relation entre la contrainte ($\sigma$) et la déformation ($\varepsilon$). Pour l'essai triaxial, la déformation est obtenue directement à partir de la hauteur initiale de l'échantillon et des déplacements mesurés au cours de l'essai. Dans le cas d'un essai de pénétration, son obtention n'est pas directe étant donné que les déformations sont non-homogènes.

  \begin{equation} \label{eq:1}
    E =\frac{\sigma}{\varepsilon}
    \end{equation}

Soit 

  \begin{equation} \label{eq:2}
    E =\frac{\sigma}{\frac{\Delta h}{h_0}}
    \end{equation}

De manière analogue à l'essai triaxial, \cite{Faugeras1979LessaiAnalogique} propose l'obtention du module pénétrométrique ($E_{pn}$) exprimé par : 

  \begin{equation} \label{eq:E}
    E_{pn} =\frac{\sigma}{\frac{\Delta h}{h_{ee}}}
    \end{equation}

Soit

$\Delta \sigma$ : variation de contrainte

$\Delta h$ : enfoncement associé à cette variation de contrainte

$h_{ee}$ : hauteur élastique équivalente (définie par la suite)

Pour obtenir le module pénétrométrique, \cite{Faugeras1979LessaiAnalogique} introduit la notion de hauteur élastique équivalente ($h_{ee}$) afin de rendre $E_{pn}$ homogène à une contrainte. Cette hauteur correspond à la hauteur équivalente d'un échantillon cylindrique de 10$cm^2$ de section, soumis à la même pression verticale, à une pression latérale égale à la pression des terres au repos à la profondeur de l'essai et qui aurait une déformation de $\Delta$h. Il doit tenir compte de la forme et des dimensions de la plaque (représenté dans ce cas par le cône), de l'encastrement mais aussi du remaniement du sol car, l'essai CLT est en fait un chargement de deuxième cyclique, le premier cyclique ayant été conduit jusqu'à la « rupture » du matériau.

Pour estimer cette hauteur équivalente, on considère que le cône est assimilable à une plaque circulaire de rayon R dans un milieu élastique semi-infini de module E et coefficient de Poisson $\upsilon$. Par conséquent, l'application de la formule classique de Boussinesq permet de relier le tassement h et à la contrainte $\sigma$.
    
  \begin{equation}  \label{eq:h}
    \Delta h =\frac{\pi R}{2} (1-\upsilon^2) \frac{\sigma}{E}
    \end{equation}
    
Soit

  \begin{equation}  \label{eq:E}
    E =\frac{\pi R}{2} (1-\upsilon^2) \frac{\sigma}{\Delta h}
    \end{equation}

La relation entre le module et la pente initiale appliquée par \cite{Arbaoui2003MesurePenetrometre} est la même que celle proposée pour la raideur linéaire du ressort dans l'analogie de \cite{Lysmer1965VerticalFooting}. Cela est normal vu que la solution proposée par \cite{Lysmer1965VerticalFooting} est valable pour les sollicitations statiques et dynamiques. En effet lorsqu'on écrit cette relation en fonction de la charge (Q) et du module de cisaillement (G) (Équation \ref{eq:G}), on obtient la raideur linéaire K du ressort du modèle de \cite{Lysmer1965VerticalFooting}.

  \begin{equation}  \label{eq:G}
    G =\frac{E}{2(1-\nu)} 
    \end{equation}
    
      \begin{equation}  \label{eq:K}
    \sigma =\frac{Q}{A} = \frac{Q}{\pi R^2} 
    \end{equation}

Comme \cite{Loukidis2008AssessmentProcedure}, \cite{Arbaoui2003MesurePenetrometre} introduit un facteur d'encastrement (ici nommé $k_M$). Pour des grandes profondeurs (D>>B), la valeur retenue est celle proposée par \cite{Mindlin1936ForceSolid} pour un encastrement infini ($k_M$=2). En égalant les expressions \ref{eq:E} et \ref{eq:h}, on obtient la hauteur $h_{ee}$ (Équation \ref{eq:hee}).

  \begin{equation}  \label{eq:hee}
    h_{ee} =\frac{\pi R}{2} (1-\upsilon^2) \frac{1}{k_M}
    \end{equation}

Le module pénétrométrique est exprimé par l'équation \ref{eq:Epn}. Ceci est déterminé à partir de pente initiale de la courbe CLT.

  \begin{equation} \label{eq:Epn}
    E_{pn} = \frac{\pi R}{2} (1-\upsilon^2) \frac{\Delta \sigma}{\Delta h} \frac{1}{k_M}
    \end{equation}


  \begin{equation} \label{eq:Epn}
    \varepsilon = \frac{h}{h_{ee}} 
    \end{equation}

\cite{Faugeras1979LessaiAnalogique} a comparé le module pénétrométrique obtenu dans le sol intact (premier cycle) avec celui issu du deuxième cycle. Dans l'essai CLT, on peut considérer que l'installation de la pointe à la profondeur de l'essai correspond à un premier cycle de chargement sur un terrain vierge. L'essai CLT correspondrait ainsi à un 2ème cycle et aux cycles suivants. Le sol est déjà remanié avant un premier chargement l'essai CLT \citep{Zhou1997CaracterisationPanda}.

\cite{Arbaoui2003MesurePenetrometre} évoque l’utilité de la réalisation des cycles chargement-déchargement-rechargement au lieu d'un simple chargement monotone. On peut évoquer, à l’appui d’un tel essai, l’espoir d’effacer, en partie, les effets du remaniement \citep{Combarieu2001LPressiometre}. 

D’autres essais géotechniques (essai à la plaque, essai déchargement de fondation profonde, essai œdométrique, etc.) comportent dans leur procédure une phase de « déchargement-rechargement » avec exploitation des caractéristiques de déformation mesurées lors de cette boucle. 

Si l’on considère par exemple le cas du pressiomètre, le principe de la réalisation d’essais cycliques avec cet appareil n’est pas une nouveauté. En effet, Ménard et Rousseau (1962) ont introduit cette procédure (au cours de la dilatation de la sonde, un déchargement suivi d’un rechargement est effectué) et en baptisant le module de déformation pressiométrique mesuré sur le cycle (diagramme ($\sigma$, $\varepsilon$) de « module alterné Ea ». On constate que ceci s'approche du vrai module élastique (\emph{i.e.} à faibles déformations) et qu'il serait ainsi plus adapté pour étudier le comportement des sols sous machines vibrantes, par exemple \cite{Arbaoui2003MesurePenetrometre}.

\cite{Zhou1997CaracterisationPanda} et plus tard \cite{Arbaoui2003MesurePenetrometre} ont réalisé une série d'essais CLT en laboratoire. Ils ont constaté que le module tangent de déchargement est sensiblement constant. Le module de chargement est, quant à lui, pratiquement constant à partir du 3$^{ème}$ cycle. Le comportement global enregistré correspond à tout ce que l'on connaît des essais cycliques \cite{Zhou1997CaracterisationPanda}. Sur la Figure \ref{fig:ana_zhou_arbaoui_cycle}, on voit les résultats de \cite{Zhou1997CaracterisationPanda} et de \cite{Arbaoui2003MesurePenetrometre}. 

  \begin{figure}[H]
   \begin{center}
        \includegraphics[scale = .7]{media/ana_zhou_arbaoui_cycle.JPG}
        \caption{Résultats des essais CLT réalisés par Zhou (1997) pour le sable d'Allier (a et b) et par Arbaoui (2003) pour le (c) sable de Fontainebleau et pour (d) le sable d'Allier}
          \label{fig:ana_zhou_arbaoui_cycle}
      \end{center}
 \end{figure}
 

 \newpage
 
\section{Bilan et proposition pour l'essai DCLT}

On a pu voir l'effort des différents chercheurs afin d'améliorer le modèle de réaction initialement proposé par \cite{Smith1960Pile-drivingEquation}. Différents aspects du phénomène de la pénétration sont intégrés afin de mieux modéliser la résistance en pointe.

%Les modèles développés plus tard montrent que la raideur du ressort (KLys) et le coefficient d'amortissement (CLys) ne sont pas proportionnel à la résistance limite, mais fonction du module de cisaillement du sol, de la densité du sol et rayon du pieu \citep{Lysmer1965VerticalFooting}, Nguyen et al. (1988), de \cite{Holeyman1988ModellingBase}, Deeks et Randolph (1995)). Quasiment tout le modèle mis au point emploie l'analogie de Lysmer. La solution de \cite{Lysmer1965VerticalFooting} considère une semelle en surface d'un espace semi-infini. Comme le rapport entre la profondeur et le diamètre est assez élevé pour les fondations profondes, la raideur doit être corrigée à l'aide d'un coefficient tenant compte de la profondeur.

%L'effet visqueux dans la résistance est fonction de la vitesse de chargement. Cela est traduit soit par une relation linéaire soit par une relation exponentielle. Par conséquent, pour ces différents modèles, au moment où la vitesse en pointe est nulle, les effets visqueux devrait être négligeable.

%Or, \cite{Loukidis2008AssessmentProcedure} intègre le comportement non-linéaire du sol à partir de la dégradation de la raideur $K_{max}$ à partir d'un modèle hyperbolique.

La courbe contrainte-déplacement issue de l'essai DCLT permet d'obtenir la réaction en pointe suite à l'impact. Celle-ci traduit la réponse globale en pointe à la sollicitation dynamique. Par conséquent, elle est qualitativement comparable à une courbe de chargement en pointe jusqu'à la rupture.

Or, le pénétromètre est assimilable à un pieu isolé dont la réaction se doit à la réaction en pointe. Comme évoqué, les méthodes développées dans le domaine de pieux permettent de relier la réaction en pointe aux paramètres rhéologiques (E, $\phi$, c, ...) du sol ausculté.

Dans les différents modèles évoqués, on a vu que la réponse globale correspond à la somme de la réponse statique et dynamique. Par ailleurs, au contraire de ce que \cite{Smith1960Pile-drivingEquation} a proposé, les différents amortissements (amortissement radiatif, hystérésis et effet visqueux dans la résistance) sont expliqués par des mécanismes bien différents. L'amortissement radiatif est modélisé comme un amortisseur avec le coefficient qui est fonction des paramètres élastiques du sol ($G_{max}$, $v_p$) \citep{Lysmer1965VerticalFooting, Loukidis2008AssessmentProcedure}. L'effet visqueux dans la résistance est fonction de la vitesse de chargement. Cela est traduit soit par une relation linéaire \citep{Smith1960Pile-drivingEquation} soit par une relation exponentielle \citep{Coyle1970EmpiricalClays}.

Dans le cas de l'essai DCLT, les vitesses en pointe étant assez importantes, l'augmentation de la résistance attribuée aux effets visqueux sont non-négligeable. Toutefois, dans cet essai, on permet de connaître les grandeurs (force, vitesse, déplacement) en pointe tout au long du chargement, notamment lorsque la vitesse devient nulle. En ce moment précis, \emph{a priori}, la résistance mesurée se doit uniquement à la réaction statique en pointe. Pour les instants qui suivent, les valeurs de vitesse en pointe étant assez faibles, on suppose que la composante dynamique dans la raideur en phase de décharge est négligeable.

%Les différents travaux relient la raideur de courbe charge-déplacement aux paramètres de déformabilité du sol \citep{Lysmer1965VerticalFooting, Randolph2003ScienceDesign, Loukidis2008AssessmentProcedure, Arbaoui2003MesurePenetrometre}.

Selon la solution proposée par Lysmer, cette raideur est fonction de paramètres élastiques du sol. Pour \cite{Loukidis2008AssessmentProcedure}, cette raideur varie à partir de la valeur maximale (domaine supposé élastique) avec l'enfoncement et doit être corriger pour les applications en profondeur. Cette raideur maximale initiale correspondrait à la pente en décharge \citep{Holeyman1988ModellingBase, Loukidis2008AssessmentProcedure}.

Néanmoins, on sait que le sol présente un comportement élastique pour des déformations très petites (en général, $\varepsilon<10^{-4}$, \cite{Burland1989NinthStrains}). C'est pourquoi la plupart des techniques classiques ne permettent pas de remonter aux vrais paramètres élastiques ($G_{max}, E_{max}, v_s, v_p$). \emph{A priori}, l'obtention de ces paramètres requiert des méthodes bien spécifiques permettant d'entraîner des déformations suffisamment petites \citep{Burland1989NinthStrains, Atkinson2000Non-linearDesign}. 

\emph{In situ}, il est spécialement compliqué d'obtenir les paramètres à faibles déformations. Souvent même l'installation de l'équipement ou le procédure avant le chargement entraîne un remaniement non-négligeable empêchant d'accéder à des paramètres associés à de déformation inférieure à 0,01. Étant donné cela, il serait compliqué de remonter aux paramètres élastiques (\emph{i.e.} à petites déformations) à partir d'une courbe de chargement.

En effet, la pente en décharge pourrait permettre d'accéder à un module plus important que le module initial en chargement (notamment pour le premier chargement, \cite{Arbaoui2003MesurePenetrometre}), par conséquent, associé à une gamme de déformation moins importante (effet décrit comme d'effacement du remaniement \citep{Combarieu2001LPressiometre}. Toutefois, difficilement, aux modules à faibles déformations ($G_{max}$ ou $E_{max}$).

Comme évoque par \cite{Loukidis2008AssessmentProcedure}, même si à la rigueur, la solution proposée \cite{Lysmer1965VerticalFooting} suppose un comportement parfaitement élastique, dans la pratique la relation qui a proposé aussi appliquée pour de gamme de déformations plus importantes, notamment pour des problèmes dynamiques. Cette approche est celle appliquée dans le cadre d'autres travaux \citep{Zhou1997CaracterisationPanda, Arbaoui2003MesurePenetrometre, Reiffsteck2009MeasurementsPenetrometer} afin de remonter à un module pénétromètre.

En supposant que pour cette phase, la raideur correspond à la valeur maximale ${K_{max}}$ et que la contribution des composantes visqueuses sont négligeables (la vitesse étant faible pour cette phase de l'essai), on propose d'exploiter la phase en décharge de la courbe DCLT afin de remonter à deux modules. 

Pour ce faire, on applique l'analogie de Lysmer afin de définir deux modules de décharge. Les déformations associées à chacune de ces valeurs de module sont définies à partir la définition du module et la déduction de la hauteur élastique équivalente (Équation \ref{eq:E} et Équation \ref{eq:hee}). Suivant cette démarche, on définie deux modules à partir de deux raideurs déduites de la courbe DCLT. La première raideur correspond à la pente entre le point A (moment où la vitesse est nulle) et le point C (moment où la vitesse est minimale), comme présenté sur la Figure \ref{fig:ana_Eun}. La deuxième raideur correspond à la pente entre le point A et le point B (point moyen, en terme de déplacement, entre A et C).

  \begin{figure}[H]
   \begin{center}
        \includegraphics[scale = .6]{media/ana_Eun3.png}
        \caption{Interprétation de la courbe DCLT : détermination des modules $E_{un1}^{DCLT}$, à partir de la pente entre le point A et le point B, et du module $E_{un2}^{DCLT}$ à partir de la pente entre le point A et le point C}
          \label{fig:ana_Eun}
      \end{center}
 \end{figure}

Par la suite, on propose d'évaluer ses modules $E_{un1}^{DCLT}$ et $E_{un2}^{DCLT}$ à partir de résultats triaxiaux. Pour ce faire, on propose de comparer ces deux paramètres à deux valeurs de module issus de l'essai triaxial : $E_{0,1\%}$, $E_{50}$. Cela pour les sables d'Hostun et de Fontainebleau à trois états de densités et trois confinements.

Ces deux valeurs de module permettent d'avoir des mesures à deux niveaux de déformation et ainsi de mieux situer les mesures issues DCLT. On admet que le module à un niveau de déformation de 0,1\% correspond à un module initial triaxial car ceci est le niveau de déformation plus petit associé à l'essai triaxial classique (statique) \citep{Obrzud2018TheGuidebook, Benz2007Small-StrainConsequences, FlavignyE.DesruesJ.Paylayer1990NoteRF}. Le module sécant à 50\% du pic de résistance au cisaillement ($E_{50}$) permet d'avoir une deuxième mesure de référence associé à niveau de déformation plus important. Par ailleurs, ce module connaît de nos jours une utilisation de plus en plus répandue dans la modélisation du comportement des sols par éléments finis \citep{Schanz1999TheVerification, Cami2017ImagerieDeformations}.

Cela permettra de valider la pertinence de cette méthode pour les matériaux testés sur la base d'une méthode classique d'usage courant en géotechnique. 

Par la suite, on propose d'estimer la courbe de dégradation du module à partir de l'essai. Pour ce faire, on applique un modèle hyperbolique simple \citep{Fahey1993AModel} (Équation \ref{eq:Esec}).

\begin{equation} \label{eq:Esec}
    E = \frac{E_{max}}{1+f (\varepsilon) ^ g} 
\end{equation}

Le module $E{max}$ est obtenu, de manière similaire aux autres techniques in situ (\emph{e.g.} sDMT\footnote{Dilatomètre sismique}, sCPT\footnote{Sismo-cône}, \emph{cross-hole}) : à partir de la vitesse d'ondes $v_{p}$ et de la masse volumique du milieu $\rho$ (Équation \ref{eq:Emax_}). 

\begin{equation} \label{eq:Emax_}
    E_{max} = v_p^2 \rho 
\end{equation}

Souvent, on n'a pas d'information sur la masse volumique \emph{in situ}. Dans la pratique, celle-ci est souvent supposé à 1800 $kg/m^3$ \citep{Burns2007Small-Penetrometer} ou encore estimée à partir les valeurs indicatives selon le type de matériau (sable, limon ou argile) \citep{Plumelle2013TheorieOuvrages}. Pour appliquer cette relation au sCPT, par exemple, ce paramètre est estimé à partir de la résistance de pointe ($q_c$). Ceci car la résistance de pointe est fortement corrélée à la masse volumique \citep{Burns2007Small-Penetrometer, Robertson2010EstimatingCPT}. 

Dans le cas de l'essai pénétrométrique Panda, différents travaux ont étudié la relation entre les paramètre d'état (de densité et hydrique) et résistance dynamique $q_d$ \citep{Zhou1997CaracterisationPanda, Chaigneau2001CARACTERISATIONPENETROMETRE, Morvan2016OptimisationEstimation}. \cite{Chaigneau2001CARACTERISATIONPENETROMETRE} a démontré que pour un sol donné et pour un l’état hydrique connu, il existe une relation biunivoque entre $q_d$ et la densité sèche $\gamma_d$. Cette relation du type logarithmique a été établie en laboratoire différents type de sol.

Le module $E_{max}^{DCLT}$ estimé, ceci sert à alimenter le modèle (Équation \ref{eq:Esec}) comme la valeur limite du module. Ensuite, les modules $E_{un}^{DCLT}$ définis auparavant fournissent des mesures de déformabilité dans la gamme de grandes ou moyennes déformations. La Figure \ref{fig:ana_degradation_pressio}b illustre la démarche proposée.

Canépa et al. (2002) a appliqué une approche similaire au pressio-pénétromètre. Les auteurs ont estimé la dégradation du module de cisaillement en ajustant le modèle hyperbolique (Équation \ref{eq:Esec}) à l'aide des modules différents. Un module est obtenu à partir d'un cycle de faible amplitude et l'autre, nettement plus faible, est issu du pressiomètre. Sauf, que pour appliquer cette démarche, les auteurs devaient connaître $G_{max}$. Ceci est donc obtenue par essais de propagation d’ondes au sCPT \citep{Cami2017ImagerieDeformations}. La Figure \ref{fig:ana_degradation_pressio}b illustre la démarche proposée par l'auteur.

  \begin{figure}[H]
   \begin{center}
        \includegraphics[scale = .8]{media/ana_degradation_pressio.PNG}
        \caption{(a) Courbe de dégradation illustrant la méthode proposée et (b) Courbe normalisée obtenue par Canépa et al. (2002) avec le pressio-pénétromètre (coefficient d'ajustement : f=82.5, g=0.69)}
          \label{fig:ana_degradation_pressio}
      \end{center}
 \end{figure}

Il s'avère nécessaire d'évaluer la pertinence de la courbe estimée. Pour ce faire, on propose des confronter les courbes obtenues à partir de cette démarche aux résultats rapportés dans la littérature pour les sables d'Hostun et de Fontainebleau \citep{Duttine2005ComportementAxes}.

%\cite{Reiffsteck2009MeasurementsPenetrometer} a comparé les modules sécants CLT issus de cette démarche à de différents modules obtenus à partir d'autres méthodes. Pour des essais in situ réalisés, \cite{Reiffsteck2009MeasurementsPenetrometer} n'a pas pu accéder à de gammes de déformation inférieure à 1\%.


\section{Résistance ultime}

La résistance ultime est paramètre incontournable dans toutes les modèles d'interaction sol-pointe évoqués.

Un des atouts de l'essai DCLT c'est que ceci permet de suivre les signaux de force, vitesse et déplacement pendant le chargement.

Cela permet d'appliquer un nombre de méthode pour obtenir la résistance ultime.

On applique la méthode du point de décharge connue comme UPM (de l'anglais \emph{Unloading Method Point}) a l'essai DCLT. 

Selon cette méthode proposée par \cite{Middendorp200630Characteristics}, la résistance ultime correspond à la valeur de contrainte au moment où la vitesse est nulle.

%%%________BIBLIO__________
\newpage
\lhead{Bibliographie}
\nocite{}
\bibliographystyle{apalike}
\bibliography{library}
  
\end{document}
