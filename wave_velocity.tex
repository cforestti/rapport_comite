\documentclass{article}
\usepackage[utf8]{inputenc}
\usepackage[sectionbib]{natbib}
\usepackage[french]{babel}
\usepackage[titletoc]{appendix}
\usepackage{chapterbib}
\usepackage{url}
\usepackage{amsmath}
\usepackage{graphicx}
\usepackage{fancyhdr}
\usepackage{lmodern}
\usepackage{vmargin}
\usepackage[T1]{fontenc}
\usepackage{float}
\usepackage[table,xcdraw]{xcolor}
\usepackage{caption}
\usepackage{hyperref}
\usepackage{setspace}
%\usepackage{filecontents}
\usepackage{calc}
\usepackage{lipsum}
\usepackage{enumitem}
\graphicspath{ {images/} }
\usepackage{xcolor}
\usepackage{rotating}
\usepackage{gensymb}
\usepackage{siunitx}
\usepackage{textcomp}
\setcounter{secnumdepth}{3}

\pagestyle{fancy}
\fancyhf{}
%\lhead{\leftmark}
\rfoot{\thepage}

\begin{document}

\begin{titlepage}
    \begin{center}
    Thèse CIFRE N° 2018/1627
    
    Sol Solution
    
    Institut Pascal – Axe M3G
            \vspace*{5cm}
              %  \vspace*{5cm}
    
    %{Note technique}\\[0.3cm]
    
     {\textsc{\textbf{Note technique : comparatifs $c_p$ - $c_p^DCLT$}}}\\[1.cm]
     {Caroline F. Oliveira}\\[0.3cm]

     
    \end{center}
        \vspace*{8cm}
    
    
\textbf{Encadrement}
\begin{itemize}
	\item[] \textsc{Pierre BREUL}
	\item[] \textsc{Bastien CHEVALIER}
	\item[] \textsc{Miguel Angel BENZ NAVARRETE}
	\item[] \textsc{Quoc Anh TRAN}
\end{itemize}

        \vspace*{1cm}

\centering
\today

\end{titlepage}


\section{Introduction}

Comme on a vu, dans le domaine élastique, le module à faibles déformations peut être déterminé en connaissant la masse volumique du milieu.

La vitesse d’ondes est fonction de divers facteurs, à savoir : l’indice de vides (ou DR), l’état de contrainte, l’état hydrique et les caractéristiques intrinsèques du matériau (tailles, forme des grains).

Pour les sables, les facteurs majeurs étant l’indice de vides et l’état de contrainte pour les sables. Quant à la l’état hydrique, on a vu que pour les sables non saturés (S<95\%), celui-ci est un facteur secondaire puisque ceci engendre des variations de l’ordre de 10\% \citep{Emerson2005}.

Or, dans notre plan de validation, on se concentra sur certains matériaux rapportés afin de minimiser les effets intrinsèques liés aux matériaux qui sont considéré ici identiques (distribution granulométrique, forme des grains). 

En outre les particularités de chaque technique de mesure, il s’avère important de tenir compte du fait que, les essais fournissent des différents paramètres à faibles déformations (Emax, Gmax, vs, vp, comme synthétisé dans le Tableau 5). La vitesse d’ondes estimée par l'essai DCLT est, par hypothèse, de compression. Donc celle-ci ne serait comparable directement qu’à des techniques fournissant une mesure de $c_p$. Toutes les autres comparaisons reposent sur des hypothèses quant aux les valeurs de poisson e/ou des masses volumiques. Afin de prendre cela en compte, les comparatifs qui seront présenté son issue techniques déterminant $c_p$.

\section{Objectif}

Ce document a pour objectif de comparer les résultats de la vitesse d'ondes de compression issus d'essai DCLT ($c_p^{DCLT}$) aux résultats issus d'autres techniques. On souhaite vérifier si la mesure de vitesse d’ondes issue de l’essai DCLT correspond la vitesse d’ondes de compression $c_p$ et ainsi si celle-ci permettrait, comme dans le cas d'autres techniques, d’estimer le module à faibles déformations. 

Les comparatifs se focalisent uniquement dans les résultats obtenus en laboratoire pour les sables propres de référence (sable d'Hostun HN31 et de Fontainebleau NE34). Le choix de ces sables permettent de profiter des différents résultats rapportés dans la littérature pour ces matériaux.

\section{RAPPEL : détermination de la vitesse d'ondes à partir de l'essai DCLT}

Au cours de l’essai de pénétration dynamique, l’impact du marteau produit une onde incidente de vitesse vi et une contrainte i qui se propage par la tige. Au niveau de la pointe, l’interface entre la tige et sol, on assume que le sol et la tige soient séparés par une surface plane normale à la vitesse de propagation. 

Une partie de l’onde incidente de force FI dans le sol est réfléchie vers la tige avec une vitesse vR et une force FR, tandis qu’une autre partie est transmise au sol. Dans la tige, l’onde réfléchie s’ajoute à l’onde incidente (Figure 7a). Tant que les deux milieux adjacents restent collés, leurs particules situées des deux côtes de la surface de l’onde de choc, présentent la même vitesse et des contraintes normales de même intensité (principe d’action et réaction). Ceci constitue le principe général d’obtention des contraintes transmises au sol pour chaque impact.

\cite{Omidvar2012} expliquent que les essais des propagations d’ondes en géotechnique sont complexes à mettre en place car il est souvent difficile d’avoir le suivi des contraintes dans le sol et de s’affranchir de réflexions des ondes. Dans ce type d’essai l’instrumentation la plus courante est l’insertion de capteur à l’intérieur de l’échantillon ce qui conduit souvent à des erreurs.

Cette limitation a été résolue avec l’instrumentation de la tige et l’application des méthodes découplage et de reconstruction d’ondes dans la tige. Les capteurs situés dans au niveau de la tête enregistrent les contraintes engendrées par le passage de l’ébranlement et son retour de manière quasi-continue (50 kHz) pour un intervalle $\Delta t$ égal à 2L/c. Avec L la longueur de la barre et c la vitesse d’ondes dans l’acier (de l’ordre de 5200m/s). Le découplage et reconstruction d’ondes permet par la suite d'estimer les contraintes et la vitesse à l'interface sol-pointe \citep{Benz-Navarrete2009}. 

La vitesse d’ondes ($c_p^{DCLT}$) est déduite à partir de la vitesse particulaire en pointe ($v_p$), de la contrainte en pointe ($\sigma_p$) et de la masse volumique $\rho$ du milieu (Équation 34).

\begin{equation} \label{eq:celerite}
    v = E_{ref} \frac{\sigma_3}{P_{ref}}^n
\end{equation}

\cite{Benz-Navarrete2009} a appliqué cette démarche, initialement proposée par \cite{Aussedat1970}, à l'essai DCLT. Il a testé différents matériaux (béton, bois et différents sols) en laboratoire. Les polaires de choc obtenues sont répétables et sensibles aux matériaux. Les valeurs de $c_p^{DCLT}$ obtenus sont de l'ordre de grandeur conforme aux matériaux testés.

\begin{figure}[H]
    \begin{center}
        \includegraphics[scale = .7]{media/polaire.PNG}
        \caption{(a) Polaires de choc expérimentales pour différents sols \cite{Benz-Navarrete2009} et (b) Valeurs de $v_p^{DCLT}$ obtenues et valeurs typiques pour les matériaux testés}
        \label{fig:polaire}
  \end{center}
\end{figure}

\cite{Escobar2015} a également réalisé des comparatifs de ($v_p^{DCLT}$) et des estimations de $v_p$ à la partir de la méthode de surface MASW. Ces essais ont été importants pour présenter la technique, toutefois les comparatifs n'ont pas permis de valider la mesure. 

\section{Comparatifs}


\subsection{Données rapportées dans littérature pour les sables d'Hostun et Fontainebleau}

En laboratoire, la vitesse d'ondes de compression est déterminée l'aide de capteurs piézoélectriques souvent appelés extenders éléments. Les résultats présentés sont des déterminations obtenus en laboratoire pour des échantillons reconstitués de ces sables réalisé par pluviation à sec.

\subsection{Données DCLT}

Les essais DCLT ont été réalisé en chambre d'étalonnage simulant les conditions au repos. Les éprouvettes ont été réalisé par pluviation à sec. La chambre cylindrique permet de confectionner des éprouvettes de 730 mm de hauteur et de 547 de diamètre. Un système de pression hydraulique est installé en dessous du couvercle supérieur. Ce couvercle est doté d'une membrane en caoutchouc qui se dilate appliquant ainsi une pression verticale à l'échantillon. La pression appliquée est pilotée à l'aide d'une cellule de charge. Pour plus de détail sur sur le système s'adresser à \cite{LeKouby2008} et \cite{Muhammed2016}.

\subsection{Comparatif des mesures}

Parmi des résultats de $vp$ rapportées dans la littérature, on compare les résultats correspondant aux conditions des essais DCLT réalisés.

\begin{figure}[H]
    \begin{center}
        \includegraphics[scale = .7]{media/comparatifs_.PNG}
        \caption{(a)Comparatifs entre $c_p$ et $c_p^{DCLT}$ pour le SF et SH
        \label{fig:comparatifs_}
  \end{center}
\end{figure}

La Figure on présente ces données en format graphique, on obtient  

\begin{figure}[H]
    \begin{center}
        \includegraphics[scale = .7]{media/comparatifs_graphique_.PNG}
        \caption{(a)Comparatifs entre $c_p$ et $c_p^{DCLT}$ pour le SF et SH
        \label{fig:comparatifs_}
  \end{center}
\end{figure}




\section{Analyse globale des données DCLT}
 
 

\section{Conclusion}

  
 
 
 
%%%________BIBLIO__________
\newpage
\lhead{Bibliographie}
\nocite{}
\bibliographystyle{apalike}
\bibliography{library}


  
\end{document}


a citer : \cite{Sauzeat2003}

explication BE : \cite{SHARIFIPOUR}