\documentclass[12pt]{report}
\usepackage[utf8]{inputenc}
\usepackage[sectionbib]{natbib}
\usepackage[french]{babel}
\usepackage[titletoc]{appendix}
\usepackage{chapterbib}
\usepackage{url}
\usepackage{amsmath}
\usepackage{graphicx}
\usepackage{fancyhdr}
\usepackage{lmodern}
\usepackage{vmargin}
\usepackage[T1]{fontenc}
\usepackage{float}
\usepackage[table,xcdraw]{xcolor}
\usepackage{caption}
\usepackage{hyperref}
\usepackage{setspace}
%\usepackage{filecontents}
\usepackage{calc}
\usepackage{lipsum}
\usepackage{enumitem}
\graphicspath{ {images/} }
\usepackage{xcolor}
\usepackage{rotating}
\usepackage{gensymb}
\usepackage{siunitx}
\usepackage{textcomp}
\setcounter{secnumdepth}{3}

\pagestyle{fancy}
\fancyhf{}
\lhead{\leftmark}
\rfoot{\thepage}

\begin{document}

\begin{titlepage}
    \begin{center}
    Thèse CIFRE N° 2018/1627
    
    Sol Solution
    
    Institut Pascal – Axe M3G
            \vspace*{5cm}
              %  \vspace*{5cm}
    
    %{Note technique}\\[0.3cm]
    
     {\textsc{\textbf{Note technique}}}\\[1.cm]
     {Caroline F. Oliveira}\\[0.3cm]
     
     \today
     
    \end{center}
        \vspace*{8cm}
    
    
    
\begin{table} [H]
\begin{tabular}{ll}
\centering
   %     \element{\textbf{Pierre BREUL}} & \element{Directeur de thèse}\\
    %    \element{\textbf{Bastien CHEVALIER}} & \element{Encadrant}\\
      %  \element{\textbf{Claude BACCONNET}} & \element{Encadrant}\\
      %  \element{\textbf{Miguel Angel BENZ NAVARRETE}} & \element{Encadrant en entreprise}\\
      %  \element{\textbf{Quoc Anh TRAN}} & %\element{Encadrant en entreprise}\\
\end{tabular}
\end{table}
\centering
\vspace*{\fill}
\end{titlepage}


\chapter{Introduction}

\section{Objectif}

Ce document a pour objectif de comparer les résultats de la vitesse d'ondes de compression issus d'essai DCLT ($v_p^{DCLT}$) aux résultats issus d'autres techniques.

Les comparatifs se focalisent uniquement dans les résultats obtenus en laboratoire pour les sables propres de référence (sable d'Hostun HN31 et de Fontainebleau NE34). Le choix de ces sables permettent de profiter des résultats rapportés dans la littérature. 

On souhaite vérifier si la mesure de vitesse d’ondes issue de l’essai DCLT correspond la vitesse d’ondes de compression $v_p$ et ainsi que cette mesure permettrait d’estimer le module à faibles déformations. Comme on a vu, dans le domaine élastique, le module à faibles déformations peut être déterminé en connaissant la masse volumique du milieu.

La vitesse d’ondes est fonction de divers facteurs, à savoir : l’indice de vides (ou DR), l’état de contrainte, l’état hydrique et les caractéristiques intrinsèques du matériau (tailles, forme des grains).

Pour les sables, les facteurs majeurs étant l’indice de vides et l’état de contrainte pour les sables. Quant à la l’état hydrique, on a vu que pour les sables non saturés (S<95\%), celui-ci est un facteur secondaire puisque ceci engendre des variations de l’ordre de 10\% \cite{Emerson2005}

Or, dans notre plan de validation, on se concentra sur certains matériaux/sites rapportés afin de minimiser les effets intrinsèques liés aux matériaux qui sont considéré ici identiques (distribution granulométrique, forme des grains). 

En outre les particularités de chaque technique de mesure, il s’avère important de tenir compte du fait que, les essais fournissent des différents paramètres à faibles déformations (Emax, Gmax, cs, cp, comme synthétisé dans le Tableau 5). La vitesse d’ondes issue du Panda 3 est, par hypothèse, une vitesse d’ondes de compression. Donc celle-ci ne serait comparable directement qu’à des techniques fournissant une mesure de $c_p$. Toutes les autres comparaisons reposent sur des hypothèses quant aux les valeurs de poisson e/ou des masses volumiques. Afin de prendre cela en compte, on envisage poursuivre comme montré sur le Tableau 4 selon chaque cas.


  
\section{Données rapportées dans littérature}

On s'intéresse par des résultats de vitesse d'ondes de compression ($v_p$) obtenus en laboratoire pour les sables d'Hostun et de Fontainebleau.

En laboratoire, la vitesse d'ondes de compression est obtenu à partir de capteurs piézoélectriques souvent appelés extenders elements.

Les résultats présentés sont des déterminations obtenus en laboratoire pour des échantillons reconstitués de ces sables réalisé par pluviation à sec.



\section{Données DCLT}

\section{Comparaison ponctuel}



  
\section{Analyse globale des données}
  

  
\end{document}


a citer : \cite{Sauzeat2003}

explication BE : \cite{SHARIFIPOUR}