\documentclass{article}
\usepackage[utf8]{inputenc}
\usepackage[sectionbib]{natbib}
\usepackage[french]{babel}
\usepackage[titletoc]{appendix}
\usepackage{chapterbib}
\usepackage{url}
\usepackage{amsmath}
\usepackage{graphicx}
\usepackage{fancyhdr}
\usepackage{lmodern}
\usepackage{vmargin}
\usepackage[T1]{fontenc}
\usepackage{float}
\usepackage[table,xcdraw]{xcolor}
\usepackage{caption}
\usepackage{hyperref}
\usepackage{setspace}
%\usepackage{filecontents}
\usepackage{calc}
\usepackage{lipsum}
\usepackage{enumitem}
\graphicspath{ {images/} }
\usepackage{xcolor}
\usepackage{rotating}
\usepackage{gensymb}
\usepackage{siunitx}
\usepackage{textcomp}
\setcounter{secnumdepth}{3}
\usepackage{amsmath}

\pagestyle{fancy}
\fancyhf{}
%\lhead{\leftmark}
\rfoot{\thepage}

\begin{document}

\begin{titlepage}
    \begin{center}
    Thèse CIFRE N° 2018/1627
    
    Sol Solution
    
    Institut Pascal – Axe M3G
            \vspace*{5cm}
              %  \vspace*{5cm}
    
    %{Note technique}\\[0.3cm]
    
     {\textbf{Note technique : comparatifs $c_p$ - $c_p^{DCLT}$}}\\[1.cm]
     {Caroline F. Oliveira}\\[0.3cm]

     
    \end{center}
        \vspace*{8cm}
    
    
\textbf{Encadrement}
\begin{itemize}
	\item[] \textsc{Pierre BREUL}
	\item[] \textsc{Bastien CHEVALIER}
	\item[] \textsc{Miguel Angel BENZ NAVARRETE}
	\item[] \textsc{Quoc Anh TRAN}
\end{itemize}

        \vspace*{1cm}

\centering
\today

\end{titlepage}

\section{Objectif}

Ce document a pour objectif de comparer les résultats de la vitesse d'ondes de compression issus de l'essai DCLT ($c_p^{DCLT}$)\footnote{On adopte la notation $c_p$ pour la vitesse d'ondes, la notation $v_p$ faisant référence à la vitesse particulaire} aux résultats issus d'autres techniques. On souhaite vérifier dans quelle mesure la vitesse d’ondes issue de l’essai DCLT correspond la vitesse d’ondes de compression $c_p$ et ainsi celle-ci permettrait, comme dans le cas d'autres techniques, d’estimer le module à faibles déformations.

Pour ce faire, on compare les résultats de $c_p^{DCLT}$ et $c_p$ sous des conditions similaires. Ces comparatifs se focalisent uniquement sur les résultats obtenus en laboratoire pour les sables propres de référence : sable d'Hostun HN31 (SH) et de Fontainebleau NE34 (SF). Le choix de ces sables permet de profiter des différents résultats rapportés dans la littérature pour ces matériaux.

\section{Introduction}

Le module de déformation d'un sol est un des paramètres les plus difficiles à déterminer car il dépend de nombreux facteurs (granulométrie, densité, confinement, histoire de chargement, niveau de déformation …) \citep{Briaud2001}.

Par ailleurs, à partir des années 1970, le développement de techniques expérimentales de plus en plus précises ont mis en évidence la limitation des essais classiques pour traduire le comportement des sols à petites déformations \citep{Burland1989,Atkinson2000, GomesCorreia2004}.

Ces travaux ont montré que le domaine élastique pour les sols est assez restreint et difficile à déterminer (souvent associé à des déformations de l'ordre de $10^{-5}$). Les essais de propagations d’ondes (capteurs piezo-électriques en laboratoire et essais de forage in situ) induisent des déformations très faibles, inférieures à $10^{-5}$. L’hypothèse d’un comportement élastique linéaire pour ces essais est donc justifiée \citep{ Benz2007}. Il est donc possible d’obtenir le module à faibles déformations ($E_{max}$ ou $E_0$) à partir des mesures de vitesse d’ondes de compression ($c_p$) et de la masse volumique du milieu $\rho$ (équation \ref{eq:Emax}).

\begin{equation} \label{eq:Emax}
   E_{max} = \rho c_p^2
\end{equation}

Soit $c_p$ l'onde de volume de compression se propagent au sein du matériau. Les ondes $c_p$ ou ondes-P sont comparables aux ondes sonores. Celles-ci se propagent dans le sens de vibration. Elles se propagent à travers des fluides et des solides et sont a priori plus rapides que les ondes de volume de cisaillement ($c_s$ ou onde-S). Les ondes-S se propagent perpendiculairement à la direction de vibration. Elles se propagent à la condition que le milieu soit capable de reprendre des efforts de cisaillement. C’est pourquoi celles-ci ne se propagent pas à travers des fluides.

Ces essais de propagations d’ondes sont, toutefois sont peu courants dans la pratique. Au delà des limitations pratiques : manque de personnel et d'équipement pour la réalisation et l'interprétation de ces essais, on peut citer des limitations techniques. Dans le cas des essais en laboratoire (capteurs piezo-électriques) réalisés sur des petits échantillons, les limitations sont liées à la représentativité et aux conditions des échantillons (qualité des échantillons et contrainte initiale en place). Dans le cas des essais de forage, ils sont délicats à réaliser et aussi coûteux.

La limitation des essais de vitesse d'ondes en laboratoire comme  Dans la pratique les valeurs appliquées sont souvent obtenues à partir des estimations parfois peu réalistes. Par ailleurs, le manque des données à faibles déformations constitue une limitation à l'application des modèles de comportement plus poussés intégrant le dégradation du module et permettant d'avoir des estimations plus réaliste des déformations entrainées par les ouvrages.

L'essai de chargement dynamique en pointe pénétrométrique (DCLT) au pénétromètre dynamique léger Panda 3$\up{\textregistered}$  est un technique simple et à faible coût. Lors des premiers instants suivant l'impact, cet essai est assimilable à un essai de choc \citep{Aussedat1970, Semblat2012, IskanderMaguedOmidivar2015}. De ce fait, comme pour les essais de choc, l’exploitation des signaux de force et de vitesse dans le sol juste après la sollicitation permet d’étudier le comportement à faibles déformations et d'estimer la vitesse d’ondes de compression ($c_p^{DCLT}$).

Dans ce cas, il resterait à évaluer la précision des vitesse ainsi déterminées. Dans ce document, on compare les résultats de $c_p^{DCLT}$ et $c_p$ sous des conditions similaires. La vitesse d'ondes de compression issue de à l'essai DCLT étant par hypothèse de compression, les valeurs de $c_p$ présentées sont issues techniques déterminant $c_p$ (aucune estimation à partir de $c_s$).


\section{Détermination de la vitesse d'ondes à partir de l'essai DCLT : analogie avec les essais de choc}

Suite à une brève introduction sur les essais de choc, on va expliquer la méthode d'obtention de $c_p^{DCLT}$ à l'aide de l'essai DCLT en deux étapes. Dans un premier temps, on explique la méthode d'obtention des contraintes et de vitesses en pointe. Dans un deuxième temps, on explique la démarche et les hypothèses permettant de déterminer la vitesse d'ondes $c_p^{DCLT}$ à partir de mesures de historique de contrainte et de vitesse en pointe et de la masse volumique du milieu (polaire de choc).

\subsection{Introduction : les essais de choc}

L'application d'un impact dans le sol crée un état de contrainte complexe. Selon la location par rapport au point d'application de l'impact, le sol adjacent peut être affecté par des ondes de compression, de cisaillement ou des ondes Rayleigh ou encore par une combinaison de celles-ci. Cet état de contrainte est encore plus complexe en surface.

L'onde de contrainte crée par l'impact est caractérisé par une montée de rapide, des amplitudes importantes et des petites durées. L'étude de la réponse du sol subit à des impacts est normalement mené en laboratoire en se basant sur les concepts suivants : (a) reproduction d'un état de contrainte uniforme en utilisant un échantillon assez petit, (b) propager une onde de contrainte au long d'échantillon de sol assez long.

Les techniques courantes appliquant ces concepts sont : (1) essai de compression uniaxial (aussi appelée compression confiné ou teste oedométrique, (2) les essais de barres de Hopkinson (connu par l'acronyme anglais Split Hopkinson Pressure Bar - SHPB) et les essais de choc plan.

Certains travaux rapportent la détermination de vitesse d'ondes des sols. Charlie et al. (1990) ont déterminé la vitesse d'ondes de compression pour une sable silicieux  à des différentes teneur en eau à l'aide des essais SHPB. Il a utilisé des échantillons suffisamment long afin d'assurer une seule passage de l'onde de contrainte. Ils ont obtenus des valeurs variant entre 212-454 m/s. Brown et al. (2007) ont déterminé la vitesse d'ondes de compression dans des sols siliceux de 243 m/s à l'aide des essais de flyer plate experiments. Martin et al. (2009) ont utilisé l'essai de SHPB pour déterminer la vitesse d'ondes de sables.

Pour des matériaux caractérisés par des vitesse d'ondes importantes, il est plus facile de garantir une homogénéité des contraintes au sein de l'éprouvette. Afin d'assurer un distribution plus homogène dans les premiers instants du chargement, l'évolution des contraintes appliquées doit être controllé de façon à entrainer une augmentation des contraintes dans l'échantillon quasi-uniforme. Cela est souvent réalisé par le biais des techniques de ajustement du pulse (pulse shaping techniques).

\begin{figure}[H]
    \begin{center}
        \includegraphics[scale = .7]{media/ana_cp2.PNG}
        \caption{Effets de l'application de la méthode d'ajustement de pulse (pulse shaping techniques) dans les historiques de contraintes des essais SHPB}
        \label{fig:ana_cp2}
  \end{center}
\end{figure}

\cite{Omidvar2012} expliquent que les essais des propagations d’ondes en géotechnique (comme les essais de choc) sont complexes à mettre en place car comme déjà évoqué les hypothèse de base sont difficiles assurées et, en plus, d'un point de vue technique il souvent difficile d'avoir le suivi des contraintes et des vitesses dans le sol suite l'impact. Dans ce type d’essai, l’instrumentation la plus courante est l’insertion de capteur à l’intérieur de l’échantillon ce qui conduit souvent à des erreurs.

Des problèmes liés à l'obtention des historiques de contraintes et des vitesses a été résolue par l’instrumentation du pénétromètre et l’application des méthodes de découplage et de reconstruction d’ondes dans la tige à l'essai DCLT. Des capteurs situés au niveau de la tête enregistrent les forces engendrées par le passage de l’ébranlement et son retour de manière quasi-continue (100 kHz) pour un intervalle $\Delta t$ égal à 2L/c. Avec L la longueur de la barre et c la vitesse d’ondes dans l’acier (de l’ordre de 5200m/s).

\subsection{Obtention des contraintes et des vitesses en pointe : essai DCLT}

Au cours de l’essai DCLT, chaque impact de marteau crée une onde incidente. Celle-ci se propage le long du train de tige. En pointe, une partie de cette onde incidente est réfléchie vers la tête, tandis qu’une autre partie de l'onde incidente est transmise au sol. Dans la tige, l’onde réfléchie s'ajoute à l’onde incidente. Pour comprendre l'obtention des historiques de contraintes et de vitesse en pointe, il faut s'intéresser aux phénomènes dynamiques ayant lieu après l'impact.

La déformation entraînée par le passage de cette onde de compression incidente $\varepsilon$ peut être exprimée par l'équation \ref{eq:wave_def}.

\begin{equation} \label{eq:wave_def}
    \varepsilon = \frac{\partial u}{\partial x}
\end{equation}

%En appliquant la relation deuxième loi de Newton (F=ma), avec la masse m comme m=Atrhoρt dx et l'accélération $a=\partial^2/\partial t^2$ pour une portion %de la tige, on obtient à l'équation \ref{eq:stress}.

\begin{equation} \label{eq:wave_stress}
    \frac{ \partial \sigma}{\partial^2}= \rho_t^2 \frac{\partial u}{\partial x}
\end{equation}

Par ailleurs, en supposant la tige élastique,  $\sigma=E_t \varepsilon$, avec $\varepsilon$ donné par l'équation \ref{eq:wave_def}, on obtient l'équation différentielle qui gouverne la propagation de l'onde u(x,t), connue simplement équation de l'onde. 

\begin{equation} \label{eq:wave}
    \frac{ \partial \sigma}{\partial t^2}= c_t^2 \frac{\partial u}{\partial x}
\end{equation}

Il est possible l'équation de l'onde analytiquement par différentes méthodes (transformée de Laplace, séparation des variables, méthode des caractéristiques) ou numériquement. Chacun de ces méthodes est discuté par \cite{Verruijt2009}. Dans le cadre de ce travail, on retient la solution obtenue par la méthode des caractéristiques, proposé initialement par \cite{Saint-Venant1898}. Selon celle-ci, l'équation \ref{eq:wave} correspond à la superposition deux ondes $u_{d}(t-x/c_t)$ (descendante) et $u_{u}(t-x/c_t)$ (remontante) se propageant le long des tiges de manière continue à une vitesse d'ondes ($c_t$) (Équation \ref{eq:wave_deplacement}).

\begin{equation} \label{eq:wave_deplacement}
u(x,t) = u_d(t-x/c_t)-u_u(t+x/c_t)
\end{equation}

L’impact engendre une onde de compression u(x,t) dans le pénétromètre qui se propage à vitesse constante dans le sens longitudinal. La vitesse de propagation d’une onde de compression $c$ dans un milieu solide infini est déterminée par l'équation (\ref{eq:wave_c}). La vitesse de propagation $c$ étant une fonction des propriétés du matériau du milieu traversé (module d'Young E, coefficient de Poisson $\nu$ et densité $\rho$).

\begin{equation} \label{eq:wave_c}
    c = \sqrt{\frac{1}{\rho}\frac{E(1-\nu)}{(1+\nu)(1-2\nu)}}
\end{equation}

Pour le cas des tiges élancées, la vitesse de propagation de l’onde de compression u(x,t) se déplaçant longitudinalement est indépendante du coefficient de Poisson $\nu$ \citep{Timoshenko1951} et est donc donnée par l’équation \ref{eq:wave_ct}.

\begin{equation} \label{eq:wave_ct}
    c_t = \sqrt{\frac{E_t}{\rho_t}}
\end{equation}

Soit $c_t$, $E_t$, et $\rho_t$ la vitesse de propagation de l'onde, le module d'Young et la densité des tiges respectivement. 

Le passage de l'onde u(x,t) à un instant t engendrera des variations de déformation et de vitesse particulaire v(x,t) à un point donnée de la tige x.  La vitesse particulaire v(x,t) est obtenue en dérivation de l'équation \ref{eq:wave_deplacement} en fonction du temps.

\begin{equation} \label{eq:wave_vitesse}
v(x,t) = v_d(t-x/c_t)-v_u(t+x/c_t)
\end{equation}

La déformation entraînée par le passage de l'onde u(x,t), celle-ci obtenue à partir de la dérivation de l'équation \ref{eq:wave_def} en fonction de x. Cette déformation $\varepsilon$ peut être exprimée en terme de la vitesse particulaire des ondes, incidente et remontante, afin d'obtenir l'équation \ref{eq:wave_def2}.

\begin{equation} \label{eq:wave_def2}
    \varepsilon(x,t)=-\frac{1}{c_t}[v_d(t-x/c_t]-v_u(t+x/c_t)]
\end{equation}

En tenant compte de l'élasticité des tiges, c'est-à-dire de $F=\varepsilon A_t E_t$, on obtient l'équation \ref{eq:wave_FZv}

\begin{equation} \label{eq:wave_FZv}
    F(x,t)=-\frac{A_t E_t}{c_t}[v_d(t-x/c)-v_u(t-x/c)]
\end{equation}

Comme constaté sur l'équation \ref{eq:wave_FZv}, dans le problème de la propagation longitudinale d'une onde mécanique dans un milieu élastique, la force F(x,t) et la vitesse particulaire v(t,x) sont proportionnels. Le rapport de proportionnalité entre ceux deux grandeurs étant ${A_t E_t}/{c_t}$, c'est-à-dire, l'impédance mécanique de la tige $Z_t$.

Ainsi, à partir des enregistrements de déformation $\varepsilon_A(t)$ et de vitesse $v_A(t)$ réalisés sur les tiges dans une section A au voisinage de l’enclume, les ondes $\varepsilon_d$ et $\varepsilon_u$ sont découplées selon les Équation \ref{eq:decouplage_down} et \ref{eq:decouplage_up} respectivement. 

\begin{equation} \label{eq:decouplage_down}
    \varepsilon_d (t) = \frac{1}{2} \left[ \varepsilon_A (t) - \frac{v_A(t)}{c_t} \right]
\end{equation}

\begin{equation} \label{eq:decouplage_up}
    \varepsilon_u (t) = \frac{1}{2} \left[ \varepsilon_A (t) + \frac{v_A(t)}{c_t} \right]
\end{equation}

En connaissant les signaux $\varepsilon_A(t)$ et $v_A(t)$, il est possible de reconstruire les signaux de déformation $\varepsilon(t)$, de force $F(t)$ et de vitesse particulaire $v(t)$ pour une section donnée C situé en dessous du point de mesure. Cela en supposant que les efforts extérieurs (frottement latéral) nuls entre le point de mesure (A) et la section analysée (C). Avec une tige homogène et de section constante $A_t$ (impédance uniforme $Z_t$), la reconstruction des signaux de force $F(t)$ et de vitesse $v(t)$ est donné par les Équations \ref{eq:reconstruction_F} et \ref{eq:reconstruction_v} respectivement. Soit $\Delta x = x_C - x_A$.

\begin{equation} \label{eq:reconstruction_F}
    F (t) = \frac{E_t A_t}{2} \left[ \varepsilon_A (t + 2 \Delta x / c_t) + \varepsilon_A (t) \right] + \left[ \frac{Z_t}{2}  v_A (t) + v_A (t + 2 \Delta x/c_t)  \right]
\end{equation}


\begin{equation} \label{eq:reconstruction_v}
    v (t) = \frac{1}{2} \left[ v_A \left( t + \frac{2 \Delta x}{c_t} \right) + v_A (t)  \right] + \left( \frac{E_t A_t}{2 Z_t} \right) \left[ \varepsilon_A \left( t + \frac{2 \Delta x}{c_t} \right) + \varepsilon_A (t) \right]
\end{equation}

Ainsi, on obtient les suivis de contrainte et de vitesse particulaire en pointe pour chaque impact. On s'intéresse par la suite à méthode et aux hypothèses permettant de déterminer la vitesse d'ondes à partir de mesures de contrainte et de vitesse particulaire. Ces relations et ces hypothèses sont appliqués aux essais de choc évoqués dans l'introduction de cette section (à savoir essais de choc unixial, essais SHPB). Ces relations et ces hypothèses ont été plus tard appliqués aux pénétromètres dynamiques par \cite{Aussedat1970} et \cite{Meunier1974}.

\subsection{Détermination de la vitesse d'ondes à partir de mesures de contrainte et de vitesse particulaire en pointe}

%High wave speeds in the specimen facilitate an evenly distributed stress field in the specimen and is required to acquire stress equilibrium

%Characterization of sand using the Kolsky bar is more complex than conventional materials such as metals. The longitudinal wave speed in a granular system, such as sand, is governed by complex interactions between sand particles, pore air, and pore water.

%The tests covered include uniaxial and triaxial compression, split Hopkinson pressure bar (SHPB), and plate impact tests. Results ofthese tests are summarized, in order to provide a deeper understanding ofthe response ofgranular media to HSR loading


%Sand is subjected to high-rate compressive loading in many engineering applications including explosions, air blasts, mine blasting, drilling, vehicle and aircraft wheel loading, dynamic compaction, pile driving, and projectile penetration. Sand

%Dans les années 1970, \cite{Aussedat1970} et \cite{Meunier1974} ont introduit l’application des essais de choc à l’essai de pénétration. Ils ont également réalisé des tests SHPB. 

La détermination de la vitesse d'ondes de compression $c_p$ à partir des mesures de contrainte et de vitesse particulaire s'appuie sur des principes des chocs et sur les équations d’état (aussi connue comme équations de Rankine-Hugoniot), qui régissent leur propagation. A l’aide des équations de la mécanique classique et des lois de comportement (équations rhéologiques), ces équations dites « mixtes » décrivent la relation entre les contraintes et les vitesses (nommé polaire de choc) lors de la propagation d’un ébranlement. Ces relations permettent d’étudier le comportement du sol soumis à un impact et de déterminer l’impédance mécanique de celui-ci (Z) et, la masse volumique connue, déduire la vitesse de propagation d’ondes de compression ($c_p$).

%Les variables qui caractérisent le matériau sont la masse volumique, le tenseur des contraintes (6 composantes inconnues) et le tenseur des déformations (6 composantes inconnues). Pour simplement par les éléments de leur diagonale. Or, par symétrie ?:: = ?;;. Les particules sont l’onde longitudinale plane, l’axe 0x est principal et les deux tenseurs sont décrits tout sollicitées verticalement surtout sous le centre de tige et développent une vitesse verticale. Les déformations générées sont essentiellement normales au plan de propagation (εxx). Cette hypothèse s’appuie sur la vitesse de l’impulsion. Ce phénomène étant trop rapide, le bulbe de uniquement trois variables ?<<, ?:: et ?<< principales décrivant le problème.

%On analyse un tube cylindrique dans un milieu infini de masse et de section S constantes lors du passage de la surface d’onde de choc. A un instant t donné, lorsque l’onde vient juste de passer le plan A, l’élément AB présente une masse volumique ρ0. Lors du passage de l’onde, toute la matière initialement comprise entre A et B sera comprise entre A’ et B. En appliquant la conservation de la masse à cet élément et puis en exprimant les longueurs de ces régions en fonction de la vitesse des particules v et de la vitesse d’onde D, on obtient l’Équation 21


%%________________________

Le passage de l’onde de choc entraine un changement de volume dans le milieu et par conséquent un changement de sa masse volumique. Pour l’ébranlement longitudinal, on assume que la variation de volume $\Delta V$ est essentiellement axiale et exprimée en fonction des masses volumique initiales $\rho_0$ et de la masse volumique finale  $\rho$ (Équation 19).

Les variables qui caractérisent le matériau sont la masse volumique, le tenseur des contraintes (6 composantes inconnues) et le tenseur des déformations (6 composantes inconnues). Dans le cas des essais de choc, on suppose onde longitudinale plane, l'axe 0x est principal et les deux tenseurs sont décrits tout simplement par les éléments de leur diagonale. Par ailleurs, par symétrie, $\sigma_{yy} = \sigma_{zz}$. Les particules sont sollicités verticalement et développent une vitesse verticale. Les déformations générées sont essentiellement normales au plan de propagation $\varepsilon$. Cette hypothèse s'appuie sur la vitesse de l’impulsion. Ce phénomène étant trop rapide, le bulbe de déformation n'est pas encore formé. Donc, pour un ébranlement longitudinal plan, on aura uniquement trois variables $\sigma_{xx}$, $\sigma_{yy}$ et $\epsilon_{yy}$  principales décrivant le problème


\begin{figure}[H]
    \begin{center}
        \includegraphics[width=\linewidth]{media/ana_choc.PNG}
        \caption{(a) Onde longitudinale plane et surface de choc assumée plane (b) Conservation de la masse au passage d’une onde de choc}
        \label{fig:ana_choc}
  \end{center}
\end{figure}


%Au cours de l’essai DCLT, l’impact du marteau crée une onde incidente. Une partie de l’onde incidente de force $F_I$ et de vitesse $v_I$ est réfléchie. L'onde réfléchie de force $F_R$ se propage dans le sens inverse à une vitesse $v_R$, tandis qu’une autre partie de l'onde incidente est transmise au sol. Dans la tige, l’onde réfléchie s’ajoute à l’onde incidente. Tant que les deux milieux adjacents restent collés, leurs particules situées des deux côtes présentent la même vitesse et des contraintes normales de même intensité (principe d’action et réaction). Ceci constitue le principe général d’obtention des forces transmises ($F_T$) au sol pour chaque impact. La contrainte en pointe $\sigma_p = \frac{F_T}{A}$ (Figure \ref{fig:polaire1}).

\begin{figure}[H]
    \begin{center}
        \includegraphics[width=\linewidth]{media/polaire1.PNG}
        \caption{(a) ondes découplées et (b) polaire de choc de la tige et du sol}
        \label{fig:polaire1}
  \end{center}
\end{figure}

\[
{\begin{pmatrix}
\sigma_{xx}    &   \tau_{xy}     &   \tau_{xz}\\
\tau_{yx}      &   \sigma_{yy}   &   \tau_{yz}\\
\tau_{zx}      &   \tau_{zy}     &   \sigma_{zz}\\
            \end{pmatrix}
            }
    = f
{\begin{pmatrix}
\varepsilon_{xx}    &   \varepsilon_{xy}   &   \varepsilon_{xz}\\
\varepsilon_{yx}    &   \varepsilon_{yy}   &   \varepsilon_{yz}\\
\varepsilon_{zx}    &   \varepsilon_{zy}   &   \varepsilon_{zz}\\
            \end{pmatrix}
            }
\]


\begin{equation} \label{eq:deplacement}
    \varepsilon_y = \varepsilon_z = \frac{\partial \xi_x}{\partial x} = \frac{\partial \xi_y}{\partial y} = 0
\end{equation}



\[ \therefore
{\begin{pmatrix}
\sigma_{xx}    &   \tau_{xy}     &   \tau_{xz}\\
\tau_{yx}      &   \sigma_{yy}   &   \tau_{yz}\\
\tau_{zx}      &   \tau_{zy}     &   \sigma_{zz}\\
            \end{pmatrix}
            }
    = f
{\begin{pmatrix}
\varepsilon_{xx}    &   \varepsilon_{xy}   &   \varepsilon_{xz}\\
\varepsilon_{yx}    &   \varepsilon_{yy}   &   \varepsilon_{yz}\\
\varepsilon_{zx}    &   \varepsilon_{zy}   &   \varepsilon_{zz}\\
            \end{pmatrix}
            }
\]







L'exploitation de signaux de force et d'accélération mesurés au niveau de la tête permet d'estimer les contraintes et les vitesse en pointe, à l'interface sol-pointe \citep{Benz-Navarrete2009}. La vitesse d’ondes ($c_p^{DCLT}$) est déduite à partir de la vitesse particulaire en pointe ($v_p$), de la contrainte en pointe ($\sigma_p$) et de la masse volumique $\rho$ du milieu (Équation \ref{eq:celerite}) pour l'intervalle de temps [0 ; 2L/c].

\begin{equation} \label{eq:celerite}
    c_p^{DCLT} = \frac {\sigma_p}{\rho.v_p} 
\end{equation}

L'équation \ref{eq:celerite} est déduite à partir de la conservation de la masse et de la quantité d'énergie avant et après le passage de l'ébranlement. La déduction de cette relation est présentée dans le Rapport de 1er Année (en pièce-jointe) et également dans \cite{Aussedat1970,Meunier1974}. L'application de l'équation \ref{eq:celerite} s'appuie sur un certain nombre d'hypothèses pour cet intervalle de temps analysé [0 ; 2L/c] :
\begin{itemize} [label=\textbullet]
\item le contact entre le sol et la pointe pendant la transmission de l'onde est supposé plan et perpendiculaire ;
\item l'amortissement au long des tiges est négligeable ;
\item le contact entre le sol et la pointe pendant la transmission de l'onde est supposé plan et perpendiculaire ;
\item la déformation radiale dans le sol est négligeable.
\item la connaissance de la masse volumique du sol.
\end{itemize}


\cite{Benz-Navarrete2009} a appliqué la méthode proposée par \cite{Aussedat1970} à l'essai DCLT. Il a testé différents matériaux (béton, bois et différents sols) en laboratoire. Les polaires de choc obtenues ont été répétables et sensibles aux matériaux. Les valeurs de $c_p^{DCLT}$ obtenues ont des ordres de grandeurs conformes aux valeurs rapportées pour ces matériaux.

\begin{figure}[H]
    \begin{center}
        \includegraphics[scale = .7]{media/polaire.PNG}
        \caption{Polaires de choc expérimentales pour différents sols \citep{Benz-Navarrete2009}}
        \label{fig:polaire}
  \end{center}
\end{figure}


\begin{figure}[H]
    \begin{center}
        \includegraphics[scale = .7]{media/cp.PNG}
        \caption{Valeurs de $c_p^{DCLT}$ obtenues expérimentalement et valeurs typiques pour les matériaux testés \cite{Benz-Navarrete2009}. $^{(*)}$ \cite{Krause2015}, $^{(**)}$ \cite{Cassan1988}}
        \label{fig:cp}
  \end{center}
\end{figure}

\cite{Escobar2015} a également réalisé des comparatifs de $c_p^{DCLT}$ et des estimations de $c_p$ obtenus avec la méthode de surface MASW. Ces essais ont été importants pour montrer la faisabilité et répétabilité de la méthode. Bien qu'encouragents, ces comparatifs n'ont pas permis de valider la mesure car méthode MASW s’appuie sur la dispersion des ondes de surface et celle-ci est utilisée pour estimer le profil vertical du terrain, donc les résultats sont moins précis que ceux issus du DCLT. Par ailleurs, la méthode MASW dépend largement du modèle du terrain utilisé pour l’inversion des données. C'est pourquoi, on s'appuie par la suite sur les techniques plus précises, en laboratoire et permettant d'obtenir directement $c_p$.

\section{Comparatifs}

La vitesse d’ondes de compression dans le sol est fonction de divers facteurs (état de densité, hydrique et de contrainte, plasticité, ...). Pour les sables propres, les facteurs majeures sont : l’indice de vides (ou densité relative indique par l'acronyme DR), l’état de contrainte, l’état hydrique et les caractéristiques intrinsèques du matériau (tailles, forme des grains) \citep{Hardin1972}.

Pour l’état hydrique, il est permet de distinguer l'état saturé et le non saturés. Pour le différent état non saturés, c'est-à-dire degré de saturation variant entre 0\% et 95\%, ce facteur est secondaire car il n'engendre une variation maximale que de l’ordre de 10\% \citep{Emerson2005}. Par contre, au-delà de 95\%, $c_p$ croît brutalement vers la valeur de la vitesse d'ondes dans l'eau dépassant 1500 m/s. La capacité de $c_p$ de détecter la présence de zones saturés, c'est-à-dire distinguer les milieux quasi-saturés aux milieux saturés a été sujet de nombreux travaux portant sur l'évaluation du risque de liquéfaction des milieux granulaires \citep{Jamiolkowski2012, Tsukamoto2002, Bachrach1998a}.

Par ailleurs, afin de minimiser les effets intrinsèques liés aux matériaux, on travaille avec des sables de référence dont on dispose des résultats DCLT et des résultats $c_p$ rapportés.

\subsection{Données rapportées dans littérature pour les sables d'Hostun et Fontainebleau}

En laboratoire, la vitesse d'ondes de compression est déterminée l'aide de capteurs piézo-électriques de compression (méthode connue comme \emph{extenders elements}, méthode analogue de la méthode \emph{bender elements}) \citep{Duttine2005, Sauzeat2003}. Autre méthode d'obtention de la vitesse d'ondes de compression est la méthode \emph{cross-hole}. Celle-ci a été appliquée par \cite{Emerson2005} et par \cite{Emerson2006} en laboratoire pour les sables SH et SF. Les résultats présentés sont des déterminations obtenues en laboratoire pour des échantillons reconstitués par pluviation à sec.


\begin{figure}[H]
    \begin{center}
        \includegraphics[scale = .5]{media/cp_SF.PNG}
        \caption{Données rapportées dans la littérature de $c_p$ pour SF}
        \label{fig:cp_SF}
  \end{center}
\end{figure}

\begin{figure}[H]
    \begin{center}
        \includegraphics[scale = .45]{media/cp_SH.PNG}
        \caption{Données rapportées dans la littérature de $c_p$ pour SH}
        \label{fig:cp_SH}
  \end{center}
\end{figure}



\subsection{Données DCLT à Navier}

Les essais DCLT ont été réalisés en chambre d'étalonnage simulant les conditions au repos ($k_0$). Les éprouvettes ont été constituées par pluviation à sec. La chambre cylindrique permet de confectionner des éprouvettes de 730 mm de hauteur et de 547 mm de diamètre. Un système de pression hydraulique est installé en dessous du couvercle supérieur. Ce couvercle est doté d'une membrane en caoutchouc qui se dilate appliquant ainsi une pression verticale à l'échantillon. La pression appliquée est pilotée à l'aide d'une cellule de charge. Pour plus de détail sur sur le système s'adresser à \cite{LeKouby2008} et \cite{Muhammed2016}.

Les essais DCLT ont été réalisés dans l'axe de l'éprouvette en faisant varier la pression appliquée. Un nombre d'impact est réalisé (> 9) ensuite la pression est incrémentée, une nouvelle série d'impact est effectuée. La procédure continue pour toute la profondeur de l'éprouvette. Les pressions appliquées sont 10 kPa, 25 kPa, 50 kPa, 75 kPa, 100 kPa, 200 kPa, 300 kPa et 400 kPa \citep{LopezRetamales2020}.

\subsection{Comparatif des mesures}

Parmi des résultats de $c_p$ rapportés dans la littérature, on compare celles correspondantes aux conditions (DR et $\sigma_v$) ou le plus proche possible de celles des essais DCLT (Table \ref{tab:comparatifs_}). La Figure \ref{fig:comparatifs_graphique_} présente les données sous forme graphique.

\begin{table}[H]
    \begin{center}
        \includegraphics[width=\linewidth]{media/comparatifs_.PNG}
        \caption{Comparatifs pour le SF et SH}
        \label{tab:comparatifs_}
  \end{center}
\end{table}

On constate que les valeurs de $c_p^{DCLT}$ sont supérieures à celles de la littérature. La seule éprouvette pour laquelle la différence reste inférieure à 15\% étant SF à DR=50\% et contrainte de 50 kPa. On constate également que $c_p^{DCLT}$ est sensible à la densité. Pour les résultats SH à la contrainte de 50 kPa, on observe une augmentation de $c_p^{DCLT}$ avec la densité. L'écart-type moins important est celui de l'éprouvette de SH à DR de 70\% et à la contrainte verticale de 50 kPa. Cela est en accord avec les constatations d'\cite{Emerson2005}. Il relate de dispersion importantes les valeurs de $c_p$ obtenus à faibles contraintes (en surface).


\begin{figure}[H]
    \begin{center}
        \includegraphics[scale =.7]{media/comparatifs_graphique_.PNG}
        \caption{Comparatifs pour le SF et SH}
        \label{fig:comparatifs_graphique_}
  \end{center}
\end{figure}
  

\section{Bilan et perspectives}

Ce document présente des comparatifs entre les résultats de vitesse d'ondes de compression rapportés dans la littérature et ceux issus de l'essai DCLT. L'objectif était de vérifier dans quelle mesure $c_p^{DCLT}$ est comparable à $c_p$. 

Ces comparatifs se sont basés sur deux sables de référence, (SH et SF), secs. Pour les matériaux analysés, les principaux de facteurs influençant la vitesse d'ondes de compression sont l'état de densité et de contrainte. Ainsi, on compare les mesures pour lesquelles ces facteurs sont identiques ou, lorsque cela n'est pas possible, le plus proche.

Les comparatifs ont permis de constater que pour ce cas de figure les valeurs de $c_p^{DCLT}$ sont supérieures à $c_p$ pour des conditions semblables. Cette constatation est en accord avec les résultats obtenus par \cite{Escobar2015}. L'auteur avait rapporté des résultats élevés de $c_p^{DCLT}$ pour des valeurs supérieures à 300 m/s in situ. \cite{Escobar2015} a attribué cette différence à des perturbations dues aux raccourcissements élastiques des tiges (L$\geq$6m). Les comparatifs présentés dans ce document permettent de constater que la différence persiste pour des essais réalisés avec les tiges d'un mètre.

Toutefois, on a pu constater que $c_p^{DCLT}$ est sensible à la densité et au confinement des éprouvettes et que toutes les valeurs obtenues restent globalement dans la gamme attendue pour des sables sèches (Figure \ref{fig:cp}, \citep{Cassan1988}). 

Comme expliqué, la démarche appliquée pour déterminer $c_p^{DCLT}$ est celle des essais de choc. Traditionnellement, les essais de choc sont réalisés en laboratoire avec des éprouvettes de petite taille. \cite{Omidvar2012, IskanderMaguedOmidivar2015} relatent que la caractérisation de sols à partir des essais de choc est plus complexe que pour d'autres matériaux (comme les métaux par exemple). Pour des essais de choc pour les matériaux de vitesse d'ondes plus élevées une distribution de contraintes dans l'échantillon est plus uniforme. Dans le cas de matériaux granulaires comme les sables, la vitesse de propagation n'étant pas très élevée, parfois, l'échantillon n'atteint pas une distribution uniforme de contrainte pour des chargements rapides. D'après les auteurs cela constitue un aspect essentiel pour l'application de essais de choc \citep{Felice1986}.

Selon \cite{Lv2019}, dans le cadre des essais de choc, deux solutions sont envisagées pour minimiser ce problème : réduire l'épaisseur de l'échantillon ou modifier l'onde incidente afin d'augmenter le temps de chargement quasi-uniforme (technique de pulse-shaping).

D'ailleurs pour les matériaux peu cohésifs (comme les sables) les éprouvettes doivent être confinés avant le chargement dynamique. Souvent un tubage est utilisé afin d'assurer un confinement latéral. 

Étant donné les difficultés relatées pour les essais de choc dans les sols, il paraît pertinent d'étudier dans un premier temps l'influence de la forme de l'onde, temps de montée de l'onde (matériau et géométrie du marteau, ...). Dans un deuxième temps, l'influence de la taille de l'échantillon et des conditions aux limites afin de mieux répondre aux hypothèses initiales (déformations radiales négligeables, surface plane normale à la vitesse de propagation à l'interface sol-pointe).

Pour cela, le modèle numérique serait utile afin de mener une étude paramétrique initiale afin guider vers la solution plus économique capable de fournir des résultats attendus. La solution retenue devrait par le suite être validée expérimentalement.

Il serait pertinent également de compléter les résultats afin de proposer des comparatifs de mesures in situ. 


%%%________BIBLIO__________
\newpage
\lhead{Bibliographie}
\nocite{}
\bibliographystyle{apalike}
\bibliography{library}
  
\end{document}
