\documentclass{article}
\usepackage[utf8]{inputenc}
\usepackage[sectionbib]{natbib}
\usepackage[french]{babel}
\usepackage[titletoc]{appendix}
\usepackage{chapterbib}
\usepackage{url}
\usepackage{amsmath}
\usepackage{graphicx}
\usepackage{fancyhdr}
\usepackage{lmodern}
\usepackage{vmargin}
\usepackage[T1]{fontenc}
\usepackage{float}
\usepackage[table,xcdraw]{xcolor}
\usepackage{caption}
\usepackage{hyperref}
\usepackage{setspace}
%\usepackage{filecontents}
\usepackage{calc}
\usepackage{lipsum}
\usepackage{enumitem}
\graphicspath{ {images/} }
\usepackage{xcolor}
\usepackage{rotating}
\usepackage{gensymb}
\usepackage{siunitx}
\usepackage{textcomp}
\setcounter{secnumdepth}{3}

\pagestyle{fancy}
\fancyhf{}
%\lhead{\leftmark}
\rfoot{\thepage}

\begin{document}

\begin{titlepage}
    \begin{center}
    Thèse CIFRE N° 2018/1627
    
    Sol Solution
    
    Institut Pascal – Axe M3G
            \vspace*{5cm}
              %  \vspace*{5cm}
    
    %{Note technique}\\[0.3cm]
    
     {\textbf{Note technique : comparatifs $c_p$ - $c_p^{DCLT}$}}\\[1.cm]
     {Caroline F. Oliveira}\\[0.3cm]

     
    \end{center}
        \vspace*{8cm}
    
    
\textbf{Encadrement}
\begin{itemize}
	\item[] \textsc{Pierre BREUL}
	\item[] \textsc{Bastien CHEVALIER}
	\item[] \textsc{Miguel Angel BENZ NAVARRETE}
	\item[] \textsc{Quoc Anh TRAN}
\end{itemize}

        \vspace*{1cm}

\centering
\today

\end{titlepage}


\section{Introduction}

Le module de déformation est un des paramètres les plus difficiles à déterminer car il dépend de nombreux facteurs (granulométrie, densité, confinement, histoire de chargement, niveau de déformation …) \citep{Briaud2001}.

Par ailleurs, à partir des années 70, le développement de techniques expérimentales de plus en plus précises ont mis en évidence la limitation des essais classiques pour traduire le comportement des sols à petites déformations \citep{Burland1989,Atkinson2000, GomesCorreia2004}.

Ces travaux ont montré le domaine élastique pour les sols est assez restreint et difficile à déterminer (souvent associé à déformations $10^{-5}$). Les essais de propagations d’ondes (capteurs piezo-électriques en laboratoire et essais de forage in situ) induisent des déformations très faibles, inférieures à $10^{-5}$. L’hypothèse d’un comportement élastique linéaire est donc justifiée \citep{DanoChristophe;Hicher2018}. Il est donc possible d’obtenir le module à faibles déformations à partir des mesures de vitesse d’ondes et de la masse volumique du milieu $\rho$. Les essais de propagations d’ondes sont, toutefois, peu courant dans la pratique.

Cela induit à des estimations de déformations peu réalistes dans les dimensionnements et constitue une limitation à l'application des modèles de comportement plus poussés (alimentés par les paramètres à faibles déformations).

L'essai de chargement dynamique en pointe pénétrométrique (DCLT) au pénétromètre dynamique léger Panda 3$\up{\textregistered}$  est un technique simple et à faible coût. Lors des premiers instants suivant l'impact, cet essai est assimilable à un essai de choc \cite{Aussedat1970, Semblat2012, IskanderMaguedOmidivar2015}. Comme un essai de choc, l’exploitation des signaux de force et de vitesse dans le sol permet d’étudier le comportement à faibles déformations et d'estimer la vitesse d’ondes de compression ($c_p^{DCLT}$).

Il reste évaluer dans quelle mesure $c_p^{DCLT}$ correspondrait à la vitesse d'ondes de compression $c_p$ issue des techniques connues. Dans ce document on compare les résultats de $c_p^{DCLT}$ et $c_p$ sous des conditions semblables.

La vitesse d'ondes de compression issue de à l'essai DCLT étant par hypothèse de compression, les valeurs de $c_p$ utilisés sont issue techniques déterminant $c_p$ (pas de estimation à partir de $c_s$).

Les comparatifs se focalisent uniquement dans les résultats obtenus en laboratoire pour les sables propres de référence (sable d'Hostun HN31 et de Fontainebleau NE34). Le choix de ces sables permettent de profiter des différents résultats rapportés dans la littérature pour ces matériaux.

\subsection{Objectif}

Ce document a pour objectif de comparer les résultats de la vitesse d'ondes de compression issus de l'essai DCLT ($c_p^{DCLT}$)\footnote{On adopte la notation $c_p$ pour la vitesse d'ondes, la notation $v_p$ faisant référence à la vitesse particulaire} aux résultats issus d'autres techniques. On souhaite vérifier dans quelle mesure la vitesse d’ondes issue de l’essai DCLT correspond la vitesse d’ondes de compression $c_p$ et ainsi si celle-ci permettrait, comme dans le cas d'autres techniques, d’estimer le module à faibles déformations ($E_0$ ou $E_{max}$ = $\rho c_p^2$).

\subsection{RAPPEL : détermination de la vitesse d'ondes à partir de l'essai DCLT}

Dans les années 70, \cite{Aussedat1970} et \cite{Meunier1974} ont introduit l’application des essais de choc à l’essai de pénétration. Ils ont également réalisé des tests selon la méthode de barres de Hopkinson (Split Hopkinson Pressure Bar - SHPB) adaptés pour les sols. Ils se sont appuyés sur des principes des chocs et sur les équations d’état (équations de Rankine-Hugoniot), qui régit leur propagation. A l’aide des équations de la mécanique classique et des lois de comportement (équations rhéologiques), ces équations dites « mixtes » décrivent la relation entre les contraintes et les vitesses (nommé polaire de choc) lors de la propagation d’un ébranlement. Ces relations permettent d’étudier le comportement du sol soumis à un impact, permettant de déterminer l’impédance mécanique de celui-ci (Z) et d’en déduire la vitesse de propagation d’ondes de compression.

\cite{Omidvar2012} expliquent que les essais des propagations d’ondes en géotechnique (comme les essais de choc) sont complexes à mettre en place car il est souvent difficile d’avoir le suivi des contraintes dans le sol et de s’affranchir de réflexions des ondes. Dans ce type d’essai l’instrumentation la plus courante est l’insertion de capteur à l’intérieur de l’échantillon ce qui conduit souvent à des erreurs.

Cette limitation a été résolue avec l’instrumentation de la tige et l’application des méthodes découplage et de reconstruction d’ondes dans la tige appliqué à l'essai DCLT. Les capteurs situés dans au niveau de la tête enregistrent les contraintes engendrées par le passage de l’ébranlement et son retour de manière quasi-continue (50 kHz) pour un intervalle $\Delta t$ égal à 2L/c. Avec L la longueur de la barre et c la vitesse d’ondes dans l’acier (de l’ordre de 5200m/s).

Au cours de l’essai DCLT, l’impact du marteau crée une onde incidente. Une partie de l’onde incidente de force $F_I$ et de vitesse $v_I$ est réfléchie. L'onde réfléchie de force $F_R$ se propage dans le sens inverse à une vitesse $v_R$, tandis qu’une autre partie est transmise au sol. Dans la tige, l’onde réfléchie s’ajoute à l’onde incidente. Tant que les deux milieux adjacents restent collés, leurs particules situées des deux côtes présentent la même vitesse et des contraintes normales de même intensité (principe d’action et réaction). Ceci constitue le principe général d’obtention des forces transmises ($F_T$) au sol pour chaque impact et ainsi la contrainte en pointe $\sigma_p$ (\ref{fig:polaire1}).

\begin{figure}[H]
    \begin{center}
        \includegraphics[scale = .7]{media/polaire1.PNG}
        \caption{(a) ondes découplées et (b) polaire de choc de la tige et du sol}
        \label{fig:polaire1}
  \end{center}
\end{figure}

L'exploitation de signaux mesurés de force et accélération mesurés au niveau de la tête permet d'estimer les contraintes et les vitesse en pointe, à l'interface sol-pointe \citep{Benz-Navarrete2009}. La vitesse d’ondes ($c_p^{DCLT}$) est déduite à partir de la vitesse particulaire en pointe ($v_p$), de la contrainte en pointe ($\sigma_p$) et de la masse volumique $\rho$ du milieu (Équation \ref{eq:celerite}) pour l'intervalle de temps [0 ; 2L/c].

\begin{equation} \label{eq:celerite}
    c_p^{DCLT} = \frac {\sigma_p}{\rho.v_p} 
\end{equation}

L'équation \ref{eq:celerite} est déduite à partir de la conservation de la masse et de quantité d'énergie avant et après le passage de l'ébranlement. La déduction de cette relation est présentée dans le Rapport de 1er Année (en pièce-jointe) et également dans \cite{Aussedat1970,Meunier1974}. L'application de l'équation \ref{eq:celerite} requiert un certain nombre d'hypothèses pour cet intervalle de temps analysé [0 ; 2L/c] :
\begin{itemize} [label=\textbullet]
\item le contact entre le sol et la pointe pendant la transmission de l'onde est supposé plan et perpendiculaire ;
\item l'amortissement au long des tiges est négligeable ;
\item le contact entre le sol et la pointe pendant la transmission de l'onde est supposé plan et perpendiculaire ;
\item la déformation radiale dans le sol est négligeable.
\end{itemize}

\cite{Benz-Navarrete2009} a appliqué la méthode proposée par \cite{Aussedat1970}, à l'essai DCLT. Il a testé différents matériaux (béton, bois et différents sols) en laboratoire. Les polaires de choc obtenues sont répétables et sensibles aux matériaux. Les valeurs de $c_p^{DCLT}$ obtenus sont de l'ordre de grandeur conforme aux valeurs typiques rapportées pour ces matériaux.

\begin{figure}[H]
    \begin{center}
        \includegraphics[scale = .7]{media/polaire.PNG}
        \caption{(a) Polaires de choc expérimentales pour différents sols \citep{Benz-Navarrete2009} et (b) Valeurs de $c_p^{DCLT}$ obtenues et valeurs typiques pour les matériaux testés}
        \label{fig:polaire}
  \end{center}
\end{figure}


\begin{figure}[H]
    \begin{center}
        \includegraphics[scale = .7]{media/cp.PNG}
        \caption{Valeurs de $c_p^{DCLT}$ obtenues et valeurs typiques pour les matériaux testés \cite{Benz-Navarrete2009}. $^{(*)}$ \cite{Krause2015}, $^{(**)}$ \cite{Cassan1988}}
        \label{fig:cp}
  \end{center}
\end{figure}

\cite{Escobar2015} a également réalisé des comparatifs de ($c_p^{DCLT}$) et des estimations de $c_p$ à la partir de la méthode de surface MASW. Ces essais ont été importants pour montrer la faisabilité et répétabilité de la méthode. Bien qu'encouragents ces comparatifs n'ont pas permis de valider la mesure. La méthode MASW s’appuie sur la dispersion des ondes de surface. Celle-ci est utilisée pour estimer le profil vertical du terrain. A contrario, les mesures Panda 3 sont quasi- continues et possèdent un caractère ponctuel. En outre la précision de la mesure, la méthode MASW a comme limitation le fait de dépendre largement du modèle du terrain utilisé pour l’inversion des données

\section{Comparatifs}

La vitesse d’ondes de compression dans le sol est fonction de divers facteurs (état de densité, hydrique et de contrainte, plasticité, ...). Pour les sables propres, les facteurs majeures sont : l’indice de vides (ou DR), l’état de contrainte, l’état hydrique et les caractéristiques intrinsèques du matériau (tailles, forme des grains) \cite{Hardin1972}.

Quant à la l’état hydrique, ceci est important pour distingué l'état saturé du non saturés (S<95\%). Pour les sables l'état non saturé, ce facteur secondaire est secondaire car il n'engendre que des variations de l’ordre de 10\% \citep{Emerson2005}. Tous les comparatifs analysés dans ce document correspond à l'état non-saturé.

Afin de minimiser les effets intrinsèques liés aux matériaux, on travaille avec des sables de référence dont on dispose des résultats DCLT et des résultats $c_p$ rapportés.

\subsection{Données rapportées dans littérature pour les sables d'Hostun et Fontainebleau}

En laboratoire, la vitesse d'ondes de compression est déterminée l'aide de capteurs piézo-électriques de compression (méthode connue comme \emph{extenders elements}, méthode analogue de la méthode \emph{bender elements}) \citep{Duttine2005, Sauzeat2003}. Autre méthode d'obtention de la vitesse d'ondes de compression est la méthode \emph{cross-hole}. Celle-ci a été appliquée par \cite{Emerson2005} et par \cite{Emerson2006} en laboratoire avec les sables SH et SF. Les résultats présentés sont des déterminations obtenus en laboratoire pour des échantillons reconstitués par pluviation à sec.


\begin{figure}[H]
    \begin{center}
        \includegraphics[scale = .5]{media/cp_SF.PNG}
        \caption{Données rapportées dans la littérature de $c_p$ pour SF}
        \label{fig:cp_SF}
  \end{center}
\end{figure}

\begin{figure}[H]
    \begin{center}
        \includegraphics[scale = .45]{media/cp_SH.PNG}
        \caption{Données rapportées dans la littérature de $c_p$ pour SH}
        \label{fig:cp_SH}
  \end{center}
\end{figure}



\subsection{Données DCLT à Navier}

Les essais DCLT ont été réalisé en chambre d'étalonnage simulant les conditions au repos ($k_0$). Les éprouvettes ont été constitué par pluviation à sec. La chambre cylindrique permet de confectionner des éprouvettes de 730 mm de hauteur et de 547 de diamètre. Un système de pression hydraulique est installé en dessous du couvercle supérieur. Ce couvercle est doté d'une membrane en caoutchouc qui se dilate appliquant ainsi une pression verticale à l'échantillon. La pression appliquée est pilotée à l'aide d'une cellule de charge. Pour plus de détail sur sur le système s'adresser à \cite{LeKouby2008} et \cite{Muhammed2016}.

Les essais DCLT ont été réalisé dans l'axe de l'éprouvette en faisant varier la pression appliquée. Un nombre d'impact est réalisé   (> 9) ensuite la pression est incrémentée, une nouvelle série d'impact est effectuée. La procédure continue pour toute la profondeur de l'éprouvette. Les pressions appliquée sont 10 kPa, 25 kPa, 50 kPa, 75 kPa, 100 kPa, 200 kPa, 300 kPa et 400 kPa \citep{LopezRetamales2020}.

\subsection{Comparatif des mesures}

Parmi des résultats de $c_p$ rapportées dans la littérature, on compare ceux correspondant aux conditions (DR et $\sigma_v$) égales ou le plus proche possible de celles des essais DCLT (Table \ref{tab:comparatifs_}). La Figure \ref{fig:comparatifs_graphique_} présente les données sur la forme de graphe.


\begin{table}[H]
    \begin{center}
        \includegraphics[scale = .65]{media/comparatifs_.PNG}
        \caption{Comparatifs pour le SF et SH}
        \label{tab:comparatifs_}
  \end{center}
\end{table}

On constate que les valeurs de $c_p^{DCLT}$ sont supérieures à ceux figurant dans la littérature. La seule éprouvette pour laquelle la différence reste inférieure à 15\%.

\begin{figure}[H]
    \begin{center}
        \includegraphics[scale = 1.]{media/comparatifs_graphique_.PNG}
        \caption{Comparatifs pour le SF et SH}
        \label{fig:comparatifs_graphique_}
  \end{center}
\end{figure}

 
\section{Analyse globale des données DCLT}
 
 

\section{Bilan et perspectives}

Ce document présente des comparatifs entre les résultats de vitesse d'ondes de compression rapportés dans la littérature et ceux issus de l'essai DCLT. L'objectif était de vérifier dans quelle mesure $c_p^{DCLT}$ est comparable à $c_p$. 

Ces comparatifs se sont basés en deux sables de référence, (SH et SF), secs. Pour les matériaux analysés, les principaux de facteurs influençant la vitesse d'ondes de compression sont l'état de densité et de contrainte. Ainsi, on compare les mesures pour lesquels ces facteurs sont identiques ou, lorsque cela n'est pas possible, le plus proche.

Les comparatifs ont permis de constaté que pour ce cas de figure les valeurs de $c_p^{DCLT}$ sont supérieures à $c_p$. Cette constatation est en accord avec les résultats obtenus par \cite{Escobar2015}. L'auteur avait rapporté des résultats élevés de $c_p^{DCLT}$ pour des valeurs supérieures à 300 m/s in situ. \cite{Escobar2015} a attribué cette différence à des perturbations dues aux raccourcissements élastiques des tiges (L$\geq$6m). Les comparatifs présentés dans ce document permettent de constater que la différence persiste pour des essais réalisés avec les tiges d'un mètre.

Comme expliqué, la démarche appliquée pour déterminer $c_p^{DCLT}$ est celle des essais de choc. Les essais de choc sont réalisés en laboratoire avec des éprouvettes de petite taille. \cite{Omidvar2012, IskanderMaguedOmidivar2015} relatent que la caractérisation de sols à partir des essais de choc est plus complexe que pour d'autres matériaux (comme les métaux par exemple). Pour des essais de choc pour les matériaux de vitesse d'ondes plus élevés une distribution de contraintes dans l'échantillon est plus uniforme. Dans le cas de matériaux granulaires comme les sables, la vitesse de propagation n'étant pas très élevée, parfois, l'échantillon n'atteint pas une distribution uniforme de contrainte pour des chargements rapides. D'après les auteurs cela constitue un aspect essentiel pour l'application de essais de choc \citep{Felice1986}.

Selon \cite{Lv2019}, dans le cadre des essais de choc, deux solutions sont envisagées pour minimiser ce problème : réduire l'épaisseur de l'échantillon ou modifier l'onde incidente afin d'augmenter le temps de chargement quasi-uniforme (technique de pulse-shaping).

D'ailleurs pour les matériaux peu cohésifs (comme les sables) les éprouvettes doivent être confinés avant le chargement dynamique. Souvent un tubage est utilisé afin d'assurer un confinement latérale, même si certains travaux ont montré qui cela influe sur les résultats (Junyu et al., 2014; Lu et al., 2009; Song et al., 2009). D'autres travaux essaient de mettre en place un système triaxial afin de mieux étudier l'effet du confinement.

Étant donné les constations des essais de choc pour les sols, il serait d'utiliser d'étudier dans un premier temps l'influence de la forme de l'onde, temps de monte de l'onde (matériau et géométrie du marteau, ...). Dans un deuxième temps, l'influence de la taille de l'échantillon et des conditions aux limites afin de mieux répondre aux hypothèses initiales (déformations radiales négligeables, ).

Pour cela, le modèle numérique serait utile afin de mener une étude paramétrique initiale afin guider vers la solution plus économique capable de fournir des résultats attendus. Cette étude pourrait par le suite être valider expérimentalement.

%%%________BIBLIO__________
\newpage
\lhead{Bibliographie}
\nocite{}
\bibliographystyle{apalike}
\bibliography{library}
  
\end{document}
